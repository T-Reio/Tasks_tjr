\documentclass[11pt]{jsarticle}
\usepackage{mathpazo}
\usepackage{amsmath,amssymb}
\usepackage[top=25truemm,bottom=30truemm,left=25truemm,right=25truemm]{geometry}
\begin{document}

\title{産業組織論 レポート課題7}
\author{経済学研究科 23A18014 丹治伶峰}
\date{}
\maketitle

\large
Fowlie, Reguant and Ryan (2016) ``Market-Based Emissions Regulation and Industry Dynamics,'' \textit{Journal of Political Economy}

\normalsize

\section{目的}

温室効果ガスの排出規制が施行されている、かつ完全競争の仮定が満たされない市場の産業構造推定を行うための計量経済学的な方法を検討する。その上で、4つの排出規制政策に対する企業の意思決定の変化をシミュレーションし、それぞれによる社会的余剰の減少と、排出を削減することによる恩恵との変化を考察し、政策提言を行う。

\section{新規性}

貢献の一つに、動学モデルにおける構造推定の手法を、環境保全のための規制に対する産業の行動選択について応用した点がある。動学モデルにおける構造推定の手法の発達にも関わらず、同様の先行研究はほとんど行われてこなかった。また、先行研究で指摘されてきた、市場の寡占状態から生じる歪みの問題を伴うデータについて、外部性を内部化する政策の効果を識別する方法を示している。推定とシミュレーションによって得られた政策提言は、環境保護政策の効率性・公平性を推定し、適切な政策をデザインするための示唆を与えるとしている。

\section{産業の説明}

\paragraph{Industry}

Portlandのセメント産業は非常に寡占的な市場で、生産量上位の5企業が全生産量の54\%を占めている。需要は主にコンクリートの調合産業によるもので、その需要は経済状況や建設事業の有無によって周期的に変動する。また、コンクリートに代わる選択肢としては、アスファルトなどの原料が代替品として用いられている。

アメリカ国内に関しては、製品の陸路輸送に係るコストが大きいことから、生産工場が海岸線に近いエリアに散在する傾向がみられる。また同様の理由から、輸入によって賄われる需要量も大きい。海路輸送の技術革新から、アメリカのセメント輸入量は増加傾向にあり、1980年から2006年の間に、輸入が全体需要に占める割合は3\%から25\%にまで増加した。主な輸入先はカナダで、これにより統計には表れない温室効果ガスの排出を引き起こしている可能性が指摘されている。

セメント生産は、操業のために多量の温室効果ガスの排出を必要としており、NOx、SOxなどの指定物質に限れば発電設備に続いて2番目に多くの汚染物質を放出している産業である。また、$CO_2$の排出量は産業の中でも最も排出量が大きい。二酸化炭素の排出は、セメントの原料を合成するために行う化石燃料の燃焼によって起こる割合が最も高く、その燃焼効率は窯のタイプによって大きく異なる。

削減のための方策としては、古い窯を新しいものに交換することによる燃焼効率の改善、セメントに代わる別の建築資材への代替、燃焼工程に用いる燃料の変更による効率の改善が挙げられている。また、長期的な削減方法として、発生した二酸化炭素の分解技術の確立にも触れられている。

\paragraph{Market Regulation}

市場レベルでの排出削減の取り組みについて、この論文では4つの方法を挙げている。

まず前提として、分野を越えた国際的な排出量の上限設定を行う。この上限に基づいて、制限なく排出を行うことが出来る上限値(排出権)が与えられる。この排出権は、市場で自由に取引を行うことができるものとする。

その上で議論されるのが、排出権の分配をどのようにして行うかという問題である。この論文では、4つの方法について議論・推定を行う。

1つめは一様価格オークションによって分配を求める方法で、理論上は排出に対して課税を行う場合と同様になる。この方法にはstakeholderからの反発があるため、多くの施策で行われているのが2つめの無料配布方式である。この制度の下では、企業の過去の生産活動に基づき、無料で排出権の分配が行われる。3つめはgrandfatheringと呼ばれる方法で、全体の生産量をもとに一定期間ごとに排出権の分配量を更新していくものである。これにより、輸入によって起こる排出への寄与も考慮することが可能になる。4つめのBTAは、セメントの輸入に直接課税を行う方法で、同じく海外での排出代行を防ぐことが期待される。

\section{データ}

企業の特性と意思決定に関するデータはUS Geological Survey (USGS)とPortland Cement Association (PCA)から取得する。USGSはポートランドのセメント産業に属する全ての国内企業のデータを施設レベルで収集したデータセットで、これを窯レベルで公開しているのがPCAである。具体的には窯の所在地、窯のタイプ、原油と操業キャパシティに関するデータを持っている。また、企業の参入・退出、企業規模の変遷に関するデータもここから得られており、25年間における企業の参入退出、及び保有する窯の増減に関するデータを取得する。データの信頼性を担保するため、USGSのaggregated dataとの相互参照を行う。

市場の区分けについては、USGSではなく、EPAによるものを用いる。EPAのものの方が市場の区分けが詳細であるため、USGSから取得した生産量等に関するデータは、それぞれの市場に存在する窯の数に応じて分配したものを用いる。また、シミュレーションを行う際は、企業数が5以下の市場のみを対象にする。

このデータによると、市場内に存在する企業数や温室効果ガスの排出量には、市場間で隔たりがあった。また、市場に入ってくる輸入製品の量にも市場間の特性が見られ、海岸線に近いエリアの市場は輸入量が大きかった。

また、需要推定における操作変数として、US Energy Information AdministrationとUS Census Bureau's County Business Patternsから、弾力性、石油価格、天然ガス価格、賃金に関するデータを取得する。価格は2000年を基準とした実質価格を用いる。

\section{モデル}

\paragraph{静学モデル}

はじめに、筆者は静学モデルによって、貿易のある独占市場における企業の意思決定と、外部性の内部化がもたらす社会的余剰について分析している。ここから、環境外部性が生じる市場において政府が生産規制を行うと、(1)輸入品が市場に占める割合が増加することによる余剰の減少(rent leakage)(2)温室効果ガスの排出を抑制することによる余剰の増加(3)規制を受けない海外市場の排出増加による余剰の増加 の3つの影響が発生し、結果として社会的余剰が減少することを確認している(second-best)。

これを踏まえ、筆者は続いて生産者への補助金政策について評価している。補助金は国内生産者の供給増加を促し、結果外部性を内部化した状態における最適な価格を実現することができる。一方で、この方法は、国内生産者に比べて限界費用の高い輸入製品による供給を必要としているため、分配の効率性の問題は依然として残存する。この問題に対処するための方策を求めるのがこの論文の意図であると言える。

より信頼性の高いシミュレーションを行うための追加的な仮定として、上記のモデルを寡占市場に修正し、さらに動学的なモデルによる分析を行う。動学モデルにおいては、企業が生産や投資の計画を修正することができる一方で、不完全競争市場における排出規制を動学モデルで分析することで、これが引き起こす効率性の低下や、国内企業の退出行動が起こることによる輸入の増加などの悪影響を記述することが出来る。

\paragraph{動学モデル}

無限期間・離散時間(年刻み)における動学ゲームによる分析を行う。市場に存在する企業数の最大$\bar{N}$で、各企業が生産キャパシティに関するstate $s$、及び生産量あたりの温室効果ガスの排出量を規定するstate $e$に基づいて意思決定を行う。$e$は具体的にはその企業が持つ窯のタイプを表し、古いタイプから順にwet, dry, state-of-the-art dryの3つの値を取る。

各期における企業の意思決定は次の通り。

\begin{enumerate}
  \item 既存企業がその期の退出コストを元に、市場から退出するかどうかを決定する

  \item 企業が投資コストを観察する。市場に参入していない潜在参入企業は、さらに参入のためのコストも受け取る。

  \item 既存・潜在企業がcapacityへの投資・参入に関する意思決定を行う

  \item 既存企業がその期の生産量を決定する。
\end{enumerate}

退出を決めた企業もその期までは操業を行うものとする。また、capacityへの投資は実現までに1期分のラグがあると仮定している。

\subsection{Static Payoffs}

同質材市場$m$における、constant-elasticity aggregate demandは価格$P_m$によって

\[
\ln Q_m (P_m; \alpha) = \alpha_{0m} + \alpha_1 \ln P_m
\]

で与えられる。一方、輸入製品の供給は

\[
\ln M_m (P_m; \rho) = \rho_0 + \rho_1 P_m
\]

で与えられ、国内企業への需要は、両者の差によって定義される。各期において、企業はその期のstateを元に、静学的な利潤$\bar{\pi}$を最大化する生産量$q_i$を選択する。

\[
\bar{\pi}(s, e, \tau; \alpha, \rho, \delta) \equiv \max_{q_i \leq s_i} P \left( q_i + \sum_{j \neq i} q^*_j ; \alpha, \rho \right) q_i - C_i (q_i; \delta) - \phi(q_i, e, \tau)
\]

$C(q_i; \delta) = \delta_{i1}q_i + \delta 1 (q_i > v s_i) (q_i / s_i - v)^2$は生産に係るコスト関数で、生産量に関わらず一定のコストを表す第一項と、キャパシティの限界に近づくにつれて逓増する第二項に分解される。一方、$\phi(.,.,.)$は環境規制によって生じるコストであり、この項の表現形によってシミュレーションを行う政策を識別することが出来る。$\tau$は二酸化炭素の排出コストを規定する外生変数で、この論文では技術革新による変動を考慮せず、分析を行う全ての期間において一定であると仮定する。

それぞれの政策における$\phi(.,.,.)$は以下のように表現される。

\begin{enumerate}
  \item Emissions tax or emissions trading with auctioned permits

  \[
  \phi(q_i, e_i, \tau) = \tau e_i q_i
  \]

  \item Grandfathering

  \[
  \phi(q_i, e_i, \tau) = \tau(e_i q_i - A_i)
  \]

  \item Output-based allocation updating/rebating

  \[
  \phi(q_i, e_i, \tau) = \tau \cdot (e_i - \Psi_d) \cdot q_i
  \]

  \item Border tax adjustment with auctioned permits

  \[
  \phi(q_i, e_i, \tau) = \phi(q_i, e_i, \tau) = \tau e_i q_i
  \]
\end{enumerate}

\begin{enumerate}
  \item

  \item

  \item

  \item BTA

  \[
  \ln M (P; \rho, \tau) = \rho_0 + \rho_1 \ln (P - \tau e_M)
  \]
\end{enumerate}

\section{推定方法}

\section{結果}



\section{コメント}



\end{document}
