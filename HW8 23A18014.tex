\documentclass[11pt]{jsarticle}
\usepackage{mathpazo}
\usepackage{amsmath,amssymb}
\usepackage[top=25truemm,bottom=30truemm,left=25truemm,right=25truemm]{geometry}
\begin{document}

\title{産業組織論 レポート課題7}
\author{経済学研究科 23A18014 丹治伶峰}
\date{}
\maketitle

\large
Kalouptsidi (2014) ``Time to Build and Fluctuations in Bulk Shipping'' \textit{American Economic Review}

\normalsize

\section{目的}

需要の不確実性が企業レベルの不可逆的、かつ調整費用が伴う投資に関する意思決定、および価格に与える影響を、海運産業のデータを用いた動学参入・退出ゲームのモデル分析を用いて検討する。

\section{新規性}

この研究は2つの研究の流れを汲んでいる。1点目は企業の投資に関する意思決定において不可逆性・調整費用・不確実性が伴う場合のモデルについての研究である。投資に関する意思決定を行ってから実際に操業を行うことが可能な状態に移行するまでの時差を仮定するモデルの重要性はKydland and Prescott (1982) で指摘されており、この点についての実証分析を行ったことが第一の貢献である。2点目は産業の動学分析の流れであり、この論文では参入・退出行動よりも先に挙げたような市場の特性の影響を検討することに重きを置いている。

\section{産業の説明}

Bulk Shipping とは、特定の貨物の輸送を、定期便が就航していない航路に対して提供するサービスである。供給を担うShipownerは全体の3\%の船を保有するものが最大で、数多くの企業がそれぞれ少数の船を保有して経営している状況である。ここでは、各企業が保有しているそれぞれの船が、それぞれの航路を一市場として同質材を提供しているとみなしている。

産業に対する需要量の変化には世界経済の情勢が影響しており、特に原料製品の輸入量は大きな影響を与える。一方、サービスの供給量は短期変動と長期変動とで影響する要因が異なる。短期的にはその市場を利用する輸送の回数が供給量を示すが、長期的には新たな貨物船の製造を伴う参入、船の売却を伴う市場からの退出に関する意思決定が影響する。また、新たな貨物船の製造には時間を要するため、参入の意思決定から実際に操業を行うまでには時差が生じる。そのため、参入を検討している企業はその時点での需要量を観察して意思決定を行うことが出来ず、需要の不確実性に直面することになる。また、一旦参入の意思決定を行って、あとからそれをキャンセルすることはきわめてまれである。

\section{データ}

イギリスに拠点を置くブローカー企業の4つの調査を利用する。一つ目は中古の貨物船の譲渡契約のデータで、日付、操業期間、船舶のサイズなどの属性データと売却価格を含んでいる。データは1998年8月から2010年6月までの48四半期に起こった1838の契約を記載している。二つ目は貨物輸送の契約に関するデータで、2001年1月から2010年6月までに結ばれた輸送契約について、使用された船舶の情報とサービスの価格に関する情報を含んでいる。最も小さなタイプの船舶については、そのすべての航行を追跡することが不可能であるため、すべてを記録することでできている2企業のデータを用いる。また、このタイプの船舶が行った輸送については、輸送に掛かる時間と距離のみのデータが得られていることに注意する。三つ目は貨物船の発注・完成・取り壊しと、各期に存在する船舶の数、および納品前の船舶数のデータである。市場に存在する船舶の操業年数の分布はこのデータに基づく。操業年数は四半期区切りで設定し、20年を超えるものは同質として扱う。四つ目は制作中の船舶に関するデータで、納品前の船舶数についてのstateを算出する際に用いる。

\section{モデル}

\subsection{Environment}

船舶一隻が一つの企業であると仮定した、離散時間・無限期間の動学参入・退出ゲームを考える。企業は対称な潜在参入者と既存企業の二つのタイプに分類される。また、船舶を制作するための期間が必要であることから、企業は参入の意思決定を行ってから実際に操業を始めるまでの間に数期間のラグが生じることを考える。

状態変数は(1)船舶の累計操業期間$j \in \{ 0, \ldots, A\} $、(2) $j$の分布$\mathbf{s}_t \in S \subset \mathbb{R}^{A+1}$ ($\mathbf{s}$はperiod $t$に市場に存在する age $i$の企業数$s_t^i$)、(3)参入を決定した企業で、まだ操業を開始していないものの数$\mathbf{b}_t \in B \subset \mathbb{R}^{\bar{T}}$ ($ \mathbf{b}_t$は period $t$時点で$i$期後に操業を開始する企業数$b_t^i$)、そして貨物輸送船サービスに対する需要$d_t$によって定義される。また、価格$P_t$は逆需要関数$P_t = P(d_t, Q_t)$によって定義される。$P$はサービスの価格、$Q_t$は期間内に行われた輸送の総数である。age $j$の企業の利潤は、市場の競争状態について言及しない誘導形関数$\pi_j(\mathbf{s}_t, d_t)$で表され、いずれの$(\mathbf{s}_t, d_t)$の組み合わせについても、$j$に関してnonincreaseingである。

また、操業中の企業は、periodごとにそのscrap value $\phi$を(他企業・研究者からは観察できない変数として)観察し、市場から退出するかどうかの意思決定を行うと仮定する。$\phi$の分布は時間、企業とは独立に$F_{\phi}$に従う。

市場への参入は、その参入コスト$\kappa(\mathbf{s}_t, \mathbf{b}_t, d_t)$以外に障壁が存在しない、自由参入・退出構造を仮定する。また、参入を決定した場合に必要になる待機期間は状態変数$T:(S \times B \times D)$によって1から$\bar{T}$の間に定まり、同じperiodに参入した企業は同じ待期期間を要求されるものとする。

この論文では船舶そのものと企業を同一視するため、他企業が中古の船舶を買い取って操業を始める場合、企業ごとの特性は存在せず、所有者が替わっても企業は変わらず操業しているとみなす。

\subsection{Firm Behavior}

age $j$ の既存企業は、それぞれexpected discounted stream of profitを最大化することを目指す。企業は操業した時の利潤と退出する際のscrap valueから、参入するか退出するかを選択するため、value functionは

\[
V_j(\mathbf{s}_t, \mathbf{b}_t, d_t) = \pi_j(\mathbf{s}_t, d_t) + \beta E_{\phi} \max \{ \phi, VC_j(\mathbf{s}_t, \mathbf{b}_t, d_t) \}
\]

$VC$は操業を続けた時のvalueで

\[
VC_j(\mathbf{s}_t, \mathbf{b}_t, d_t) \equiv E[V_{j+1}(\mathbf{s}_t, \mathbf{b}_t, d_t) | \mathbf{s}_t, \mathbf{b}_t, d_t]
\]

$t$期に参入・退出する企業はそれぞれ$N_t, Z_t^j$で定義される。ここから、$\mathbf{s}_t$、$\mathbf{b}_t$はそれぞれ、以下のように定義される。

\begin{align*}
  s_{t+1}^0 &= b_{t}^1 & \\
  s_{t+1}^j &= s_t^{j-1} - Z_t^j, & j = 1, \ldots, A
\end{align*}

\begin{align*}
  b_{t+1}^i &= b_t^{i+1}, & i \neq T_t, \bar{T} \\
  b_{t+1}^T &= b_t^{T+1} + N_t, b_{t+1}^T = 0, & \text{ if }T_t < \bar{T}\\
  b_{t+1}^{\bar{T}} &= b_t^{\bar{T}} + N_t, & \text{ if }T_t = \bar{T}
\end{align*}

既存企業はその期のVCが、そのscrap value $\phi$を上回った時に退出を行う。その可能性は$\zeta_j (\mathbf{s}_t, \mathbf{b}_t d_t) \equiv \text{Pr}(\phi > VC_j(\mathbf{s}_t, \mathbf{b}_t, d_t)) = 1 - F_{\phi}(VC_j(\mathbf{s}_t, \mathbf{b}_t, d_t))$で定義され、したがって各期の退出企業数$Z_t$はパラメータ$\zeta$の二項分布に従う。

一方、潜在参入者にとって、参入した時のvalueは$VE(\mathbf{s}_t, \mathbf{b}_t d_t) \equiv \beta^{T_t}E[V_0(\mathbf{s}_{t+T_t}, \mathbf{b}_{t+T_t}, d_t)]$で与えられる。すなわち、参入を検討する際には、操業を開始するまでに起こる参入・退出・需要量についての予測を立てる必要がある。この論文では、参入企業数$N_t$は、状態変数によって定まる$\lambda(\mathbf{s}_t, \mathbf{b}_t, d_t))$をパラメータとするポアソン分布に従うとみなす。状態変数$(\mathbf{s}_t, \mathbf{b}_t, d_t)$を$\mathbf{x}$に置き換えると、value functionは

\begin{align*}
  V_j(\mathbf{x}; \zeta' \zeta, \lambda) = & \pi_j (\mathbf{s}_t, d_t) + \zeta' \beta E(\phi | \phi > VC_j(\mathbf{x}; \zeta, \lambda)) (1-\zeta') \beta VC_j (\mathbf{x}; \zeta, \lambda)
\end{align*}
$\zeta'$は企業が市場から退出した際に1を取るダミー変数である。

マルコフ均衡は退出と参入について、次の条件を満たす。

\begin{enumerate}
  \item 既存企業

  \begin{align*}
    \sup_{\zeta_j' (\mathbf{x} \in [0,1])} & V_j(\mathbf{x}; \zeta' \zeta, \lambda) = V_j(\mathbf{x}; \zeta' \zeta, \lambda), \\
    & \text{ all } j \in \{0, 1, \ldots, A \} \mathbf{x} \in (S \times B \times D)
  \end{align*}

  \item 参入者にとっての参入コストと参入した際の割引期待利潤との関係

  \[
  VE(\mathbf{x}) \leq \kappa (\mathbf{x})
  \]

  $\lambda > 0$、すなわち正の参入企業数が期待されるときのみ、等号が成立する。
\end{enumerate}

\section{推定方法}

この論文では、中古市場での船舶の売却価格のデータを利用して、通常観察が困難である$VC$の値を推定する。各企業が退出する際は、$VC$がそのscrap value である$\phi$を下回るが、ここでは企業が退出を行った期の$VC$は$\phi$に等しいという仮定を置く。ここから導出されたvalue functionと推定された state transition function から、参入コスト、scrap value の分布、利潤関数の推定に必要な情報を得る。

推定は2段階に分かれて行われる。1段階目はvalue functionとtransition matrixの推定、併せて動学均衡から定めることが出来ない外生的な要因である、需要関数とそのtransition、参入を決定してから実際に操業を行うまでのラグを設定するtime to build functionの推定を含む。2段階目では、ここから利潤関数、scrap valueの分布関数、参入コストの推定を行う。

\subsection{Exogeneous and Equilibrium Objects}

\begin{itemize}
  \item 需要

  逆需要関数は、標準的なlog-linear funtionを仮定する。すなわち

  \[
  P_t = D_t Q_t^{\eta}
  \]
  $d_t = \log D_t$である。この時、需要曲線は

  \[
  \log (P_t) = \eta_0 + \eta_1 \log (\mathbf{H}_t) + \eta \log (Q_t) + \epsilon_t
  \]

  で記述される。$\mathbf{P}_t$は$t$期のサービス価格、$\mathbf{H}_t$はdemand shifter、$Q_t$は実現した輸送委託契約の総数である。$\mathbf{H}_t$にはOECDから取得した世界産業生産(WIP)とUNCTADから取得した中国の鉄鋼生産、最も小さなタイプの船舶数、食品の価格指数、第一次産品の価格、鉱石価格を用いる。推定には市場に存在する企業数、その累積操業年数の分布を操作変数とする二段階最小二乗法を利用する。推定の結果、2段階目の推定ではWIPと中国の鉄鋼生産が価格に正の影響を、$\mathbf{Q}_t$が負の影響を持つという結果が得られた。ここから、状態変数$d_t$は逆需要関数の切片
  \[
  d_t = \hat{\eta_0} + \hat{\eta_1} \log (\mathbf{H}_t) + \hat{\epsilon}
  \]
  として記述される。

  \item demand state transition

  $d_t$は一階の自己回帰モデルに従って決定されるとみなす。すなわち

  \[
  d_t = c + \rho d_{t-1} + \epsilon_{t}
  \]
  $\epsilon$の分布はその分散が未知のため、ガンマ分布に従うと仮定する。

  \[
  f_{\epsilon}(\epsilon) = \dfrac{\Gamma(\alpha + \frac{1}{2})}{\gamma^{\alpha} \sqrt{2 \pi} \Gamma(\alpha)} \left( \dfrac{2 \gamma}{2 + \gamma \epsilon^2} \right)^{\alpha + \frac{1}{2}}
  \]

  これらのパラメータ$(c, \rho, \alpha, \gamma)$を最尤法で推定する。推定された$\rho$の値は0.954で、各四半期間で需要が高い自己相関を示すことが分かった。また、$\epsilon$の分布について正規分布を仮定する方法は、尤度比検定から棄却された。

  \item value function

  市場には同質の企業と潜在的な参入者が豊富に存在するため、市場から退出する際の船舶の売却価格はその価値と一致する。すなわち、状態$\mathbf{x}$におけるage $j$の企業の売却価格は

  \[
  p^{SH} = V_j(\mathbf{s}, \mathbf{b}, d) + \epsilon
  \]
  で与えられる。$V$はそれぞれのstateにおける売却価格の条件付き期待値

  \[
  V_j(\mathbf{s}, \mathbf{b}, d) = E[p^{SH} | j, \mathbf{s}, \mathbf{b}, d]
  \]
  と表現される。ここで、$\mathbf{s}$はage $i = 0, \ldots, 80$の船舶の数$s_t^i$、$mathbf{b}$は$t=0, \ldots, 11$期後に操業する企業数を表現する。変数の煩雑さからそのままでは扱うことが容易でないため、ここでは、$s_t^i$をageが10年以内、20年以内、それ以上の3つのグループの企業数の合計をそれぞれ示す$\mathbf{S}_t = (S_t^1, S_t^2, S_t^3)$に、また$b_t^i$を$B_t = \sum_{i=1}^T b_t^i$の一つに集約し、合計5つのパラメータに基づく状態変数$\mathbf{X} = [S_t^1, S_t^2, S_t^3, B_t, d_t]$として推定を行う。

  value functionの推定には、局所線形近似の手法を用いる。具体的には、それぞれのstate $[j, \mathbf{X}]$における誤差を、該当するポイントにおける状態への荷重を下げる形で最小化する目的関数を定義する。

  \begin{align*}
    \min_{\beta_0, \beta_j, \beta_S, \beta_B, \beta_d} \sum_i & \{p^{SH}_i -\beta_0(j, \mathbf{X}) - \beta_j (j, \mathbf{X})(j_i - j) - \beta_S(j, \mathbf{X})(\mathbf{S}_i - \mathbf{S}) \\
    & - \beta_B (j, \mathbf{X})(B_i - B) - \beta_d (j, \mathbf{X})(d_i - d) \}^2 K_h ([j, \mathbf{X}] - [j, \mathbf{X}])
  \end{align*}
  $K(.)$は正規カーネル関数である。

  推定の結果、当初の仮定通り、value functionはいずれの$\mathbf{X}$についても企業のageに関してnonincreasingであると推定された。

  推定に当たって考えられる問題は、船舶の故障・事故歴が分からないことによる逆淘汰の問題と、取引に耐えうる品質の船舶のみが市場に出回ることで起こるsample selectionの問題である。これらについて筆者は、船舶の故障・事故に関する情報がpublic imformationとして知られていること、中古市場で取引される船舶の数が総数に対して大きな割合を占めていることをもってこれらの問題が起こる可能性を否定している。

  \item time to build function

  参入決定から操業を開始するまでの時間は状態変数$(\mathbf{s}_t, \mathbf{b}_t, d_t)$に依存する。前項と同様、煩雑さを避けるため、これを$(\mathbf{s}, \mathbf{b}, d)$として推定を行う。

  \[
  \log (T_t) = a_0^{TTB} + a_D^{TTB} \mathbf{S}_t + a_B^{TTB} B_t + a_d d_t
  \]

  \item transition matrix

  $t+1$期における状態変数$\mathbf{x}_{t+1}$は、$t$期の状態$\mathbf{x}_t$に依存する参入・退出企業数$N_t, Z_t$の条件付確率密度関数によって次のように定義される。

  \[
  \text{Pr}(\mathbf{x}_{t+1} | \mathbf{x}_t) = Q_(\mathbf{x}_t, \mathbf{x}_{t+1}) f_N (N_t = n | \mathbf{x}_t) f_Z(Z_t = z | \mathbf{x}_t)f_{\epsilon}(\epsilon)
  \]
  ただし

  \[
  Q(\mathbf{x}_t, \mathbf{x}_{t+1}) = \delta \{s_{t+1}^0 = b_t^1 \} \Pi_{i=1}^{A-1} \delta \{s_{t+1}^i = s_t^{i-1} \} \Pi_{i \neq T(\mathbf{x}_t)} \delta \{b_{t+1}^i = b_t^{i+1}
  \]

  \begin{align*}
    n &= b_t^{T(\mathbf{x}_t)} - b_t^{T(\mathbf{x}_t) +1} \\
    z &= s_t^A + s_t^{A-1} - s_{t+1}^A \\
    e &= d_{t+1} - c - \rho d_t
  \end{align*}

  これまでと同様に状態変数には$\mathbf{x}$ではなく$\mathbf{X}$を用いるため、それぞれについてtransition probabilityを推定する必要がある。$f_N$及び$f_Z$はそれぞれパラメータ$lambda(\mathbf{X}_t)$および$\mu(\mathbf{X}_t)$に従うポアソン分布で、パラメータは状態変数を説明変数とするOLSによって推定する。

  推定の結果、参入は

  \begin{itemize}
    \item 需要に対して単調増加

    \item 既存企業数に対して単調減少
  \end{itemize}

  することが分かった。特に既存企業数の影響は、ageの若い企業のものが大きく、これらの企業の競争力が大きいとする仮説と整合的な結果であった。
\end{itemize}


\subsection{Model Primitives}

割引率$\beta$の値は年率5\%に等しい$\beta = .9877$に設定する。

\begin{itemize}
  \item scrap value distribution

  第一段階で推定したtransition matrixから、操業を継続することのvalueは

  \begin{align*}
    VC_j &= PV_{j+1} & j = 0, \ldots, A-1 \\
    VC_A &= PV_A
  \end{align*}

  age $A$に達した船舶の$V$は$V_A$で固定される。この値を用いて、scrap value distributionを局所線形近似で推定する。退出の確率は

  \[
  \dfrac{Z}{s^A} = [1-F_{\phi}(v)]
  \]
  で与えられ、近似の目的関数は

  \[
  \min_{\beta_0, \beta_{VC}} \sum_t \left\{ dfrac{Z_t}{s_t^A} - \beta_0(v) - \beta_{VC}(v)(VC_A(\mathbf{X}) - v) \right\}^2 K_h(VC_A (\mathbf{X} - v))
  \]

  推定結果は代替的なBajari, Benkard, and Levin (2007)の方法で推定された値と比較しても統計的な差異が認められないものであった。

  \item profit

  利潤関数はBellman Equationから

  \[
  \pi_j = V_j - \beta VC_j
  \]
  で与えられる。ageが$A$に達した企業は退出を検討するので

  \[
  \pi_A = V_A - \beta E_{\phi} \max \{\phi, VC_A \}
  \]

  利潤は2007年頃まで0近傍を上下しているが、この期間に大きく上昇し、続いて下落している。この傾向は経営者が状況の改善による利潤の上昇を狙って退出の意思決定が遅れる(hysterisis)性向によって生まれる。投資は基本的に不可逆的で、一度退出した企業は再び市場に戻ることが出来ない。

  \item entry cost

  free entry conditionの下で、均衡状態での参入コストは操業初期のvalueに等しいので

  \[
  \beta^{T(X)}P_X^{T(X)}V_0 = \kappa(X)
  \]
  が成立する。ここから、参入コストを推定することが出来る。推定の結果、参入コストは船舶の製造価格にほぼ等しく、企業が市場に参入するためのコストは、事業に使用する船舶の製造費用のみによるとされた。
\end{itemize}

\section{結果}

上記の推定方法に基づいて導出されたパラメータ・関数から、参入から操業までのタイムラグが企業の意思決定にどのように影響を与えるかをシミュレーションする。具体的には、創業までのラグが状態変数によらず一定である場合、及び参入の意思決定を行ったその期から操業することが出来る場合の2つの状況を仮定し、シミュレーションを行った。

\begin{enumerate}
  \item perfect competition profits

  価格と企業の利潤を推定するために、市場の競争形態を特定する必要がある。この論文では、完全競争モデルを仮定する。仮定の頑健性の検討のため、Cournot競争を仮定したモデルでのシミュレーションを行ったが、均衡価格は限界費用に等しいものであった。

  まず、推定された需要関数を元に、企業の固定費用と可変費用を推定する。推定の結果、それぞれ企業のageについて単調増加的な費用が推定された。推定された費用は先行研究とも整合的なものであった。

  \item counterfactual results

  先に挙げた2つの状況についてそれぞれシミュレーションを行い、タイムラグによって起こる需要の不確実性が意思決定に起こる影響を推定する。完全競争均衡における利潤関数、scrap value distribution、参入費用から、参入・退出のパラメータ$\lambda, \mu$とvalue functionを求める。

  \begin{enumerate}
    \item impulse respoonse

    需要に正の影響を及ぼすイベントが起こった際、供給はTTBのために、すぐにそれに追い付くことがない。このため、既存企業は価格を引き上げ、消費者余剰の改善分を奪って自らの利潤を改善することが出来る。従って、仮にTTBが減少、あるいは存在しない場合は、供給弾力性は改善され、価格・および企業の利潤は低下することになる。また、需要に正の影響があった場合、参入する企業数は増加する。しかしこの場合にも、実際には参入から操業までにラ
    が存在しているため、企業はその需要増加の恩恵を充分に享受することが出来ない。TTBが存在しない場合をシミュレーションすると、参入企業数は上昇した。

    また、TTBが存在しない場合、新規参入企業はageの若い船舶ですぐに操業を開始することが出来るため、累積操業年数が大きな企業はすぐに退出するようになる。TTBが状態変数によらず一定の場合は、新規参入者の参入のタイミングが予測できるため、彼らの操業が実際に起こるまで市場に残り、操業を始めると一斉に退出するようになる。結果、サービスの対価は低下し、消費者余剰も改善される。

    \item long-run simulations

    次に、TTBが参入のための投資とその費用に及ぼす影響を考察する。

    TTBが減少すると、参入企業数、及び市場に存在する企業数はともに上昇する。その結果、参入のための費用・および輸送サービスの価格は下落する。また、市場に存在する企業のageは低下する。

    特に大きな影響がみられたのは投資のvolatilityで、船舶数の分散は、TTB一定の場合は元の条件に比べて45\%、TTBなしの場合は同じく二倍にまで拡大した。同様に、参入企業数の分散も拡大した。一方で、価格の分散は元の状況よりも縮小した。

    \end{enumerate}

    シミュレーションの結果、TTBを減少させることは、需要の変動に対する供給の即時的な反応を可能にし、同時に退出する企業数をも増加させるものの、参入企業数の増加を実現し、元の状況で生じていた供給のラグを改善することが分かった。一方で、国際貿易・経済情勢に影響されて起こる需要の不確実性は今回のシミュレーションでは導入することができなかったとしている。

    TTBは政府による補助金の支給によって疑似的に変化させることが可能で、実際に東アジアではこうした政策が導入されている。この論文のシミュレーションの結果、補助金政策は第一段階である投資の意思決定だけでなく、サービスの供給量や企業の利潤にも影響を与えることが分かった。

    最後に、筆者は参入企業が大きく増加すると、一定期間後に再び参入企業数が大きく増加するecho effectについて言及している。TTBが存在しない状況下では、その他の条件下で観察されていた市場に存在する企業数の振動の幅は小さくなり、echo effectはより短いスパンで生じるようになる。

\end{enumerate}

\section{コメント}

この論文では、参入から操業までにタイムラグが存在する場合の企業の意思決定についての分析を行った。海運産業に限らず、参入に当たって必要な初期投資コストが大きく、操業開始までに長い待機期間・あるいは審査期間を要する業界についての分析への応用が考えられる。また、寡占市場への参入には参入障壁を突破するための時間的コストの存在も考慮する必要があるため、こうした業界についても、同様の手法を応用することが可能であると考えられる。

\end{document}
