\documentclass[11pt]{jsarticle}
\usepackage{mathpazo}
\usepackage{amsmath,amssymb}
\usepackage[top=25truemm,bottom=30truemm,left=25truemm,right=25truemm]{geometry}
\begin{document}

\title{経済哲学 期末レポート}
\author{23A18014 丹治伶峰}
\date{2019/2/8 提出}
\maketitle

\section*{認識論と存在論との同居可能性に関する考察}

\section{}

ヒュームによる「実体」「原因」の追放に始まる「本質」に対する姿勢の二分化

実際は両者は同一の方向を持った哲学: 物事の本質に辿り着かんとする

「認知されるものが本質」「認知したものの本質は存在しない」

今一般に言われる"経済学"も、実在論と矛盾することなく社会事象の本質を追究することができるのでは

\section{実証主義と形而上学}

\section{素粒子論と社会存在論}

\section{心身二元論}

\section{確率とバブル}




\end{document}
