\documentclass{jsarticle}
\usepackage{mathpazo}
\usepackage{amsmath,amssymb}
\begin{document}

\title{産業組織論 レポート課題3}
\author{経済学研究科 23A18014 丹治伶峰}
\date{}
\maketitle

Doi \& Ohashi (forthcoming)

\begin{enumerate}

\item 目的

2001年に日本の航空産業で起こった合併による総余剰への影響を、市場のモデルを推定することによって検証する。

\item 新規性

旅客輸送業界の構造推定にあたって、企業が価格だけでなく、Flight Frequency をコントロールできる状態をモデルに組み込んでいる。

\item データ、産業

2002年に起こったJAL (Japan Airline) と JAS (Japan Air System) の合併を利用。

当時の航空産業は、2001年9月に起こった同時多発テロの影響により、世界的な不振に陥っていた。テロの発生は予測可能であるとは言えず、このわずか2ケ月後に起こったこの合併は、産業に対する外生的なショックであると考えられる。

データは2000年4月から2005年10月の期間内に集計された3ケ月毎の統計をパネルデータとして利用(Market level data)。2000年は改正航空法が施行された年で、それ以前の期間には価格や参入について規制が存在していたこと、またJJ mergerの決行による市場への影響はJFTCによって合併から3年間、注意深く観察されており、この期間中のデータの入手が容易であったことから、この集計期間が選択されている。

搭乗価格はTravel Survey for Domestic Air Passengers、石油価格はUS Department of Energy、搭乗者とフライト回数の増分による追加的なコストはMLITからそれぞれ入手し、それぞれCPIによる割引処理を行う。また、石油価格はドルから円への換算、限界費用は空港のウェブサイトや報道を元にした修正を行っている。

\item モデル、推定方法

\begin{itemize}

\item Effect on Market outcomes

DID分析を用いて、合併による市場への直接的な影響を推定する。time $t$における市場 $m$ の商品(便)$j$の価格、本数、乗員数を被説明変数 $y_{jmt}$ として、それぞれ以下の2つのモデル

\[\ln(y_{jmt})=\gamma^A_1 \cdot \textit{JJ}_{jmt} + \gamma^A_2 + \cdot \textit{post}_t +\gamma_3^A \cdot \textit{JJ}_{jmt} \cdot \textit{post}_t + \mathbf{x}'_{jmt} \cdot \lambda^A +\kappa_{jmt}^A \]

\begin{align*}
\ln(y_{jmt}= &\gamma^B_1 \cdot \textit{JJ}_{jmt} + \gamma^B_2 + \cdot \textit{post}_t +\gamma_3^B \cdot \textit{JJ}_{jmt} \cdot \textit{post}_t \\
 & + \textit{MtM}_{jmt} \cdot (\gamma^B_4 \cdot \textit{JJ}_{jmt} + \gamma^B_5 + \cdot \textit{post}_t +\gamma_6^B \cdot \textit{JJ}_{jmt} \cdot \textit{post}_t ) \\
 & +\textit{MtO}_{jmt} \cdot (\gamma^B_7 \cdot \textit{JJ}_{jmt} + \gamma^B_8 + \cdot \textit{post}_t +\gamma_9^B \cdot \textit{JJ}_{jmt} \cdot \textit{post}_t) + \mathbf{x}'_{jmt} \cdot \lambda^A +\kappa_{jmt}^A 
 \end{align*}

について推定を行う。$\textit{JJ}_{jmt}$ はJAL、JAS、合併後の$t$ についてはJAGの便であること、$\textit{post}_t $は合併後であることをそれぞれ示すダミー変数であり、両者の交差項が興味のある係数である。$\mathbf{x}'_{jmt}$は $j, m, t$ の各要素について設定するダミー変数である。

二番目のモデルは、合併によって独占市場に移行したことを示すダミー変数 $\textit{MtM}_{jmt}$ 、同じく寡占市場への移行を示す $\textit{MtO}_{jmt}$と一番目のモデルの各項との交差項を含めたモデルである。この項と先に述べた交差項との交差項の係数の推定値を確認することで、合併による効果が市場の構造に依存することを仮定したモデルを考える。

\item Demand

Market levelのNested logit モデルを用いる。

消費者 $i$ の効用関数は

\[u_{ijmt} = \alpha p_{jmt} +\beta f_{jmt}^{\rho} +\mathbf{x}^{D}_{jmt} ` \gamma + \xi_{jmt} +\epsilon _{ijmt} \]

で定義される。Flight Frequency $f_{jmt}$ は消費者が希望する時刻により近い便を選択できる可能性の高さを示しており、当該ルートの観察可能なQualityと考えられる。観察されないQualityは$\xi_{jmt}$ で表現される。

この効用関数の下で、商品$j$ の市場$m$ における時間$t$ 時点でのシェアは、以下のモデルによって記述される。ここで、$j \in J$ には当該ルートに就航する各キャリアに加え、いずれの商品も選択しない$j=0 $ : outside goods が存在することに注意する。outside goodsがもたらす効用は0に標準化される。

\begin{align*}
s_{jmt}(\mathbf{p}_{mt},\mathbf{f}_{mt}) & \equiv s_{jmt|gt} \cdot s_{gt} \\
 & = \dfrac{\exp \left( \frac{\alpha p_{jmt} +\beta f_{jmt}^{\rho} +\mathbf{x}^{D}_{jmt} ` \gamma + \xi_{jmt}}{1-\sigma} \right)}{V_{mt}^{\sigma}(1+V_{mt}^{1-\sigma})}
\text{where} & V_{mt} \equiv \sum _{k \in J_{mt}} \exp \left( \dfrac{\alpha p_{kmt} + \beta f_{kmt}^{\rho} +\mathbf{x}^{D}_{jmt} ` \gamma + \xi_{jmt}}{1 - \sigma } \right)
\end{align*}

Nest はoutside goods のみを含むものと、それ以外のすべての航路を含むものとの2種類としている。$\sigma$はNest間の相関を表し、0の場合、モデルはoutside goods を含めたsimple logit model とみなすことになる。

ここから、推定を行う線形回帰式は

\[\ln (s_{jmt}) - \ln (s_{0mt}) = \alpha p_{jmt} + \beta f_{jmt}^{\rho} + \mathbf{x}_{jmt}^D ` \gamma
 + \sigma \ln (s_{jmt | gt}) + \xi_{jmt} \]
 
 となる。
 
 $s_{jmt|gt}$には$s_{jmt}$が含まれており、また価格とFlight Frequencyには内生性があるため、従属変数と独立変数には相関があると考えられる。そのため、推定に当たってはCost Shifter、具体的には旅客輸送に用いる航空機の特徴、燃料価格と燃油サーチャージ、乗客数、フライト回数に応じて生じる空港への支払金を操作変数とした二段階最小二乗法を用いる。

\item Supply

各企業の利潤最大化問題は

\[ \max_{p_{jmt},f_{jmt}} \sum _{s \in F_l} 
[(p_{smt} - mc_{smt}^q ) \cdot q_{smt} 
( \mathbf{p}_{mt}, \mathbf{f}_{mt}) - mc_{smt}^f \cdot f_{smt}] \]

で記述される。企業がコントロールする変数として、価格に加えて Flight Frequencyが導入されていることに留意する。ここから、最大化の一階の必要条件

 \begin{align*}
 \mathbf{s} + D^{\tau} \cdot B^p (\mathbf{p},\mathbf{f})(\mathbf{p} - \mathbf{MC}^q) &=0 \\
 D^{\tau} \cdot B^f (\mathbf{p},\mathbf{f})(\mathbf{p}-\mathbf{MC}^q) &= \mathbf{MC}^f
 \end{align*}

が導かれる。これを解いて、乗客人数$q$、Flight Frequency $f$ に対する限界費用$\mathbf{MC}^x$を導出する。$B^x$は各商品のシェア$s_{jmt}$をそれぞれ要素$x$で偏微分した行列、$D$は各商品が同一企業によって販売されている場合に1、そうでない時に0をとるownership matrixである。添字$\tau$はJJ merger の前後で起こる$D$の変化を捕捉している。この$D$を変化させることで、市場がそれぞれ競争的であるか、寡占、独占的であるかを仮定することができる。

\item Marginal Cost

marginal cost は供給量、Flight Frequencyによらず一定であると仮定する。この時、要素$x=p,q$に対するMarginal Cost に空港への支払金を加算し、対数をとった費用関数は

\[ \ln(mc_{jmt}^x + apc_{jmt}^x) = b_W^x \ln (w_{jmt}^x) 
+ b_N^x \ln(\textit{nroute}_{jmt}) + e_{jmt} \]

で記述される。$w_{jmt}$には航空機の特性を表す項が含まれる。関心がある項は、各期の各市場、当該ルートの着陸地の空港において企業$j$が提供する航路の数$\textit{nroute}_{jmt}$で、これが合併による効率性改善を測る判断材料になるとしている。この変数は空港の規模に関係するため、Marginal Costと相関することが考えられるが、商品ダミーの導入によってこの相関が和らげられるとして、この研究では外生変数であるとみなして推定を行う。誤差項$e$は一階の自己相関モデル$e_{jmt}^x = \phi e_{jmt-1}^x + \nu_{jmt}$に従うと仮定する。

さらに、JJ mergerの前後それぞれの市場について、競争市場と寡占市場のどちらのモデルが当てはまりがよいか、残差二乗和を用いたRiver-Vuong Testによって検討する。

モデル$h$の下での要素$x$についての残差$\Hat{e}_{jmt}^{xh}$の二乗和を各要素について足し合わせた

\[Q_m^h(\mathbf{b}^h) = \dfrac{1}{n} \sum_{jmt} 
\left\{ \left( \Hat{e}_{jmt}^{qh}\right)^2 +\left( \Hat{e}_{jmt}^{fh} \right) \right\} \]

を導出し、モデル間の差を標準化した統計量を用いて当てはまり度合いを検定する。

\end{itemize}

\item 推定結果

 \begin{itemize}
 
 \item DID
 
 JAGの便と合併後の期間を表すダミー変数の交差項の係数$\gamma_3$は、価格に対して負、Flight Frequency と乗客数に対して正値を取り、合併は全体として産業の効率性を改善したことが分かった。一方、合併によって寡占から独占状態に変化した航路については価格が上昇し、こうした市場におけるJAGのMarket Power が上昇したことが示されたが、他方、合併がFlight Frequencyと乗客数に対して正の影響を与えた、という結果も得られた。
 
 頑健性を確かめるために、JAGとそれ以外に対するtime 、route trendが異なる可能性の検討、合併時期を実際に起こった以前の時点とするplacebo testを行ったが、いずれも上記の結果を否定しなかった。
 
 \item Demand
 
 操作変数法の第一段階推定では、F値が99\%有意であることが示され、操作変数として有効であることが確認された。
 
 OLSによる推定結果によると、価格の係数の推定値は負で、経済学的な考察と整合的であった。説明変数が観察不可能なqualityと相関することによる内生性の問題から、この値は過大に推定されている可能性があるが、二段階最小二乗法を用いた推定を用いても、価格の係数は負値であると推定された。
 
 二段階最小二乗法から推定された価格に対する需要弾力性は日本以外の航空産業について推定した先行研究とも整合的であった。
 
 また、Flight Frequencyの係数の推定値は$\beta$,$\rho$ともに負値を取り、$\beta \rho >0$,$\rho<0$から、Flight Frequencyに対する限界効用はconcaveであることが確認された。価格と同様に、需要弾力性の推定値も先行研究におけるそれに近い値が得られた。
 
 Nest 間の相関を示す$\sigma$の値は、日本以外の市場を扱った先行研究のそれよりも小さかった。これは、日本の国内旅客輸送において新幹線が航空機の代替財となっており、航空機が国内輸送のおよそ5.9\%(対距離)を占めるに過ぎないことが理由と考えられる。
 
 合併の認可は合併以前のデータに基づいて判断されるため、サンプルを合併以前のものに限ったものに対して同様の推定を行った結果も元のモデルと大きな違いが得られなかった。
 
 \item Marginal Cost
 
 River \& Vuong Testの結果は、合併以前、以後ともにBertrand-Competitionを仮定するモデルが、合併後、あるいは前後両方でCollusionを仮定するモデルよりも当てはまりが良いことを支持するものであった。よって、限界費用の推定にはこのモデルを用いる。
 
 航空機の特性を表す変数$w_{jmt}$の推定値は、Flight Frequencyに対する航空機の座席数を除いて有意な結果が得られなかった。有意な影響が見られた座席数に対する限界費用弾力性は先行研究と整合的な値となった。一方、座席数のquantityに対する影響は統計的に有意でなく、航空産業における規模の経済効果は限定的であることが明らかになった。
 
 $nroute$の係数の推定値は価格、Flight frequencyのいずれに対しても負で、効率性の改善に大きな影響を与えていることが明らかになった。合併によるノウハウの共有といった、他の改善要因をコントロールするために、JAGの便であることと合併後の期間を示すダミーの交差項をモデルに組み込んだ場合もこの効果は支持された。
 
 当時、合併による効率性の改善は360億円と推定されていたが、このモデルを用いた合併の効果の推定値は年あたり280億円であり、合併を決断した当時の判断を評価している。

 \end{itemize}
 
 \hspace{1.5zw}
 
 最後にこの論文では、合併による総余剰への影響と、合併の認可に伴ってJAGに課した制限の果たした役割について、得られた需給の推定値を用いたシミュレーションを元に議論している。
 
 合併が起こらなかった場合の総余剰を推定し、実際の変化との比較を行った結果、合併が全体としては消費者余剰と総余剰との両方を改善したことが分かった。市場を合併によって独占に移行したもの、寡占に移行したもの、それ以外に分類すると、合併により、価格は独占に移行した市場のみ上昇したが、全体では引き下げられた。一方、Flight Frequencyは全ての市場で増加し、この結果、利用者数は独占に移行した市場以外で増加した。独占に移行した市場は相対的に数が少ないため、合併によって起こった効率性の改善が、総余剰の改善に重要な役割を果たしたとしている。
 
 この効率性の改善がもたらした影響を考察するため、次のシミュレーションでは、合併は起こったが、効率性の改善は起こらなかった場合を推定する。これによると、効率性の改善を仮定しないモデルでは、消費者余剰が減少させるという結果が得られた。影響は特に独占に移行した市場で大きかった。また、Flight frequencyを外生変数とするモデルでは、独占に移行した市場において大幅に余剰が改善し、変化のない市場ではその改善度合いが小さくなることが分かった。
 
 これらを踏まえて、効率性の改善、価格とFlight Frequencyの需要代替率が消費者余剰に与える影響を推定すると、効率性の改善の係数は正値で推定され、合併による効率性の改善が消費者余剰の改善に正の影響を与えるとする仮定と整合的な結果が得られた。
 
 \hspace{1zw}
 
 一方、合併に当たり、JAGに課された制限、即ち羽田空港の滑走路の譲渡は、競争維持に余り大きな影響をもたらさなかったことが分かった。滑走路の譲渡によって消費者余剰、JAG以外の企業の余剰は改善したが、その影響は比較的小さく、この規制がなくとも多くの市場(航路)で消費者余剰が改善するという推定結果が得られた。独占市場となっている航路は市場規模が小さく、競争のインセンティブを大きくしても実際に参入する企業は少ないと考えられることも原因とひとつであるとしている。これを踏まえて、筆者は価格や便数の上限規制など、実際の行動を直接制限する政策が必要であったと結論付けている。
 
\item コメント

Flight Frequencyが財の質として読み換えられるこのモデルは、同質の財でありながら、経時的な供給量を変化させることで代替財が生まれ、利便性が向上する交通産業を描写するのに適切なモデルだと考えられる。その他の交通産業にも応用可能性が考えられるが、一方で、日本の多くの鉄道は半官半民的な性質を持ち、価格の大きな変動が起こりにくい点を念頭に、この論文と同様に、外生性の仮定を置いたモデルとの比較を行いながら、モデルが適用可能か慎重に検討する必要があると思われる。

また、筆者は競争維持を促進するための制限として、価格や便数の上限規制を提案していたが、こうした規制の存在も他の交通産業について考察する為のポイントとなると考えられる。航空産業の場合は、こうした規制が実際に提言されながら、採用には至らなかったことも踏まえ、いかにこうした規制を実現できるようなデザインを行うかについても研究の可能性が期待される。

\end{enumerate}

\end{document}