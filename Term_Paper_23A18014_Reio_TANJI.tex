\documentclass{jsarticle}[12pt]
\usepackage{mathpazo}
\usepackage{amsmath,amssymb}
\usepackage{array}
\begin{document}

\title{Behavioral Economics Term-Paper}
\author{経済学研究科  \\ 23A18014 Reio TANJI 丹治伶峰}
\date{}
\maketitle

\section{Reserch Topic}

\Large

\textbf{Economic Evaluation of the Behavior of Athletes}

\hspace{5zw}- Cases from Major League Baseball Players -

\large

\vspace{1zw}

In this research, I try to specify reference-dependent preference of the professional baseball players about their performance. More specifically, they prefer their ``batting average (打率 in Japanese)'' or ``on-base percentage (出塁率)'' with slightly larger than some round-numbers (.300, .350, and so on), to just below them.

\section{Motivation}

Exploiting these specification to design optimal contracts with the professional athletes.

\section{Literature Review and Extention}

There are many empirical research that test the existence of a certain reference points in individuals' preference.

Pope and Simonsohn(2011) specified that round numbers act as a reference point, exploiting three empirical researches. One of these studies, they utilized the case of professional baseball players in Major League Baseball (MLB).According to this, batters prefer finishing their season with their batting average just above .300 to just below that number, which makes the distribution of their batting averages at the end of each season has a discontinuity around .300. Roughly, I follow this method of specification.

On the other hand, however, there are some alternative interpretation to the discontinuity, such as the existence of monetary incentives. I have to make some analysis on the evaluation by the team managers. Hakes and Sauer(2006) measured the impact of the pulication of `\textit{Moneyball}' (Michael Lewis, 2003), which points out that they underestimates the contribution of on-base percentage , rather than batting average, to increasing their scores, and raising winning percentage.  Hakes and Sauer utilized statistical methods to find that the underestimation leads to inefficient contracts in the labour market in MLB, and that this inefficiency was relaxed after the publication of `\textit{Moneyball}.'

\section{Hypothesis}

Aggregating these two studies, I test two hypotheses below :

 \begin{itemize}
 
 \item Professional Baseball players have round-number reference point about their batting-indexes, such as batting average or on-base percentage.
 
 \item After the publication of `\textit{Moneyball},' their attachment to batting average declines, which results in smaller/no discontinuity in the distribution.
 
 \end{itemize}

\section{Key Assumtions}

Following Allen, et al. (2017), I define the reference-dependent utility function, $v(.)_r$, by three primary forms :

 \begin{enumerate}
 
 \item A jump : discontinuity at reference point $r$
 
 \[ \lim_{\epsilon \to 0} v_r (r+ \epsilon) \neq \lim_{\epsilon \to 0} v_r (r -\epsilon) \]
 
 \item A kink (1) : discontinuity in the first derivative at $r$
 
 \[ \lim_{\epsilon \to 0} v_r ` (r+ \epsilon) \neq \lim_{\epsilon \to 0} v_r ` (r -\epsilon) \]
 
 \item A kink (2) : discontinuity in the second derivative at $r$
 
 \[ \lim_{\epsilon \to 0} v_r `` (r+ \epsilon) \neq \lim_{\epsilon \to 0} v_r `` (r -\epsilon) \]
 
 \end{enumerate}

based on Kahneman and Tversky (1979)'s primary gain-loss payoff function

\[n(x|r) = \mu ( m(x) - m(r) ) \]

and its extensive study about loss aversion and diminishing sensitivity (Tversky and Kahneman 1992).

\section{Model \& Data}
 \subsection{Data}
 
 Team-level and player-level panel batting stats of MLB players can be obtained by officialy disclosed data statistics to get panel data. Also, player-level information about contracts can be obtained, covering to some details such as contract length and stats-dependent (incentive) contracts.
 
 Note that the sample is limited to the position player (exclude pitcher) with at-bats(打数 in Japanese) more than a certain number in each year. In Pope and Simonsohn (2011), they chose 200 at-bats as the threshold.
 
 \subsection{Existence of reference-dependent preference}
 
 Essentially, we follow Pope and Simonsohn (2011).
 
 First I test whether the discontinuity of the distribution in frequency of batting average and on-base percentage around the round number, by utilizing t-test. If the number of the players who finish each season with their batting average/on-base percentage just above round numbers are statistically significantly larger than those with just below them, then we get some evidence of the reference point.
 
 Moreover, I include time-series analysis in this study. Pope and Simonsohn (2011) used data from 1975 to 2008 and aggregated it as a single dataset. In this study, however, I distinguish this dataset into `Before \textit{Moneyball}' phase and `After \textit{Moneyball}' phase. According to Hakes and Sauer (2008), team manager's overestimation on batting average and underestimation on on-base percentage were relaxed immediately after the publication, so I try on the same analysis, but deviding sample into two subsets as mentioned above. Further possible extention, it is known that the 1994-1995 strike come out by the MLB players influenced on the evaluation of the managers. It may be suggestive to distinct before  and after the event.
 
 To check the robustness, I do the following tests, also following Pope and Simonsohn (2011).
 
  \begin{itemize}
  
  \item Check the frequency distribution when the last 5 plate-appearance (打席) left in each season.
  
  \item Results in the last plate appearance whose batting-average/on-base percentage is around round numbers.
  
  \end{itemize}
 
 These analysis gives us about the player's effort to meet their goals based on reference-dependent preference.
 
 \subsection{Background : Monetary incentives}
 
 Apply regerssion discontinuity design (RDD) to specify whether there exists any monetary incentives for making especially much effort to achieve their goals.
 
  \begin{align*}
    & y_{it} = \beta_0 + \beta_1 \mathbf{X}_{it} + \beta_2 \textit{Above}_{it} + u_{it} \\
   & \text{where }  \\
   y_{it} : & \text{ player } i's \text{ salary in the season } t \\
   \mathbf{X}_{it} : & \text{ Controls} \\
   \text{Above}_{it} : & \text{ whether their performance index is above the possible reference point}
   \end{align*}
 
 Controls include players' performance, and individual charactaristics such as age, position, contract length, presence of conditional bonus, posession of the right to arbitration or free agency.
 
 If the coeffient $\beta$ is significant, then there is some monetary incentive for achieving their goals, which rationalizes the players' unnaturally high aspiration. 
 
 Also in this section, distinguishing `Before' with `After' \textit{Moneyball} will be suggestive analysis. Specifically, I conduct year-specifying analysis.
 
\section{Referances and Definitions}

 \subsection{References}
 
  \begin{itemize}
  
  \item Pope and Simonsohn(2011) `Round Numbers as Goals : Evidence From Baseball, SAT Takers,
and the Lab'  \textit{Association for Psychological Science}
  
  \item Hakes and Sauer(2006) `An Economic Evaluation of the Moneyball Hypothesis' \textit{Journal of Economic Perspectives}
  
  \item Allen,Dechow, Pope \& Wu(2017) `Reference-Dependent Preferences: Evidence from
Marathon Runners' \textit{Management Science}
  
  \item Kahneman and Tversky (1979, ECTA)
  
  \item Kahneman and Tverssky (1992) `Advances in Prospect Theory:Cumulative Representation of Uncertainty'  \textit{Journal of Risk and Uncertainty}
  
  \end{itemize}
 
 \subsection{Definitions}
 
 Definitions of batting average (AVG) and on-base percentage (OBP) are as follows :
 
 \begin{align*}
\text{AVG} &= \dfrac{\text{base-hits}}{\text{at-bats}} \\
\text{OBP} &= \dfrac{\text{base-hits} + \text{walks} + \text{hit-by-pitches}} 
{\text{at-bats} + \text{walks} + \text{hit-by-pitches} + \text{sacrifice-flies}} \\
\end{align*}

OBP counts walks(四球) and hit-by-pitches(死球) as the element to raise the index, while AVG only includes base-hits(安打) as them. Studies have show that OBP is more correlated with winning average and total scores of the team he belongs to, which indicates that for the team manager OBP should be evaluated than AVG.

\end{document}