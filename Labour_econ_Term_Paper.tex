\documentclass[dvipdfmx]{jsarticle}
\usepackage{mathpazo}
\usepackage{amsmath,amssymb}
\usepackage{graphicx}
\usepackage{tikz}
\begin{document}

\title{労働経済I 期末レポート}
\author{経済学研究科 23A18014 丹治伶峰}
\date{}
\maketitle

\large

 \begin{enumerate}
 
 \item
 
 (1)
  For any $(L_a, L_b)$, marginal rate of technical rate of substitution of $L_a$ to $L_b$ is :
 
  \begin{align*}
  \textit{MRTS} &= \dfrac{\partial y / \partial L_a}{\partial y / \partial L_b} \\
  &= \dfrac{\alpha A (\frac{L_b}{L_a})^{1-\alpha}}{(1-\alpha) A (\frac{L_a}{L_b})^\alpha} \\
  &= \dfrac{\alpha L_b}{(1-\alpha) L_a}
  \end{align*}
 
 (2)
 
 For arbitrary target output, $\bar{y}$, the cost minimization problem of the planner is
 
  \begin{align*}
  \min_{L_a, L_b} &= w_a L_a + w_b + L_b \\
  \text{sub to } & \bar{y} \leq y
  \end{align*}
 
 Since $y = A(L_a)^{\alpha} (L_b)^{1-\alpha}$ produce no output if either $L_a$ or $L_b$ is zero, there cannot be any corner solution.
 
 Then, the Lagrangian function and Kuhn-Tucker condition are described as follows.
 
 \[ \mathcal{L} = w_a L_a+ w_b L_b + \lambda(\bar{y} - y) \]
 
  \begin{align*}
  w_a - \lambda \alpha A \left( \dfrac{L_b}{L_a} \right)^{1-\alpha} &=0 \\
  w_b - \lambda (1-\alpha) A \left( \dfrac{L_b}{L_a} \right)^{\alpha} &=0 \\
  A (L_a)^{\alpha} (L_b)^{1- \alpha} = \bar{y}
  \end{align*}
 
 Solving this, the conditional demand funcion for $L_a$ and $L_b$ are derived as follows :
 
  \begin{align*}
  \bar{L_a}(w_a, w_b, \bar{y}) &
  = \dfrac{\bar{y}}{A} \left( \dfrac{\alpha w_b}{(1-\alpha) w_a} \right)^{1-\alpha} \\
  \bar{L_b}(w_a, w_b, \bar{y}) &
  = \dfrac{\bar{y}}{A} \left( \dfrac{(1-\alpha) w_a}{\alpha w_b} \right)^{\alpha}
  \end{align*}
 
 (3)
 
 By the conditional demand function obtained in (3), $\bar{L_a}(w_a,w_b,\bar{y})$ is decreasing in $w_a$, while $\bar{L_b}(w_a,w_b,\bar{y})$ increasing in $w_a$, since $\bar{y}, A >0$,$\alpha \in (0,1)$, $\left( \dfrac{w_b}{w_a} \right) > 0$ and $\left( \dfrac{w_a}{w_b} \right)>0$ for any positive $w_a$ and $w_b$.
 
 Thus, for $\bar{w_a} > w_a$,
 
  \begin{align*}
  \bar{L_a}(\bar{w_a}, w_b, \bar{y}) &< \bar{L_a}(w_a, w_b, \bar{y}) \\
  \bar{L_b}(\bar{w_a}, w_b, \bar{y}) & >\bar{L_b}(w_a, w_b, \bar{y})
  \end{align*}
 
 (4)
 
 Planner's profit maximization problem is :
 
  \begin{align*}
  \max_{y} \Pi(y) &= p(y) \cdot y - w_a L_a - w_b L_b \\
  &= y^{\frac{1+\beta}{\beta}} 
  - \dfrac{y}{A} \cdot \left( \dfrac{\alpha}{1-\alpha} \right)^{1-\alpha} w_a^{\alpha} w_b^{1-\alpha}
  - \dfrac{y}{A} \cdot \left( \dfrac{1-\alpha}{\alpha} \right)^{\alpha} w_a^{\alpha} w_b^{1-\alpha} \\ 
  \end{align*} 
 
 FOC yields
 
  \begin{align*}
  \dfrac{1+\beta}{\beta} y^{\dfrac{1}{\beta}} 
  - \frac{1}{A} \left( \dfrac{\alpha^{1-\alpha}}{(1-\alpha)^{1-\alpha}}
  + \dfrac{(1-\alpha)^{\alpha}}{\alpha^\alpha} \right)
  w_a^{\alpha} w_b^{1-\alpha} &=0 \\
  \\
  y^* (w_a, w_b) &= \left[ \dfrac{\beta}{A(1+\beta)(1-\alpha)^{1-\alpha} \alpha^\alpha}
  w_a^{\alpha} w_b^{1-\alpha} \right]^{\beta}
  \end{align*}
 
 (5)
 
 When $y$ is normal goods, $p(y)$ should be decreasing in $y$, which requires that $p'(y) = \dfrac{1}{\beta} y^{\frac{1-\beta}{\beta}} < 0.$ Also, by the optimal output obtained in (4), $y^* \geq 0$ iff $\beta < -1$ or $\beta \geq 0$. 
 
 So, I assume that $\beta < -1$. Then, as $\beta < 0$, $y^*$ is decreasing in $w_a$. By (2), conditional demand for both $L_a$ and $L_b$ is increaseing in $\bar{y}$, which indicate that for any $\bar{w_a} > w_a$, $\bar{y^*} < y^*$ holds and so
 
  \begin{align*}
  \bar{L_a}(\bar{w_a}, w_b, \bar{y^*}) &< \bar{L_a}(w_a, w_b, y^*) \\
  \end{align*}
 
 Note that the relationship between $\bar{L_b}(\bar{w_a}, w_b, \bar{y^*})$ and $\bar{L_b}(w_a, w_b, y^*)$ is parametrical.
 
 (6)
 
 Summing up (2) and (4), the unconditional demand for $L_a$ and $L_b$ is obtained as follows :
 
  \begin{align*}
  L^*_a(w_a, w_b) &= \bar{L_a}(w_a, w_b, y^*) \\
  &= \dfrac{y^*}{A} \left( \dfrac{\alpha w_b}{(1-\alpha) w_a} \right)^{1-\alpha}
  L^*_b(w_a, w_b) &= \bar{L_b}(w_a, w_b, y^*) \\
  &= \dfrac{y^*}{A} \left( \dfrac{(1-\alpha) w_a}{\alpha w_b} \right)^{\alpha}
  \end{align*}
 
 Note that $y^* = y^* (w_a, w_b) = \left[ \dfrac{\beta}{A(1+\beta)(1-\alpha)^{1-\alpha} \alpha^\alpha}
  w_a^{\alpha} w_b^{1-\alpha} \right]^{\beta}$.
  
 (7)
 
 There are two effects by the change in $w_a$.
 
 \begin{itemize}
 
  \item Scale effect decreases both $L_a$ and $L_b$.
 
  \item Substitution effect decreases $L_a$, while raising $L_b$.
 
 \end{itemize}
 
 The Illustration below shows the case that scale effect  on $L_b$ exceeds the substitution effect.
 
 \begin{center}
 
 \newpage
 
 Illustration : $\bar{w_a} > w_a$
 
  \begin{tikzpicture}[domain=0:4,samples=200, >=stealth]
   \draw[->] (-0.5,0) -- (7.2,0) node[right] {$L_a$} ;
   \draw[->] (0,-0.5) -- (0,6.2) node[above] {$L_b$} ;
   \draw (0,0) node[below left] {O};
  \end{tikzpicture}
 
 \end{center}
 
 \item
 
 (1)
 
 The individual's maximization problem is :
 
 \[ \max U(C,l) \]
 
  \begin{align*}
  \text{ sub to} \\
  C &= c_D + c_M \\
  c_D &= f(h_D) \\
  c_M & \leq w h_M + R
  \end{align*}
 
 Note that for the third restriction, equation holds in the static case
 
 : Saving is irrational choice.
 
 Then, define $R_0$ be the potential income
 
 \[R_0 \equiv w L_0 + R \]
 
 where $L_0 \equiv = l + h_D + h_M$ stands for net time available.
 
 Then, by $c_M = w h_M + R$, 
 
  \begin{align*}
  c_M &= w (L_0 - l - h_D) + R_0 - w L_0 \\
  c_M &= R_0 - wl - wh_D
  \end{align*}
 
 By $C = c_D + c_M$ and $c_D = f(h_D)$,
 
  \begin{align*}
  c_M + wl &= R_0 - w h_D \\
  c_M + f(h_D) + wl &= R_0 - w h_D \\
  C - f(h_D) + wl &= R_0 - w h_D \\
  C + wl &= [f(h_D) - w h_D] + R_0
  \end{align*}
 
 These discussion rewrite the UMP as follows
 
 \[\max u (C, l)\]
 
  \begin{align*}
  \text{ sub to} \\
  C + wl &= [f(h_D) - w h_D] + R_0
  \end{align*}
 
 Finally, we derive the Kuhn-Tucker condition. The Lagrangian function is :
 
 \[ \mathcal{L} = u(C,l) + \lambda \{ R_0 + [f(h_D) - w h_D] - c -wl \} \]
 
 $\lambda$ is the Lagrangian multiplier.
 
 Kuhn-Tucker condition is derived as follows :
 
  \begin{align*}
  \dfrac{\partial u}{\partial C} - \lambda \leq 0, & 
  \hspace{1zw}C \left( \dfrac{\partial u}{\partial C} - \lambda \right) = 0 \\
  \dfrac{\partial u}{\partial l} - \lambda w \leq 0,  &
  \hspace{1zw}l \left( \dfrac{\partial u}{\partial l} - \lambda w \right) = 0 \\
  \lambda [ f'(h_D) - w h_D] \leq 0, & \hspace{1zw}\lambda [f'(h_D) - w] = 0 \\
  R_0 + [f(h_D) - w h_D] -c -wl \geq 0, &
  \hspace{1zw}\lambda \{ R_0 + [f(h_D) - w h_D] - c -wl \} = 0
  \end{align*}
 
 where $R_0 = w (l + h_D + h_M) + R$.
 
 (2)
 
 By the assumption, $l > 0$ and $h_M >0$,
 
  \begin{align*}
  \dfrac{\partial u}{\partial l} &= 0 \\
  \dfrac{\partial u}{\partial C} & = 0 \\
  \end{align*} 
 
 holds. Then, assume u(.) is monotonic in $l$, 
 
 \[0 < \frac{\partial u}{\partial l} \leq \lambda \]
 
 which indicates that
 
 \[ R_0 + [f(h_D) - w h_D - C - wl] = 0 \]
 
 and
 
 \[ f'(h_D) - w  = 0 \]
 
 Deriving marginal rate of substitution, 
 
 \[ \textit{MRS} \equiv \dfrac{\partial u / \partial l}{\partial u / \partial C} 
 = \dfrac{\lambda w}{\lambda} = w \]
 
 First, by $f'(h_D) - w  = 0$,  optimal $h_D$ is determined, and then, $R_0 + [f(h_D) - w h_D - C - wl] = 0$ and $\textit{MRS} = w$ yields the optimal $h_M$ and $l$.
 
 (3)
 
 By the assumption $h_M = 0$,
 
  \begin{align*}
  L_0 &= l + h_M \\
  R_0 &= 
  \end{align*}
 
 \item
 
 (1) 
  \begin{itemize}
  
  \item Ability bias
  
  \item Selection bias
  
  \end{itemize}
 
 (2) 
  \begin{itemize}
  
  \item Ability bias
  
  Optimal wage depending on the worker's ability $\theta$ is : 
  
   \begin{align*}
   w(\theta) & = A h_0 e^{\theta x(\theta)} \\
   & \text{where } \\
   h_0 : & \text{ initial human capital} \\
   x(.) : & \text{ optimal length of education as a function of } \theta
   \end{align*}
  
  When $\theta$ goes up, there are two different effects : direct effect on $w(\theta)$ and indirect effect, that raises $x(\theta)$, and then affect on $w(\theta)$.
  
  Then, assume simplified form regression on loggarithm wage $w_i$ by education length $t_i$,
  
  \[ \ln w_i = \beta_0 + \beta_1 t_i + \epsilon_i \]
  
  $t_i$ and error term $\epsilon_i$ are positively correlated, which causes bias by OLS.
  
  \item Selection bias
  
  Consider decision-making whether to go to college or not.
  
  Define $B^i$ be the education effect for type $i \in \{ C, H \}$ individual, then we would like to specify
  
   \begin{align*}
   B^C &= E_C^C - E_H^C \\
   B^H &= E_C^H - E_H^H \\
   \end{align*}
  
  Note that $E$ is return of education, superscript standing for the type of individual, and subscript denoting her/his actual choice ( : C is to go to college, while H not).
  
  By observed sample, however, the ``appeared'' effect $B$ is :
  
  \[ B = E_C^C - E_H^H \]
  
  since type $C$ individual usually go to school and vice versa.
  
  Again, assume the wage regression
  
  \[ \ln w_i = \beta_0 + \beta_1 t_i + \epsilon_i \]
  
  If $E_H^C < E_H^H$, where the earnings of job for high school would have been lower for type C than for type H, then $t_i$ and $\epsilon_i$ are negatively correlated, and so $B^C$ is underestimated.
  
  Similarly, if $E_C^H < E_C^C$, then $t_i$ and $\epsilon_i$ are positively correlated, so $B^H$ is overestimated.
  
  \end{itemize}
 
 (3)
 
 Exclusive instruments should have explanatory power to the independent variable you are interested in, but be independent of the error term of the original regression model.
 
 Consider the regression model
 
 \[ \ln y_i = \beta_0 + \beta_1 t_i + \beta_2 + x_i + \epsilon_i \]
 
 Then, excluded variable $Z_i$ is introduced to the following regression :
 
 \[t_i = \gamma_0 + \gamma_1 Z_i + \gamma_2 + e_i \]
 
 For the excluded variable being valid, two conditions below are required :
 
  \begin{itemize}
  
  \item $\gamma_1$ is statistically significant, and not close to zero.
  
  \item $Z_i$ is independent of any variables that determines $y_i$, except for $t_i$.
  
  : $Cov(Z_i, \epsilon_i) = 0$.
  
  \end{itemize} 
 
 In order to find a valid instruments, we should pay attention to legal engagement that make individuals to apply some choice, regardless of charactaristics of each individual. 
 
 (4)
 
 In this paper, they utilized birth-month of the students, to identify the effect of compulsory school attendance on their ability and wages.
 
 In the U.S., those who were born early in the calender year receive compulsory education for shorter duration than those who were born late. This law is guaranteed by legal regulation about compulsory education, and there is no correlation between birth-month and their ability. Thus, using this as an instrument variable, they can limit the estimation bias.
 
 (5)
 
This article specifies the marginal return to length of education on their wage, using dataset of identical twins.

They conducted a questionnaire, by twins, asking their schooling level. The important point is they requested each of them to answer the difference in education between the twins. Then, they used this sibling-report as instrument variable of the schooling level of the individuals. Correlation between self and sibling report, caused by person-specific component of mesurement error is controlled in this method, and we can see the actual effect of the difference in education on that of wage.
 
 \end{enumerate}

\end{document}