\documentclass{jsarticle}
\usepackage{mathpazo}
\usepackage{amsmath,amssymb}
\begin{document}

\title{労働経済I 期末レポート}
\author{経済学研究科 23A18014 丹治伶峰}
\date{}
\maketitle

\large

 \begin{enumerate}
 
 \item
 
 
 
 \item
 
 
 
 \item
 
 (1) 
  \begin{itemize}
  
  \item Ability bias
  
  \item Selection bias
  
  \end{itemize}
 
 (2) 
  \begin{itemize}
  
  \item Ability bias
  
  Optimal wage depending on the worker's ability $\theta$ is : 
  
   \begin{align*}
   w(\theta) & = A h_0 e^{\theta x(\theta)} \\
   & \text{where } \\
   h_0 : & \text{ initial human capital} \\
   x(.) : & \text{ optimal length of education as a function of } \theta
   \end{align*}
  
  When $\theta$ goes up, there are two different effects : direct effect on $w(\theta)$ and indirect effect, that raises $x(\theta)$, and then affect on $w(\theta)$.
  
  Then, assume simplified form regression on loggarithm wage $w_i$ by education length $t_i$,
  
  \[ \ln w_i = \beta_0 + \beta_1 t_i + \epsilon_i \]
  
  $t_i$ and error term $\epsilon_i$ are positively correlated, which causes bias by OLS.
  
  \item Selection bias
  
  Consider decision-making whether to go to college or not.
  
  Define $B^i$ be the education effect for type $i \in \{ C, H \}$ individual, then we would like to specify
  
   \begin{align*}
   B^C &= E_C^C - E_H^C \\
   B^H &= E_C^H - E_H^H \\
   \end{align*}
  
  Note that $E$ is return of education, superscript standing for the type of individual, and subscript denoting her/his actual choice ( : C is to go to college, while H not).
  
  By observed sample, however, the ``appeared'' effect $B$ is :
  
  \[ B = E_C^C - E_H^H \]
  
  since type $C$ individual usually go to school and vice versa.
  
  Again, assume the wage regression
  
  \[ \ln w_i = \beta_0 + \beta_1 t_i + \epsilon_i \]
  
  If $E_H^C < E_H^H$, where the earnings of job for high school would have been lower for type C than for type H, then $t_i$ and $\epsilon_i$ are negatively correlated, and so $B^C$ is underestimated.
  
  Similarly, if $E_C^H < E_C^C$, then $t_i$ and $\epsilon_i$ are positively correlated, so $B^H$ is overestimated.
  
  \end{itemize}
 
 (3)
 
 Exclusive instruments should have explanatory power to the independent variable you are interested in, but be independent of the error term of the original regression model.
 
 Consider the regression model
 
 \[ \ln y_i = \beta_0 + \beta_1 t_i + \beta_2 + x_i + \epsilon \]
 
 Then, excluded variable $Z_i$ is introduced to the following regression :
 
 \[t_i = \gamma_0 + \gamma_1 Z_i + \gamma_2 + e_i \]
 
 For the excluded variable being valid, two conditions below are required :
 
  \begin{itemize}
  
  \item $\gamma_1$ is statistically significant.
  
  \item $Z_i$ and $\epsilon_i$ are independent
  
  \end{itemize} 
 
 
 \end{enumerate}

\end{document}