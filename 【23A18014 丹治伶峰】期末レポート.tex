\documentclass{jsarticle}[12pt]
\usepackage{mathpazo}
\usepackage{amsmath,amssymb}
\begin{document}

\title{労働市場政策 研究計画 \\
医療扶助給付システムの設計 : \\ 過剰診療による社会的損失の特定}
\author{経済学研究科 23A18014 丹治伶峰}
\date{}
\maketitle

\large

\section{概要}

日本における生活保護の給付システムが、その設計によって引き起こしうるモラルハザードの影響を特定する。本研究では医療サービスの対価を無制限に支給する医療扶助の給付制度に着目し、対価の支払いが受給者本人を介さずに、直接医療サービス提供者に対して行われる現行制度が引き起こすとされる、過剰診療の問題の有無を統計的に検討する。

\section{フレームワーク 新規性}

生活保護の支給制度に関する研究は数多く、受給者の勤労所得や労働に対する選好を考慮し、彼らの就業インセンティブを出来るだけ低下させずに、最低限の生活水準を維持するために必要な生活支援を行うことが出来るような給付スキームの設計、あるいは実証研究において、この制度上の不備がもたらす非効率性の特定が試みられてきた。

伝統的経済学のフレームワークでは、生活保護に代表される政府による所得再分配制度は、安定した不労所得を与えることから、消費の余暇に対する減退代替率の低い受給者の就業インセンティブを低下させ、給付がない場合と比較して労働供給が減少し、結果として生活保護による収入に依存した生活を続けてしまう可能性が存在する、とされている。Lemieux \& Milligan(2008)は、カナダの公的扶助が、受給者の年齢に応じて不連続に上昇する特徴を持っていたことを利用し、回帰不連続デザイン(RDD)を用いた分析によって、給付水準の上昇が独身の低学歴男性の就業インセンティブを低下させることを明らかにした。また、Bargain \& Doorley(2011)は、同様の分析をフランスの公的扶助制度について行い、不連続な給付制度によって、若年男性の就業インセンティブが低下することを示唆している。

一方で、伝統的経済学では、給付の方法の差異が受給者の経済活動に与える影響について言及していない。Friedman(1969)の生活保護制度の設計に関するフレームワークでは、前述した給付による就業インセンティブの低下に対処する方法として「負の所得税」の導入が提唱されている他、受給者の選好に応じた柔軟な消費を許容するため、現物給付よりも現金給付を行う方が望ましいとされている。しかし、現金給付による公的扶助は、受給者の近視眼的な選好によって奢侈品に対する過剰な消費を招き、結果として必需品の対価が払えなくなったり、身体的なリスクを冒して必要な医療福祉サービスの利用を控える可能性が指摘されている。こうした行動経済学的なアプローチから懸念される問題に対処するため、用途を限定したクーポンによる公的扶助の給付を行う国も存在する。

本研究では、このアプローチをサービスを提供する側の行動決定に拡張し、公的扶助の制度設計が引き起こすモラルハザードの問題を検討する。具体的には、日本の生活保護制度について、医療サービスの対価を基本的に制限なく、受給者への現金給付を通さない形で支給する医療扶助のシステムが、医療サービスの提供者に、医療扶助の受給者に対して必要以上の医療行為を行うことでより多くの利潤を得ようとする「過剰診療」を引き起こしているかどうかを、統計的な分析を用いて検証する。

医療扶助の受給者は、その場でサービスの対価を支払う必要がないため、サービス提供者はどの患者が医療扶助の受給者であるかを判別することが出来る。対価は受給者本人の当座の支払い能力に依存せずに必ず支払われるから、これが過剰診療の誘因となる。『生活保護の経済分析』(東京大学出版会)によると、社会保障への財政支出の削減は急務で、医療扶助についても、その給付の適正化が求められている。本研究によってこうした過剰診療の存在が明らかになれば、受給者に現金を給付するシステムによる非効率性とも比較を行いながら、適切な制度設計を再考する必要があると言える。


\section{データ}

厚生労働省が年度ごとに集計、発表している、「医療扶助実態調査」と「社会医療診療行為別統計」を用いてパネルデータを作成し、分析を行う。2つの統計調査はそれぞれ、医療扶助受給者と一般の医療保険給付の利用者について、診療報酬明細書の記載をもとにした、明細書の発行件数、一傷病あたりの診療行為回数、診療報酬点数、117の傷病中分類、入院$\cdot$入院外の分類、これらの医療行為について規定される診療報酬点数についてのデータを集計している。使用する期間は、これらの集計項目を2つの統計間、及び異時点間で追跡可能な形で公表している、2014年から2017年までの4年間とし、各傷病中分類、入院$\cdot$入院外の分類、そして医療扶助の受給の有無による分類をコホートの形成単位とする。ただし、医療扶助実態調査と社会医療診療行為別統計とで分類方法が異なる6分類は除外する。傷病中分類は、(1)感染症及び寄生虫症(2)新生物(3)血液及び造血器の疾患並びに免疫機構の障害(4)内分泌、栄養及び代謝疾患 (5)精神及び行動の障害(6)神経系の疾患(7)眼及び付属器の疾患(8)耳及び乳様突起の疾患(9)循環器系の疾患(10)呼吸器系の疾患(11)消化器系の疾患(12)皮膚及び皮下組織の疾患(13)筋骨格系及び結合組織の疾患(14)腎尿路生殖器系の疾患(15)妊娠,分娩及び産じょく(16)周産期に発生した病態(17)先天奇形,変形及び染色体異常(18)症状,徴候等で他に分類されないもの(19)損傷,中毒及びその他の外因の影響 の19の大分類に対して具体的な傷病名による集計を与える形で行われている。

\section{モデル}

DID分析を用いて分析を行う。回帰式は以下の通り。

 \begin{align*}
 y_{it} = \beta_0 + \beta_1 \mathbf{X}_{it} + \beta_2 \textit{Aid}_{it} + \beta_3 \textit{Hosp}_{it} +
 \beta_4 \textit{Year}_{it} + \beta_5 \textit{Aid}_{it} \times \textit{Year}_{it} + \beta_6 \textit{Aid}_{it} \times \textit{Hosp}_{it} 
 \end{align*}

$y_{it}$は時点$t$における医療扶助受給別、及び傷病別コホート$i$における医療支出を表す。医療支出には診療報酬点数、診療実績件数、受診回数を用いて

 \begin{enumerate}
 
 \item 件数あたり点数
 
 \item 回数あたり点数
 
 \item 件数あたり回数
 
 \end{enumerate}

を算出し、代理変数として用いる。$\mathbf{X}_{it}$はコホートの傷病分類、$\textit{Aid}_{it}$は医療扶助の受給、$\textit{Hosp}_{it}$は入院治療を受けたことを表すダミー変数、$\textit{Year}_{it}$は年度ダミーである。関心のある変数は交差項の係数$\beta_5$、及び$\beta_6$である。各モデルの二つの係数が正値を取り、統計的に有意であるならば、給付のシステムによってモラルハザードが存在していることが示唆される。

\section{応用研究の可能性}

公的扶助を財・サービスの提供者に直接支払う形で行う給付形態は、日本では医療扶助以外の給付にも応用されており、本研究の分析方法を対象を変更して行うことで、異なる用途に対する給付システムの効果を一般に検定するのに有用であろう。

一方、過剰診療の存在が認められなかった場合は、この直接給付の制度をどこまで応用するかの検討が必要になる。サービスの一定期間あたりの対価が大きく変動しない住宅扶助のような給付にはこの制度が有効で、実際に日本でも代理納付制度が導入されている。ただし、こうした給付制度は、受給者の経済活動を制限する側面があり、必要最低限の利用に留めるべきである。

さらに、結果の頑健性を担保する研究として、傷病と医療扶助、あるいは個別の厚労省の分類による、診療行為との交差項を導入したDID分析や、固定効果モデルを用いた分析を利用することも、重要な成果につながる可能性がある。

\section{参考文献}

 \begin{itemize}
 
 \item Friedman(1962) ``Capitalism and Freedom''
 
 \item Lemieux \& Milligan(2008)''Incentive effects of social assistance: A regression
discontinuity approach''
 
 \item Bargain \& Doorley(2011) ``Caught in the trap? Welfare's disincentive and the labor supply of single men''
 
 \item 川口大司 『日本の労働市場』
 
 \item 東京大学出版会 『生活保護の経済分析』
 
 \end{itemize}

\end{document}