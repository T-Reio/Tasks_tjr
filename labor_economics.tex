\documentclass[dvipdfmx, 12pt]{article}
\usepackage{mathpazo}
\usepackage{amsmath,amssymb}
\usepackage{array}
\usepackage[hiresbb]{graphicx}
\usepackage{tikz}
\usepackage{textcomp}
\usepackage{dcolumn}
\usepackage{here}
\usepackage{lscape}
\usepackage[top=25truemm,bottom=30truemm,left=25truemm,right=25truemm]{geometry}
\begin{document}

\parindent = 0pt

\title{Labor Economics}
\author{}
\date{}
\maketitle

\section{Labor Supply}

\section{Labor Demand}

\section{Equilibrium}

\section{Human Capital}

\section{Compensating Wage Differential}
pp88-94

\section{Monopsony}
pp95-99

\section{Discrimination}
pp100

\section{Asymmetric Information}

\section{Moral Hazard}

\subsection{Principal-Agent Model with Moral Hazard}
pp115-118

: ex post asymmetric information

$\rightarrow$ workers choose her/his effort level after the contract.

\begin{enumerate}
  \item Agent

  Assume negatibe exponential utility:

  \begin{align*}
    U\left( W - \dfrac{c e^2}{2} \right) &= - \exp \left\{ -a \left( W - \dfrac{c e^2}{2} \right)\right\} \\
    \text{where}& \\
    U: & \text{CARA function} \\
    a: & \text{ coefficient of absolute risk-aversion} \\
    W: & \text{wage schedule} \\
    c: & \text{effort cost constant}
  \end{align*}

  $W$ is defined by the fixed wage $w$, performance wage $b$ and productivity $y$ as follows:

  \begin{align*}
    W &= w + by \\
    & \text{where} \\
    y &= e + \epsilon, \epsilon \sim N(0, \sigma^2)
  \end{align*}

  The second equation implies that principal cannot monitor the effort level of the agent $e$. Substituting this into the first one, $W = w + b(e + \epsilon)$.

  The object function of maximization problem of the agent is:

  \begin{align*}
    \max_e EU &= E \left\{ -\exp \left[-a\left(w + b (e + \epsilon) - \dfrac{ce^2}{2}\right)\right]\right\} \\
    &= E \left\{ -\exp \left[-a\left(w + be - \dfrac{ce^2}{2}\right) -ab \epsilon \right]\right\} \\
    &= E \left\{ -\exp \left[-a\left(w + be - \dfrac{ce^2}{2}\right) \right] \exp (- ab \epsilon)\right\} \\
    &= -\exp \left[-a\left(w + be - \dfrac{ce^2}{2}\right) \right] E[\exp (- ab \epsilon)]
  \end{align*}

  Since $\exp (-ab \epsilon)$ follows log-normal distribution, the mean value is $\exp \left( - \dfrac{a^2 b^2 \sigma^2}{2}\right)$.

  Thus, the maximization problem is rewritten as follows:

  \begin{align*}
    \max_e -\exp & \left[-a \left( w + be -\dfrac{c e^2}{2} - \dfrac{a b^2 \sigma^2}{2} \right)\right] \\
    & \Longrightarrow e = \dfrac{b}{c}
  \end{align*}

  If $b$ goes up, then the effort level goes down. When $c$ increases, $e$ decreases.

  \item Principal

  The xProfit of the firm:

  \begin{align*}
    E(y-W) &= E[y - (w + by)] \\
    &= E[(1-b)(e + \epsilon) - w] \\
    &= (1 - b)e - w \hspace{1.5zw}(\because E(\epsilon) = 0)
  \end{align*}
\end{enumerate}


\subsection{Double Moral Hazard}
pp119-124

\subsection{Shirking Model}

pp125-130

$\rightarrow$ deferred payment (Lazear, 1979, 1981) $\rightarrow$ efficiency wage model (Shapiro \& Stiglitz, 1984, AER)

\begin{tabular}{cc}
  \begin{minipage}{0.5\textwidth}
    \begin{tikzpicture}[domain = 0:4, samples = 200, >= stealth]
        \draw[->](-0.5, 0) -- (4.2, 0) node[right]{$x$};
        \draw[->](0, -0.5) -- (0, 3.7) node[above]{$u(x)$};
        \draw[-](2.2, -0.1) -- (2.2, 0.1);
        \draw[domain=0:2.2,samples=200,>=stealth] plot (\x, {sqrt(\x)});
        \draw[domain=2.2:4.1,samples=200,>=stealth] plot (\x, {sqrt(\x) + 0.8});
        \draw (0, 0) node[below left]{O};
        \draw (2.2, -0.3) node {$r$};
      \end{tikzpicture}
      \scriptsize
  \end{minipage} &
  \begin{minipage}{0.5\textwidth}
      \begin{tikzpicture}[domain = 0:4, samples = 200, >= stealth]
        \draw[->](-0.5, 0) -- (4.2, 0) node[right]{$L$};
        \draw[->](0, -0.5) -- (0, 3.7) node[above]{$w$};
        \draw[-](2.2, -0.1) -- (2.2, 0.1);
        \draw[domain=0.2:4.1,samples=200,>=stealth] plot (\x, 2/3*\x + 0.2) node[right]{$LS$};
        \draw[domain=0.2:4.1,samples=200,>=stealth] plot (\x, -2/3 *\x + 3.2) node[right]{$LD$};
        \draw (0, 0) node[below left]{O};
        \draw (2.2, -0.3) node {$L^*$};
      \end{tikzpicture}
      \scriptsize
    \small

    $w^*$ denotes the market clearing wage. When $w = w_E$, then, there exists involuntary worker.
  \end{minipage}
\end{tabular}

\subsubsection{Value Function}

Agents(workers)' value function is:

\begin{itemize}
  \item No shirking
  \begin{align*}
    V_t = w_t &- c_t + \delta [(1 - q)\max(V_{t+1}, V_{t+1}^s) + q \bar{V}_t] \\
    \text{where }& \\
    V_t &: \text{ the value of working hard at time }t \\
    w_t &: \text{ wage} \\
    c_t &: \text{ effort cost} \\
    \delta &: \text{ discont factor} \\
    q &: \text{ exogeneous probability separation} \\
    \bar{V}_t &: \text{ value of outside opportunity} \\
  \end{align*}

  \item Shirking
  \begin{align*}
    V_t^s = w_t &+ (1 - p) \{ \delta' (1 - q) \max (V_{t+1}, V_{t+1}^s) + q \bar{V}_t \} + P \delta \bar{V}_t \\
    \text{where }& \\
    p: & \text{ probability of being detected and fired}
  \end{align*}
\end{itemize}

\begin{enumerate}
  \item Incentive Compatible

  To prevent workers from shirking, the following condition must hold:
  \[
  V_t \geq V_t^s \text{ for } \forall t \geq 0
  \]
  The inequation is rewritten as:

  \begin{align*}
    V_t - V_t^s = -c + p \delta [(1 - q) \max (V_{t+1}, V_{t+1}^s) + q \bar{V}_{t+1}] - p \delta \bar{V}_{t+1} &\geq 0 \\
    p \delta (1 - q)(V_{t+1} - \bar{V}_{t+1}) &\geq c \\
    \Longrightarrow V_{t+1} - \bar{V}_{t+1} \geq \dfrac{c}{p \delta (1 - q)}, & \forall t \geq 0
  \end{align*}

  \begin{itemize}
    \item Incentive mechanism is forward looking.

    \item To give the worker an incentive to work hard today ($t$), the worker expects the positive rent tomorrow ($t+1$).

    \item $w_t$ does not affect effort level at $t$ ($\because w_t$ is canceled out by $V_t - V_t^s$). The effort level at $t$ comes from the prospect of the gain at $t+1$.

    \item The wage is NOT important for the incentive, but for the contract.
  \end{itemize}

  \item Participation Constraint

  \begin{align*}
    V_t \geq \bar{V}_t \hspace{1.5zw} & \\end{align*} \forall t \geq 0
  \end{align*}

  By the incentive compatible, $V_k \geq \bar{V}_k \forall k >0$ is satisfied($\because \frac{c}{p \delta (1 - q)} > 0$ by definition)

  Thus, Participation only requires $V_0 \geq \bar{V}_0$
\end{enumerate}

In sum, the set of feasible contract is:

\begin{align*}
  \mathbb{P} &= \left\{ \Pi_t, V_t | \pi_t \geq \bar{\pi}_t, V_{t+1} - \bar{V}_{t+1} \geq \dfrac{c}{p \delta (1 - q)}, V_0 \geq \bar{V}_0, \forall \geq 0 \right\} \\
  \Pi_t &= y_t - w_t + \delta [(1 - q) \pi_{t+1} + q \bar{\Pi}_{t+1}]
\end{align*}
$\Pi_t$ denotes the value of operation at period $t$.

Now we define the total surplus.

\begin{align*}
  S_t & \equiv (V_y - \bar{V}_t) + (\Pi_t - \bar{\Pi}_t)
\end{align*}
Firms make a contract iff
\begin{align*}
  \Pi_t - \bar{\Pi}_t &= S_t - (V_t - \bar{V}_t) \geq 0
\end{align*}
the set of feasible contract can be rewritten as follows:
\begin{align*}
  \mathbb{P} &= \left \{ \pi_t, V_t | S_{t+1} \geq V_{t+1} - \bar{V}_{t+1} \geq \dfrac{c}{p \delta (1 - q)}, s_0 \geq V_0 - \bar{V}_0 \geq 0, \forall t \geq 0 \right \} \\
  & \Longrightarrow \begin{cases}
    S_0 \geq 0, & \\
    S_{t+1} \geq \dfrac{c}{p \delta (1 - q)} & \forall t \geq 0
\end{cases} \hspace{2zw}: \text{ Implicit self-enforcing contract. }
\end{align*}

To maximize $\Pi_t$: minimize $V_t$ subject to the implicit self-enforcing contract.

\begin{align*}
  V_0 - \bar{V}_0 = 0 & \Rightarrow \text{then, } \Pi_0 - \bar{\Pi}_0 = S_0 \\
  V_{t + 1} - \bar{V}_{t+1} = \dfrac{c}{p \delta (1 - q)} & \Rightarrow \text{ then, } \Pi_{t+1} - \bar{\Pi}_{t + 1} = S_{t+1} - \dfrac{c}{p \delta (1 - q)}
\end{align*}
(note that $s_{t+1} \equiv (\Pi_{t+1} - \Pi_{t+1}) + V_{t + 1} - \bar{V}_{t+1})$)

\begin{align*}
  \forall t \geq 0, & \\
  V_0 &= \bar{V}_0 \\
  V_{t+1} &= \bar{V}_{t+1} + \dfrac{c}{p \delta (1 - q)} \\
  V_t &= w_t - c + \delta [(1 - q) \max(V_{t+1}, V_{t+1}^s) + q \bar{V}] \\
  w_0 &= \bar{V}_0 - \delta \bar{V}_1 + c \dfrac{c}{p} \\
  w_{t + 1} &= \bar{V}_{t + 1} - \delta \bar{V}_{t+2} + c + \dfrac{c}{p} \left[ \dfrac{1}{\delta (1 - q)} - 1 \right]
\end{align*}

\subsubsection{Two Period}

\begin{align*}
  w_0 &= \bar{V}_0 - \delta \bar{V}_1 + c \dfrac{c}{p} \\
  w_{1} &= \bar{V}_{1} - \delta \bar{V}_{2} + c + \dfrac{c}{p} \left[ \dfrac{1}{\delta (1 - q)} - 1 \right]
\end{align*}

Assume no capital accumulation and

\begin{align*}
  \bar{V}_t &= \sum_{i = 0}^{\infty} (\bar{w}_{t + 1}), \forall t \geq 0 \\
  \bar{V}_0 &- \delta \bar{V}_1 = \bar{w}_0 - c \\
  \bar{V}_1 &- \delta \bar{V}_2 = \bar{w}_1 - c
\end{align*}

Suppose $\bar{w}_0 = \bar{w}_1 = \bar{w}_2 = \ldots =\bar{w}$,

\begin{align*}
  w_0 &= \bar{w} - \dfrac{c}{p} \\
  w_1 &= \bar{w} + \dfrac{c}{p} \left[ \dfrac{1}{\delta (1 - q) - 1} \right]
\end{align*}

\subsection{Efficiency wage Model}

pp130-136

Shapiro and Stiglitz (1984, AER).

\begin{itemize}
  \item Involuntary unemployment

  \item wage downward rigidity

  \item keep the wage high to prevent the employee from shirking.
\end{itemize}

\subsubsection{stationary case}

\begin{align*}
  V_0 = V_1 = V_2 = \ldots = V_t = V \\
  \bar{V}_0 = \bar{V}_1 = \bar{V}_2 = \ldots = \bar{V}_t = \bar{V}
\end{align*}

By $w_{t + 1} = \bar{V}_{t + 1} - \delta \bar{V}_{t+2} + c + \frac{c}{p} \left[ \frac{1}{\delta (1 - q)} - 1 \right]$,

\[
w = (1 - \delta) \bar{V} + c + \dfrac{c}{p} \left[ \dfrac{1}{\delta (1 - q) - 1} \right]
\]

By $V - \bar{V} = \frac{c}{p \delta (1 - q)} > 0$, $V > \bar{V}$

: involuntary unemployment occurs.

Define $\bar{V} = z + \delta [sV + (1 - s) \bar{V}]$, where $z$ and $s$ is linear utility per period: unemployment benefit, and probability of finding a job, respectively.

Then,

\begin{align*}
  (1 - \delta) \bar{V} &= z + \dfrac{sc}{p (1 - q)} \\
  w = z + c + \dfrac{c}{p} \left[ \dfrac{1}{1 - q} \left( s + \dfrac{1}{\delta} - 1 \right) \right]
\end{align*}

Under steady state, $s(N-L) = qL$ holds: \# of the worker fired and that of those who find a job is equal. Then,

\[
s = \dfrac{q L}{N - L} < 1
\]

%図いれろ

\[
w = z + c + \dfrac{c}{p} \left[\left( \dfrac{1}{1-q} \right) \left(\dfrac{qL}{N - L} + \dfrac{1}{\delta} \right) - 1 \right] : \text{incentive curve}
\]

Firm side:

\[
\Pi = y - w + \delta [(1 - q) \Pi + q \bar{\Pi}]
\]

Free entry condition:$\Pi = \bar{\Pi} \equiv c_k$ Fixed entry cost.

\[
w^* = y -(1 - \delta)c_k
\]

\section{Search Model}

\subsection{Sequential Search Model: Individual's decision making}
pp. 134 - 141

\subsubsection{Settings}

Consider a representative agent (infinite living) to maximize the lifetime utility
\[
E \sum_{t=0}^{\infty} \delta^t u_t.
\]
Workers realize the time-invariant wage offer distribution (Firm's strategy is given).

\[
\text{draw }w \text{ with probability of }\alpha \Rightarrow
\begin{cases}
  \text{accept } (w \geq w_R) & \rightarrow \text{ finish searching} \\
  \text{reject } (w < w_R)& \rightarrow \text{ go to the next round}
\end{cases}
\]

\subsubsection{Value Function}

\begin{enumerate}
  \item Employment

  \[
  W(w) = w + \delta [(1-\lambda) W(w) + \lambda \bar{u}]
  \]


  where $lambda$ denotes the prob. of unemployment.

  $\bar{u}$ is the value of unemployment

  \item Unemployment

  \[
  \bar{u} = z + \delta \{ \alpha E \max [W(w), \bar{u}] + (1-\alpha)\bar{u} \}
  \]

  $z$ is the value of this period with unemployed.

  $\alpha$ denotes the prob. of get offer.


\end{enumerate}

\begin{itemize}
  \item Reservation Wage Property

  \[
  W(w_R) = \bar{u}
  \]

  Thus, the worker accept the offer if

  \[
  W(w) > \bar{u} \Leftrightarrow w > w_R
  \]
  And reject otherwise.

\end{itemize}

By (1), when $w = w_R$,

\begin{align*}
  W(w_R) &= w_R + \delta [(1-\lambda) W(w_R) + \lambda \bar{u}] \\
  \bar{u} &= w_R + \delta \bar{u} \hspace{2zw}(\because W(w_R) = \bar{u}) \\
  \Rightarrow & W(w_R) = \bar{u} = \dfrac{w_R}{1-\delta}
\end{align*}

By (2),
\begin{align*}
  &\bar{u} = z + \alpha \delta \left\{ \int_{\underline{w}}^{w_R} \bar{u}dF(w) + \int_{w_R}^{\bar{w}} W(w)dF(w) \right\} + \delta (1 - \alpha) \bar{u} \\
  & (\because E \max[W(w), \bar{u}] \text{ is the expected value of the maximized utility}) \\
  \\
  \Longrightarrow &(1 - \delta)\bar{u} = z + \alpha \delta \int_{w_R}^{\bar{w}}[W(w) - \bar{u}]dF(w) \\
  & \left(\because \int_{\underline{w}}^{w_R} \bar{u}dF(w) = \bar{u} \left[ 1 - \int_{w_R}^{\bar{w}} dF(w) \right] \right) \\
\end{align*}
Note that $(1 - \delta)\bar{u} = w_R$, and

\begin{align*}
  W(w) - \bar{u} &= \dfrac{w + \delta \lambda \bar{u}}{1 - \delta (1 - \lambda)} -\bar{u} \\
  &= \dfrac{w - (1 - \delta) \bar{u}}{1 - \delta(1 - \lambda)} \\
  &= \dfrac{w - w_R}{1 - \delta(1 - \lambda)},
\end{align*}

we obtain

\[
w_R - z = \dfrac{\alpha \delta}{1 - \delta(1- \lambda)} \int_{w_R}^{\bar{w}} (w - w_R) dF(w)
\]

The left-hand side indicates the marginal cost of additional search, while the right-hand side denotes the expected capital gain: expected benefit of additional search.

\subsubsection{Competitive Static Analysis}

\begin{align*}
  \dfrac{d w_R}{dz} > 0 & & \dfrac{d w_R}{d \alpha} >0 \\
  \dfrac{d w_R}{d \delta} > 0 & & \dfrac{d w_R}{d \lambda} >0
\end{align*}

\subsubsection{Expected Duration of Search}

Define the Hazard rate $\tau \equiv \alpha [1 - F(w_R)]$

The probability of finishing search at period $t$ is $(1-\tau)^{t-1} \tau$

Then, the expected time to get a job is:

\begin{align*}
  T &= \sum_{t=1}{\infty} t \times (1 - t)^{t-1} \times \tau \\
  &= \dfrac{1}{\tau}
  & T \text{ follows the negative binominal distribution.}
\end{align*}

By the static analysis, When $z$ goes up, then $w_R$ goes up. $F(w_R)$ is increasing in $w_R$, so by definition, $\tau$ decreases. As a result, $T$: the expected duration gets longer.

\begin{itemize}
  \item Unemployment Insurance

  If $w_R$ goes up, then the accepted wage also increases and it varies over serching time. With the existance of unemployment insurance, even when the worker gets a new job early, s/he tries to put off the start time, in order to receive the full payment (90 days).
\end{itemize}

\subsubsection{Continuous-Time Model}

We begin with an environment whose multiple jobs arrive at unemployed with probabilty of $a(n, \Delta t)$, for $\Delta$.

\[
a(n, \Delta t) = \dfrac{e^{- \alpha \Delta t} (\alpha \Delta t)^n}{n!}: \text{Poisson Procedure}
\]

\begin{itemize}
  \item Assumption

  $\Delta t'\perp$ the arrival probability of the $n$th job offer.
\end{itemize}

Then, we can say

\[
a(n, \Delta t) = \begin{cases}
  \alpha \Delta t + O(\Delta t) & \text{ if }n = 1 \\
  O(\Delta t) & \text{ if }n \geq 2
\end{cases}
\]
\[
\text{where } \left( \lim_{\Delta t \to \infty} \dfrac{O(\Delta t)}{\Delta t} = 0 \right), O(\Delta t) + O(\Delta t) = O(\Delta t)
\]

$O(.)$ denotes a tiny change.

\begin{itemize}
  \item Value Function

  \begin{align*}
    &\bar{u} = \left( \dfrac{1}{1 + r \Delta t} \right) \left \{ z \Delta t + \sum_{n = 1}^{\infty} a(n, \Delta t) E \max[W(w), \bar{u}] + a(0, \Delta t) \right \} \\
    & \text{RHS is rewritten as follows.} \\
    & \left( \dfrac{r \Delta t}{1 + r \Delta t} \right) \bar{u} = \left( \dfrac{1}{1 + r \Delta t} \right) \left \{ z \Delta t + \sum_{n = 1}^{\infty} a (n, \Delta t) \times [E \max [W(w), \bar{u}] - \bar{u}] \right\}
  \end{align*}

  \begin{enumerate}
    \item Multiply the both-hand side by $(1 + r \Delta t)$

    \item Devide by $\Delta t$

    \item $\Delta t \to 0$
  \end{enumerate}

  yields

  \begin{align*}
    r \bar{u} &= z + \alpha \left \{ E \max[W(w), \bar{u}] - \bar{u} \right \} \\
    & \text{or} \\
    r \bar{u} &= z + \alpha \int_{w_R}^{\bar{w}} [W(w)- \bar{u}] dF(w)
  \end{align*}

  \begin{align*}
    & W(w) = \left( \dfrac{1}{1 + r \Delta t} \right ) \{ w \Delta t + \lambda \Delta t \bar{u} + (1 - \lambda \Delta t)W(w) \} \\
    & r W(w) = w + \lambda [\bar{u} - W(w)]
  \end{align*}

  \begin{align*}
    W(w) &= \dfrac{w + \lambda \bar{u}}{r + \lambda} \\
    W(w_R) &= \bar{u}
  \end{align*}

  \begin{align*}
    W(w) - \bar{u} &= \dfrac{w - r \bar{u}}{r + \lambda} \\
    & = \dfrac{w - w_R}{r + \lambda}
  \end{align*}

  \[
  w_R = z + \dfrac{\alpha \delta}{1 - \delta (1 -\lambda)} \int_{w_R}^{\bar{w}} (w - w_R) dF(w)
  \]

  (cf. instantaneous reservation wage)

  \[
  w_R  = z + \dfrac{\alpha \delta}{1 - \delta(1- \lambda)} \int_{w_R}^{\bar{w}} (w - w_R) dF(w)
  \]

\end{itemize}

\subsection{Search Effort}

pp143 - 146

\begin{enumerate}
  \item reseavation wage: constant (endogeneous)

  \item search effort exists
\end{enumerate}

\subsubsection{Value Functions}

\begin{itemize}
  \item No search

  \begin{align*}
    r W(w) &= w + \lambda [\bar{u} - W(w)] \\
    \\
    \text{where} \\
    & r: \text{ discount factor} \\
    & w: \text{ linear instantaneous utility} \\
    & \lambda: \text{ probability of separation} \\
    \\
    & \bar{u} - W(w) \text{ stands for the capital loss.} \\
    & \text{Note that there is no search cost.}
  \end{align*}

  \item Search

  \begin{align*}
    r \bar{u} &= \max_e \left\{z - c(e) + \alpha(e) \int_{w_R}^{\bar{w}} [W(w) - \bar{u}] dF(w) \right\} \\
    \\
    &\text{where} \\
    & z: \text{ instantaneous utility} \\
    & c(e): \text{search cost, note that }c'(.) > 0, c"(.) > 0, c(0) = 0. \\
    & \alpha(e): \text{job arrival rate, note that } \alpha'(.) > 0, \alpha"(.) < 0, c(0) = 0
  \end{align*}

  FOC yields

  \begin{align*}
    (r + \lambda)c'(e) = \alpha'(e) \int_{w_R}^{\bar{w}}[W(w) - \bar{u}]dF(w)
  \end{align*}

\end{itemize}

Then, we obtain:
\[
\begin{cases}
  w_R = z - c(e^*) + \dfrac{\alpha (e^*)}{r + \lambda} \int_{w_R}^{\bar{w}} (w - w_R) d F(w) \\
  (r + \lambda)c'(e^*) = \alpha(e^*) \int_{w_R}^{\bar{w}} (w - w_R) dF(w)
\end{cases}
\]

\subsubsection{Competitive Static Analysis}

\[
\begin{bmatrix}
  a_{11} & a_{12} \\ a_{21} & a_{22}
\end{bmatrix}
\begin{bmatrix}
  \frac{d w_R}{d z} \\ \frac{d e^*}{dz}
\end{bmatrix}
= \begin{bmatrix}
  B_1 \\ B_2
\end{bmatrix}
\]

\begin{align*}
  \dfrac{d w_R}{d z} > 0, &\dfrac{d e^*}{dz} < 0
\end{align*}

\subsubsection{On-the-job Search}

Both unemployed and employed workers search, with the different rate of arriving rate: $\alpha_0$ for unemployed and $\alpha_1$ for employed.

\begin{itemize}
  \item If there is severe time constraint for the employed, then $\alpha_0 > \alpha_1$.

  \item If the employed worker have some connection about searching, then $\alpha_0 < \alpha_1$
\end{itemize}

The value function is:

\begin{align*}
  r \bar{u} &= z + \alpha_0 \int_{w_R}^{\bar{w}} [W(w)- \bar{u}] dF(w) \\
  r \bar{u} &= w + \alpha_1 \int_{w_R}^{\bar{w}} [W(w)- \bar{u}] dF(w) + \lambda [\bar{u} - W(w)]
\end{align*}

By the reservation wage property,

\begin{align*}
  r \bar{u} &= r W(w_R) = w_R + \alpha_1 \int_{w_R}^{\bar{w}}[W(w') - \bar{u}]dF(w') \\
  w_R &+ \alpha_1\int_{w_R}^{\bar{w}}[W(w') - \bar{u}]dF(w') = z + \alpha_0 \int_{w_R}^{\bar{w}} [W(w)- \bar{u}] dF(w) \\
  \\
  &\text{assume } w \text{ and } w' \text{ follow the same distribution. Then, }\\
  w_R &= z + (\alpha_0 - \alpha_1) \int_{w_R}^{\bar{w}} [W(w)- \bar{u}] dF(w)
\end{align*}

\subsubsection{Implication}

\begin{enumerate}
  \item Relationship of $\alpha_i$ and $z$
  \begin{align*}
    \alpha_0 > \alpha_1 &\Longrightarrow w_R > z \\
    \alpha_0 < \alpha_1 &\Longrightarrow w_R < z
  \end{align*}

  \item $z$ and $w_R$

  \[
  z \text{ increases } \Longrightarrow w_R \text{ increases } \Longrightarrow \text{ average } w \text{ goes up}
  \]

  Then, the probability that a higher wage offer than the current wage $w$ goes down. As a result, turnover of the employed workers gets less likely to occur.

  \item positive assotiation between the wage level and the firm size:

  High wage firms are more attractive to the workers, so those in lower wage firms move to them. Then, the size of the firm with high wage gets larger.
\end{enumerate}

\subsection{Duration Analysis}

pp146 - 152

Let $T(\geq 0)$ denote the search duration. The cumulative distribution function of the duration is:

\[
F(t) = \text{Pr}(T \leq t)
\]

Then, the survivor function: the probability of surviving after time $t$ is
\[
s(t) \equiv 1 - F(t)
\]

\subsubsection{Hazard function}

Instantaneous rate of leaving from unemployment pool.

\[
\lambda(t) = \lim_{h \to 0} \dfrac{\text{Pr}(t \leq T \leq t + h | T \geq t)}{h}
\]

(If $\lambda'(t) < 0$, the harzard rate has negative dependence on search time, otherwise if $\lambda'(t) > 0$.)

\begin{align*}
  &\text{Pr}(t \leq T \leq t + h | T \geq t) \\
  &= \dfrac{\text{Pr}(t \leq T \leq t + h)}{\text{Pr}(T \geq t)} \\
  &= \dfrac{F(t + h) - F(t)}{1 - F(t)}
\end{align*}

Substituting this into $\lambda$,

\begin{align*}
  \lambda(t) &= \lim_{h \to 0}\dfrac{F(t + h) - F(t)}{h} \times \dfrac{1}{1 - F(t)} \\
  & = \dfrac{f(t)}{1 - F(t)} \\
  & = \dfrac{f(t)}{s(t)} \\
  &\text{or} \\
  \lambda(t) &= -\dfrac{d \ln s(t)}{dt} \\
  & \left(\because \dfrac{d \ln s(t)}{d s(t)} \times \dfrac{d s(t)}{d t} = \dfrac{- f(t)}{s(t)}  \text{ by definition} \right)
\end{align*}
$\lambda'(t) > 0$: positive duration dependence (Weibull function)

\begin{table}[H]
  \caption{Two Kinds of Hazard Functions}
  \label{Hazard_Func}
  \begin{tabular}{ccccl} \hline
    & $f(t)$ & $F(t)$ & $\lambda(t)$ & \\ \hline
    exponential & $\lambda \exp (-\lambda t)$ & $1 - \exp (- \lambda t)$ & $\lambda$ & constant hazard rate \\
    Weibull & $\gamma \alpha t^{\alpha - 1} \exp (-\gamma t^{\alpha})$ & $1 - \exp(- \gamma t^{\alpha})$ & $\gamma \alpha t^{\alpha - 1}$ & positive duration dependence \\ \hline
  \end{tabular}
\end{table}

\begin{itemize}
  \item Parametric Specification

  Assume the searching duration follows the Weibull distribution.

  \begin{align*}
    \lambda(t: X) &= \exp(X \beta) \alpha t^{\alpha - 1} \\
    & X \text{ denotes the time-invariant covariate} \\
    \ln \lambda (t: X) = X \beta + \ln (\alpha t^{\alpha - 1})
  \end{align*}

  \item Semi-Parametric estimation

  (Cox's proportional hazard): do not specify the distribution.

  \[
  \lambda (t: X) = \kappa (x) \lambda_0 (t)
  \]

  $kappa(.)$ is the non-negative function of $X$: $\exp (X \beta)$.

  $\lambda_0$ stands for the baseline function $> 0$
\end{itemize}

\subsubsection{Data}

\begin{enumerate}
  \item Right-censoring

  An individual enters the initial state: visit the job-search office during the interval $a_i \in [0, b]$

  Then, those who does not finish her/his search until the end of the investigation cannot be observed: the obtained data is truncated.

  \begin{itemize}
    \item MLE

    \begin{align*}
      & a_i \in [0, b] \\
      t^*_i &: \text{duration of search} \\
      X_i &: \text{vector of observed covariates} \\
      c_i &: \text{censoring time: time left for observation for the individual }i
    \end{align*}

    Assume
    \[
    F(t^* | X_i, a_i, c_i) = F(t | X_i)
    \]
    the duration is not affected by the time starting search.

    \begin{itemize}
      \item When the observation is complete one, then $f(t^*_i | X_i: \theta)$

      \item If incomplete, $1 - F(c_i | X_i, \theta)$
    \end{itemize}

    The conditional likelihood function is:

    \[
    f(t^*_i)^{d_i} [1 - F(c_i | X_i, \theta)]^{1 - d_i}
    \]
    \[
    d_i \text{ is the indicator}
    \]

    Therefore, the log-likelihood function
    \begin{align*}
      L(\theta) = \sum^N \{ d_i - F (c_i | X_i, \theta) + (1 - d_i) \ln [1 - F(c_i | X_i, \theta)] \}
    \end{align*}

    Maximizing this, we obtain consistent and asymptotical normal estimator $\hat{\theta}$.

    \[
    (\sqrt{\hat{\theta}} - \theta) \sim N \left( 0, - \left[\dfrac{1}{N} \dfrac{\partial^2 L(\hat{\theta})}{\partial \hat{\theta} \partial \hat{\theta}'} \right]\right)
    \]
  \end{itemize}

  \item Left censoring

  We missed observation of individuals who are not searching for a job at the point of $b\rightarrow$ sample selection problem may occur

  We observe observations iff an individual is still unemployed at $b$

  \begin{align*}
    a_i & + t^*_i \geq b \\
    \text{ or}& \\
    t^*_i & \geq b - a_i
  \end{align*}

  The probability is obtained as follows:

  \begin{align*}
    &\text{Pr}(t^*_i \geq b - a_i | X_i, a_i, c_i) \\
    = & 1 -F(b - a_i | X_i) \hspace{2zw} \because \text{assumption}
  \end{align*}

  Therefore, the likelihood function is:

  \[
  \dfrac{f(t^*_i | X_i \theta)^d_i [1 - F_(c_i | X_i, \theta)]^{1 - d_i}}{1 - F(b - a_i | X_i, \theta)}
  \]

\end{enumerate}

\subsubsection{Empirical Example}

\section{Equilibrium Matching Model}


\end{document}
