\documentclass{jsarticle}
\usepackage{mathpazo}
\usepackage{amsmath,amssymb}
\begin{document}

\title{Data Management Term-Paper}
\author{経済学研究科 23A18014 丹治伶峰}
\date{}
\maketitle

\Large

\textbf{医療扶助給付システムの設計 : 過剰診療の実態調査}

\large

\section{Abstract}

生活保護の給付システムの設計上起こりうるモラルハザードの影響を測定し、社会的損失を抑え、予算制約の下で可能な限り効率的な給付システムをデザインすることを目的とする。その中でも、本調査では医療扶助に焦点を当て、給付が医療サービスの提供者に対して直接行われることによって起こるとされる、過剰診療の問題の実態を明らかにする。

厚生労働省による統計を用いて、傷病レベルの治療点数などを診療費用の代理変数としたDID分析を行った結果、医療サービス提供者が過剰診療を行っていることは統計的におおむね観察されず、厳密な点数制度を用いた診療報酬の決定システムなどがこれを抑制している可能性が指摘された。ここから、本研究では医療扶助の直接的な給付の有効性を認め、他の生活保護給付に対しても応用を検討することを示唆している。

\section{Introduction}

本研究では、日本の生活保護支給システムが経済主体の意思決定に与える影響を特定し、社会的な損失を最小限に抑える制度設計を行うための示唆的な結果を得ることを目的としている。現行制度では、医療サービスに係る費用を補助する医療扶助が、生活保護受給者を経ずに直接サービス提供者に対して支払われる給付システムを採用している。この制度は、受給者へ現金を渡さないことで、受給者が給付金を医療費以外の財、サービスの消費活動に使い込み、対価が支払えなくなったり、必要な医療行為を受けようとしなくなる問題を回避することを意図している。一方で、サービス提供者にとっては、高額な請求を行っても確実に支払いを受けられることを悪用し、比較的重要度の低い検査や治療を提供することで収入を増やすインセンティブを与える。監視費用や情報の非対称性の問題から、両者の損失を完全に排除することは難しいため、実証的な分析によって各制度の引き起こす社会的損失を特定し、最適な制度設計を模索する必要がある。

\section{Background \& Model}

伝統的経済学のフレームワークでは、生活保護に代表される政府による所得再分配制度は、消費の余暇に対する限界代替率の低い労働者の就業インセンティブを低下させ、結果として当初意図されている額よりも大きな政府支出を強いる可能性を指摘している。Lemieux \& Milligan(2008)、Bargain \& Doorley(2011)は、回帰不連続デザイン(RDD)を用いた分析により、それぞれカナダ、及びフランスの特定コーホートへの生活保護の不連続な給付システムが、労働者の就業意欲を有意に低下させることを明らかにした。

一方で、伝統的経済学では、給付の方法の差異が受給者の経済活動に与える影響について言及していない。Friedman(1969)の生活保護制度の設計に関するフレームワークでは、前述した給付による就業インセンティブの低下に対処する方法として「負の所得税」の導入が提唱されている他、受給者の選好に応じた柔軟な消費を許容するため、現物給付よりも現金給付を行う方が望ましいとされている。しかし、現金給付による公的扶助は、受給者の近視眼的な選好によって奢侈品に対する過剰な消費を招き、結果として必需品の対価が払えなくなったり、身体的なリスクを冒して必要な医療福祉サービスの利用を控える可能性が指摘されている。こうした行動経済学的なアプローチから懸念される問題に対処するため、用途を限定したクーポンによる公的扶助の給付を行う国も存在する。

本研究では、このアプローチをサービスを提供する側の行動決定に拡張し、公的扶助の制度設計が引き起こすモラルハザードの問題を検討する。具体的には、日本の生活保護制度について、医療サービスの対価を基本的に制限なく、受給者への現金給付を通さない形で支給する医療扶助のシステムが、医療サービスの提供者に、医療扶助の受給者に対して必要以上の医療行為を行うことでより多くの利潤を得ようとする「過剰診療」を引き起こしているかどうかを、統計的な分析を用いて検証する。

\section{Data}

厚生労働省が年度ごとに集計、発表している、「医療扶助実態調査」と「社会医療診療行為別統計」を用いてパネルデータを作成し、分析を行う。2つの統計調査はそれぞれ、医療扶助受給者と一般の医療保険給付の利用者について、診療報酬明細書の記載をもとにした、明細書の発行件数、一傷病あたりの診療行為回数、診療報酬点数、117の傷病中分類、入院$\cdot$入院外の分類、これらの医療行為について規定される診療報酬点数についてのデータを集計している。使用する期間は、これらの集計項目を2つの統計間、及び異時点間で追跡可能な形で公表している、2014年から2017年までの4年間とし、各傷病中分類、入院$\cdot$入院外の分類、そして医療扶助の受給の有無による分類をコホートの形成単位とする。ただし、医療扶助実態調査と社会医療診療行為別統計とで分類方法が異なる6分類は除外する。傷病中分類は、(1)感染症及び寄生虫症(2)新生物(3)血液及び造血器の疾患並びに免疫機構の障害(4)内分泌、栄養及び代謝疾患 (5)精神及び行動の障害(6)神経系の疾患(7)眼及び付属器の疾患(8)耳及び乳様突起の疾患(9)循環器系の疾患(10)呼吸器系の疾患(11)消化器系の疾患(12)皮膚及び皮下組織の疾患(13)筋骨格系及び結合組織の疾患(14)腎尿路生殖器系の疾患(15)妊娠,分娩及び産じょく(16)周産期に発生した病態(17)先天奇形,変形及び染色体異常(18)症状,徴候等で他に分類されないもの(19)損傷,中毒及びその他の外因の影響 の19の大分類に対して具体的な傷病名による集計を与える形で行われている(N=1824)。

\begin{table}[!htbp] \centering 
  \caption{} 
  \label{} 
\begin{tabular}{@{\extracolsep{3pt}}lccccccc} 
\\[-1.8ex]\hline 
\hline \\[-1.8ex] 
Statistic & \multicolumn{1}{c}{N} & \multicolumn{1}{c}{Mean} & \multicolumn{1}{c}{St. Dev.} & \multicolumn{1}{c}{Min} & \multicolumn{1}{c}{Pctl(25)} & \multicolumn{1}{c}{Pctl(75)} & \multicolumn{1}{c}{Max} \\ 
\hline \\[-1.8ex] 
実施件数 & 1,822 & 179,967.0 & 606,950.0 & 0.0 & 1,036.0 & 48,928.5 & 8,407,623.0 \\ 
回数 & 1,822 & 3,246,561.0 & 13,133,089.0 & 0.0 & 8,200.0 & 1,524,485.0 & 235,417,950.0 \\ 
点数 & 1,822 & 483,314,589.0 & 1,067,622,211.0 & 0.0 & 10,525,333.0 & 413,860,265.0 & 9,547,193,060.0 \\ 
件数あたり点数 & 1,815 & 24,858.5 & 26,843.5 & 580.4 & 1,487.1 & 43,315.4 & 144,143.7 \\ 
件数あたり回数 & 1,815 & 27.5 & 32.6 & 1.0 & 3.7 & 29.1 & 189.4 \\ 
回数当たり点数 & 1,815 & 1,355.5 & 1,611.7 & 40.5 & 293.7 & 1,809.4 & 13,564.0 \\ 
\hline \\[-1.8ex] 
\end{tabular} 
\end{table} 

\section{Analysis}

DID分析を用いて分析を行う。回帰式は以下の通り。

 \begin{align*}
 y_{it} = \beta_0 + \beta_1 \mathbf{X}_{it} + \beta_2 \textit{Aid}_{it} + \beta_3 \textit{Hosp}_{it} +
 \beta_4 \textit{Year}_{it} + \beta_5 \textit{Aid}_{it} \times \textit{Year}_{it} + \beta_6 \textit{Aid}_{it} \times \textit{Hosp}_{it} 
 \end{align*}

$y_{it}$は時点$t$における医療扶助受給別、及び傷病別コホート$i$における医療支出を表す。医療支出には診療報酬点数、診療実績件数、受診回数を用いて

 \begin{enumerate}
 
 \item 件数あたり点数
 
 \item 回数あたり点数
 
 \item 件数あたり回数
 
 \end{enumerate}

を算出し、代理変数として用いる。$\mathbf{X}_{it}$はコホートの傷病分類、$\textit{Aid}_{it}$は医療扶助の受給、$\textit{Hosp}_{it}$は入院治療を受けたことを表すダミー変数、$\textit{Year}_{it}$は年度ダミーである。関心のある変数は交差項の係数$\beta_5$、及び$\beta_6$である。それぞれの被説明変数についての推定結果を以下の表に示す。

 件数当たり点数を用いた推定では、入院が有意に点数を上昇させたほかはいずれの変数も有意性を示すことがなかった。一方、件数あたりの受診回数に対しては、入院と医療扶助受給の交差項の係数は統計的に有意であったが、その係数は負値を取り、仮説とは逆の結果を示した。また、受診回数あたりの点数は入院と医療扶助受給との交差項の係数が正値で有意性を示し、唯一仮説と整合的な結果が得られた。
 
\begin{table}[!htbp] \centering 
  \caption{} 
  \label{} 
  \small
\begin{tabular}{@{\extracolsep{5pt}}lccc} 
\\[-1.8ex]\hline 
\hline \\[-1.8ex] 
 & \multicolumn{3}{c}{\textit{Dependent variable:}} \\ 
\cline{2-4} 
\\[-1.8ex] & \multicolumn{3}{c}{件数あたり点数} \\ 
\\[-1.8ex] & (1) & (2) & (3)\\ 
\hline \\[-1.8ex] 
 医療扶助受給 & $-$779.148 & $-$492.245 & 7.112 \\ 
  & (1,212.916) & (475.584) & (1,062.068) \\ 
  & & & \\ 
 入院 &  & 45,968.070$^{***}$ & 46,617.850$^{***}$ \\ 
  &  & (475.584) & (670.770) \\ 
  & & & \\ 
 2015 & 209.774 & 205.297 & 378.048 \\ 
  & (1,713.897) & (672.006) & (948.611) \\ 
  & & & \\ 
 2016 & $-$125.476 & $-$67.154 & 438.196 \\ 
  & (1,714.770) & (672.348) & (948.611) \\ 
  & & & \\ 
 2017 & 584.914 & 584.914 & 911.247 \\ 
  & (1,713.757) & (671.951) & (948.611) \\ 
  & & & \\ 
 2014 $\times$ \textit{Aid} &  &  & 655.555 \\ 
  &  &  & (1,344.504) \\ 
  & & & \\ 
 2015$\times$ \textit{Aid} &  &  & 308.623 \\ 
  &  &  & (1,344.615) \\ 
  & & & \\ 
 2016$\times$ \textit{Aid} &  &  & $-$362.496 \\ 
  &  &  & (1,345.308) \\ 
  & & & \\ 
 入院$\times$ \textit{Aid} &  &  & $-$1,308.442 \\ 
  &  &  & (951.605) \\ 
  & & & \\ 
 Constant & 21,812.380$^{***}$ & $-$1,328.572 & $-$1,902.348 \\ 
  & (6,567.954) & (2,586.349) & (2,630.220) \\ 
  & & & \\ 
  傷病をコントロール & Yes & Yes & Yes \\
\hline \\[-1.8ex] 
Observations & 1,815 & 1,815 & 1,815 \\ 
R$^{2}$ & 0.134 & 0.867 & 0.867 \\ 
Adjusted R$^{2}$ & 0.075 & 0.858 & 0.858 \\ 
Residual Std. Error & 25,820.350 (df = 1697) & 10,123.960 (df = 1696) & 10,128.400 (df = 1692) \\ 
F Statistic & 2.253$^{***}$ (df = 117; 1697) & 93.704$^{***}$ (df = 118; 1696) & 90.573$^{***}$ (df = 122; 1692) \\ 
\hline 
\hline \\[-1.8ex] 
\textit{Note:}  & \multicolumn{3}{r}{$^{*}$p$<$0.1; $^{**}$p$<$0.05; $^{***}$p$<$0.01} \\ 
\end{tabular} 
\end{table} 


\begin{table}[!htbp] \centering 
  \caption{} 
  \label{} 
  \small
\begin{tabular}{@{\extracolsep{5pt}}lccc} 
\\[-1.8ex]\hline 
\hline \\[-1.8ex] 
 & \multicolumn{3}{c}{\textit{Dependent variable:}} \\ 
\cline{2-4} 
\\[-1.8ex] & \multicolumn{3}{c}{件数当たり回数} \\ 
\\[-1.8ex] & (1) & (2) & (3)\\ 
\hline \\[-1.8ex]
  医療扶助受給 & $-$37.757$^{***}$ & $-$37.533$^{***}$ & $-$13.945$^{***}$ \\ 
  & (1.204) & (0.831) & (1.402) \\ 
  & & & \\ 
 入院 &  & 35.890$^{***}$ & 58.116$^{***}$ \\ 
  &  & (0.831) & (0.886) \\ 
  & & & \\ 
 2015 & $-$0.803 & $-$0.806 & $-$1.472 \\ 
  & (1.701) & (1.174) & (1.253) \\ 
  & & & \\ 
 2016 & $-$2.012 & $-$1.966$^{*}$ & $-$3.711$^{***}$ \\ 
  & (1.702) & (1.174) & (1.253) \\ 
  & & & \\ 
 2017 & $-$2.022 & $-$2.022$^{*}$ & $-$3.720$^{***}$ \\ 
  & (1.701) & (1.174) & (1.253) \\ 
  & & & \\ 
 2014 $\times$ \textit{Aid} &  &  & $-$3.411$^{*}$ \\ 
  &  &  & (1.775) \\ 
  & & & \\ 
 2015$\times$ \textit{Aid} &  &  & $-$2.071 \\ 
  &  &  & (1.776) \\ 
  & & & \\ 
 2016$\times$ \textit{Aid} &  &  & 0.039 \\ 
  &  &  & (1.776) \\ 
  & & & \\ 
 入院$\times$ \textit{Aid} &  &  & $-$44.732$^{***}$ \\ 
  &  &  & (1.257) \\ 
  & & & \\ 
 Constant & 47.753$^{***}$ & 29.686$^{***}$ & 19.669$^{***}$ \\ 
  & (6.517) & (4.517) & (3.473) \\ 
  & & & \\ 
  傷病をコントロール & Yes & Yes & Yes \\
\hline \\[-1.8ex] 
Observations & 1,815 & 1,815 & 1,815 \\ 
R$^{2}$ & 0.421 & 0.724 & 0.843 \\ 
Adjusted R$^{2}$ & 0.381 & 0.705 & 0.831 \\ 
Residual Std. Error & 25.621 (df = 1697) & 17.683 (df = 1696) & 13.374 (df = 1692) \\ 
F Statistic & 10.557$^{***}$ (df = 117; 1697) & 37.796$^{***}$ (df = 118; 1696) & 74.339$^{***}$ (df = 122; 1692) \\ 
\hline 
\hline \\[-1.8ex] 
\textit{Note:}  & \multicolumn{3}{r}{$^{*}$p$<$0.1; $^{**}$p$<$0.05; $^{***}$p$<$0.01} \\ 
\end{tabular} 
\end{table} 

\begin{table}[!htbp] \centering 
  \caption{} 
  \label{} 
  \small
\begin{tabular}{@{\extracolsep{5pt}}lccc} 
\\[-1.8ex]\hline 
\hline \\[-1.8ex] 
 & \multicolumn{3}{c}{\textit{Dependent variable:}} \\ 
\cline{2-4} 
\\[-1.8ex] & \multicolumn{3}{c}{回数当たり点数} \\ 
\\[-1.8ex] & (1) & (2) & (3)\\ 
\hline \\[-1.8ex] 
  医療扶助受給 & 1,876.828$^{***}$ & 1,886.418$^{***}$ & 1,030.023$^{***}$ \\ 
  & (57.111) & (43.263) & (83.019) \\ 
  & & & \\ 
 入院 &  & 1,536.500$^{***}$ & 629.690$^{***}$ \\ 
  &  & (43.263) & (52.432) \\ 
  & & & \\ 
 2015 & 40.091 & 39.941 & 13.677 \\ 
  & (80.700) & (61.131) & (74.150) \\ 
  & & & \\ 
 2016 & 39.088 & 41.038 & 43.452 \\ 
  & (80.741) & (61.163) & (74.150) \\ 
  & & & \\ 
 2017 & 95.292 & 95.292 & 53.535 \\ 
  & (80.693) & (61.126) & (74.150) \\ 
  & & & \\ 
 2014 $\times$ \textit{Aid} &  &  & $-$83.885 \\ 
  &  &  & (105.096) \\ 
  & & & \\ 
 2015$\times$ \textit{Aid} &  &  & $-$31.304 \\ 
  &  &  & (105.105) \\ 
  & & & \\ 
 2016$\times$ \textit{Aid} &  &  & $-$86.476 \\ 
  &  &  & (105.159) \\ 
  & & & \\ 
 入院$\times$ \textit{Aid} &  &  & 1,825.036$^{***}$ \\ 
  &  &  & (74.384) \\ 
  & & & \\ 
 Constant & 117.566 & $-$655.929$^{***}$ & $-$188.975 \\ 
  & (309.256) & (235.276) & (205.597) \\ 
  & & & \\ 
  傷病をコントロール & Yes & Yes & Yes \\
\hline \\[-1.8ex] 
Observations & 1,815 & 1,815 & 1,815 \\ 
R$^{2}$ & 0.468 & 0.695 & 0.775 \\ 
Adjusted R$^{2}$ & 0.431 & 0.673 & 0.759 \\ 
Residual Std. Error & 1,215.767 (df = 1697) & 920.961 (df = 1696) & 791.708 (df = 1692) \\ 
F Statistic & 12.741$^{***}$ (df = 117; 1697) & 32.705$^{***}$ (df = 118; 1696) & 47.747$^{***}$ (df = 122; 1692) \\ 
\hline 
\hline \\[-1.8ex] 
\textit{Note:}  & \multicolumn{3}{r}{$^{*}$p$<$0.1; $^{**}$p$<$0.05; $^{***}$p$<$0.01} \\ 
\end{tabular} 
\end{table}


\section{Conclusion and Discussion}

本研究で行った三つの分析は、いずれも過剰診療の存在を明示するものではなかった。医療サービスの必要性は検証が難しく、情報の非対称性を完全に排除出来ないことから、サービス提供者による搾取のインセンティブが発生する一方で、各の診療行為に対して報酬が点数化されていることから、実際に制度の悪用に走るサービス提供者は観察されなかった、と考えられる。生活保護受給者の給付金の使途を観察するために多大なモニタリングコストが発生し、受給者による虚偽申告のインセンティブも大きいことから、政府としてはこうした財$\cdot$サービス提供者への直接給付制度を他の給付に対しても行うことが、社会的損失を抑制する方策となることを期待出来る。特に、住宅扶助のように、期間ごとの対価が基本的に一定で、サービス提供者による搾取を行うことが出来ない給付については大きな効果が期待され、実際に受給者を介さない直接的な支払い制度も導入されている。ただ、こうした給付制度は、受給者の経済活動を制限する側面があり、必要最低限の利用に留めるべきであろう。

また一方で、推定結果の頑健性についても、更に検証を行う必要がある。具体的には、今回モデルに組み込まなかった傷病分類との交差項を用いたDID分析や、具体的な診療行為の分類を用いた分析が有効であろう。また、受給者の医療サービスに対する選好を明らかにすることで、対価が実質無料であっても、その他の経済活動に傾倒することで、彼らが必要な医療行為を受けることを拒んでいる可能性を検討することも重要であろう。

\section{References}

 \begin{itemize}
 
 \item Lemieux \& Milligan(2008)''Incentive effects of social assistance: A regression
discontinuity approach''
 
 \item Bargain \& Doorley(2011) ``Caught in the trap? Welfare's disincentive and the labor supply of single men''
 
 \item 川口大司 ``日本の労働市場''
 
 \item 東京大学出版会 ``生活保護の経済分析''
 
 \end{itemize}

\end{document}