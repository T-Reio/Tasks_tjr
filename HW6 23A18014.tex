\documentclass[11pt]{jsarticle}
\usepackage{mathpazo}
\usepackage{amsmath,amssymb}
\usepackage[top=25truemm,bottom=30truemm,left=25truemm,right=25truemm]{geometry}
\begin{document}

\title{産業組織論 レポート課題6}
\author{経済学研究科 23A18014 丹治伶峰}
\date{}
\maketitle

\large
Schaumans and Verboven (2008) ``Entry and Regulation: evidence from health care professions,'' \textit{RAND Journal of Economics}

\normalsize

\section{目的}

ベルギーの総合医と薬局の参入モデルを用いて、2つの補完的なタイプのentrantが存在する市場で、一方のタイプに地理的な参入制約が設けられている場合に、それが社会的余剰の分配にどのような影響を与えているかを特定する。結果、多くの国で実施されている参入制約は、それが存在しない場合と比べて薬局の数を50\%以上も減らしていること、さらにその補完的な関係性から、総合医の数をも減少させていることが明らかになった。彼らのシミュレーションによると、規制の緩和は薬局の存在しない自治体の数を減らすことなく余剰の再分配を行うことが可能で、このことから、薬局の都市部への過剰供給を防ぐ目的で導入される参入障壁が、実際にはその意図通りに機能していない、と主張している。

\section{新規性}

Entry Gameと社会的余剰を扱った論文にはBerry and Waldfogel (1999)やHsieh and Moretti (2003)が存在する。彼らがFree Entry市場を扱っていたのに対して、この論文では、前述の通り参入制約が設けられているケースを扱う。

また、彼らの研究は垂直的な制約の効果を扱って研究とも関係する。Lafontaine and Slade (2008)は、政府による強制的な参入障壁が、法規制によらない自発的な規制と比べて社会的余剰を制限しやすい、という研究を行っている。

\section{産業の説明}

医療産業の労働供給には、全ての地域が充分な基礎医療サービスを受けることを目的に、多くの国で政府による参入障壁が設けられており、この論文で扱うベルギーも同様である。

医師になるためには、医学の学士を取得していることが求められる。かつては高卒の学位があれば誰でも資格試験を受けることが可能であったことや、'98年以降に医学部への入学者を制限する制度が導入されたことを考えると、これらが潜在的な参入障壁になると解釈することも出来るが、ここではこれらの規制を参入障壁とはみなさず、医師の資格を持つ者が自由に市場に参入することが可能である、と考える。なお、この論文で扱う医師とは内科医(特定の専門分野を持たず、総合的な診療を行う総合医)を指す。医師の資格を持つ者のうち、内科医となるのは全体の25\%程度である。また、内科医院のうち、大部分は一人の医師によって運営されている。

薬剤師も同様に学歴の取得条件を含む資格の保有が求められているが、重要なのは、薬局の設置が参入障壁によって制限されている点である。一つの自治体に参入できる薬局の上限数はその自治体の人口によって決まり、その制限がbindしている場合、薬剤師は既存の薬局を買収することでしか参入することができない。薬局一軒あたりの薬剤師の数は1.5人で、医師と同様に個人での経営が多い。

医師や薬局は広告などによる販売促進を行うことが禁じられているため、消費者による医療機関の選択は、概してその地理的要因に基づいて行われる。実際、居住している地域を出て他の自治体の医療機関を受診する消費者はほとんど存在せず、ベルギー人の多くはかかりつけの医療機関しか利用しないということが報告されている。

医師間の競争には所要時間や診療の質など、利便性に関する要素が大きな影響をもたらす。これは診療報酬に対する保険適用が一定の額までしか行われないことや、医師同士による広告活動の自主規制によって、金銭的、および広告による競争が大きく制限されていることによる。

薬局同士の競争にも同様の規制が存在する。薬品の価格は公的機関(Ministry of Economic Affairs)によって、薬局が受け取る利益率が一定になるように決定されるため、価格差別は一切行うことができない。また、広告による活動も医師の場合と同様、自主規制がかけられているため、彼らが用いることができる差別化戦略は、サービスの質の部分に依存することになる。

この論文で重視しているのが、二つの職業間に存在する補完的な関係である。医師の処方箋は薬局で受け付けられるため、両者が地理的に近い距離に位置していることは、互いの利潤に対して正の効果を持つと考えられる。一方で、薬局は医師の処方箋がなければ販売できない薬品のみを取り扱うわけではないこと、また医師の側も、薬を投与する必要がなかったり、精密検査を行うために大病院へ患者を紹介する場合などで薬を処方しないことがあることから、両者の補完関係は完全なものではないことに留意する。論文中では、医師が製薬会社の試供品を患者に処方する場合や、逆に薬剤師の行う医療行為の幅が拡大していることを挙げて、両者の関係がむしろ代替的になるような要素も存在することを指摘している。

\section{データ}

2001年の市場を用いたクロスセクションデータを利用する。データはベルギーの自治体レベルデータで、the National Institute of Health Insuranceから市場ごとの診療所、および薬局の数のデータを取得。これに3つの統計から得た地理的条件に関するデータをm巣びつけて分析を行う。サンプル数は1136、市場の重なりによって起こるバイアスの問題に対処するため、ここから大都市のデータを除いた847のサンプルを分析に利用する。

都市ごとの診療所と薬局の数の相関係数は0.85で、両者の存在が互いに強い影響を与えていることが分かる。また、薬局の数が参入規制いっぱいになっている自治体は全体の82.3\%を占めており、この規制によって参入を諦めた潜在的なentrantの存在が示唆される。

\section{モデル}

この論文で扱う参入ゲームは

\begin{enumerate}
  \item 売り手のlocation によってpayoffが決まる

  \item 2つのタイプの職業が互いに補完的なサービスを提供する

  \item 一方のタイプの参入には法的な制約が課されている
\end{enumerate}

という3点によって特徴づけられる。


\subsection{Payoff}

参入ゲームのアウトカムは、それぞれの市場に存在する診療所、薬局の数として観察される。市場に参入したプレーヤーの数は、タイプごとに区別して扱われ、薬局の数は$N_1$、診療所の数は$N_2$で表現される。また、薬局の数には上限規制が存在するため、これを$\bar{n}_1$で表す。このとき、各タイプの利潤関数は

\[
\pi^*_i(N_1, N_2) = \pi_i(N_1, N_2) - \epsilon_i
\]

で表される。$\pi_i$はそれぞれのタイプのプレーヤーの存在数によって一意に決まる部分であり、$\epsilon_i$は分析者からは観察できない要因の影響を捕捉したものである。

プレーヤー間の利潤に対する影響は、自分と同じタイプのプレーヤーについて代替的、異なるタイプについて補完的、もしくは独立であると考える。ここから、$\pi_i$に関する二つの仮定

\begin{enumerate}
  \item strategic substitutes

  \begin{align*}
    \pi_1(N_1 + 1, N_2) &< \pi_1(N_1, N_2) \\
    \pi_2(N_1, N_2 + 1) &< \pi_2(N_1, N_2)
  \end{align*}

  \item strategic complements or independent

  \begin{align*}
    \pi_1(N_1, N_2) &\leq \pi_1(N_1, N_2 + 1) \\
    \pi_2(N_1, N_2) &\leq \pi_2(N_1 + 1, N_2) \\
    & \text{and} \\
    \pi_1(N_1 + 1, & N_2 + 1) < \pi_1(N_1, N_2) \\
    \pi_2(N_1 + 1, & N_2 + 1) < \pi_2(N_1, N_2)
  \end{align*}

\end{enumerate}

を置く。前者の仮定は、同じタイプのプレーヤーが新たに参入すると自分の利潤は減少することを、後者は異なるタイプのプレーヤーが参入した場合は、利潤が影響を受けない、もしくは逆に利潤が増加すること、ただし同時に同じタイプのプレーヤーの数が増加した際は、代替性による効果が上回ることを記述している。

また、このモデルは一方のタイプが与えるもう一方のタイプの利潤への影響が非対称であることを許容する。例えば薬局が医師の利潤に与える補完効果が、医師が薬局に与えるそれと比較して異なるようなケースを記述することができる。

\subsection{Equilibrium with nonbinding entry restrictions}

次に、ある市場で$(n_1, n_2)$のプレイヤー数が実現する条件を求める。はじめに、薬局のentry restrictionがbindしていないようなケース($n_1 < \bar{n}_1$)について考える。$(n_1, n_2)$がナッシュ均衡となるための必要条件は

\begin{align*}
  \pi_1(n_1 + 1, n_2)<&\epsilon_1 \leq \pi_1(n_1, n_2) \\
  \pi_2(n_1, n_2 + 1) < & \epsilon_2 \leq \pi_2 (n_1, n_2)
\end{align*}

を満たす$(\epsilon_1, \epsilon_2)$が実現したときである。

この条件が十分条件とならないのは、Assumptionの項で仮定した、二つのタイプの(厳密な)戦略的補完性により、ナッシュ均衡が複数存在し、アウトカムが不確定となる$(\epsilon_1, \epsilon_2)$の領域が存在するからである。両方のタイプにとって、もう一方のプレーヤーの参入者数が自らの利潤に影響しない場合は、不確定な領域は存在しない。この問題に対処するため、この論文では追加的な仮定として、それぞれのプレーヤーがそれまでのhistoryを記憶し、subgame perfect均衡の修正を行うとする。この仮定により、プレーヤーはそれぞれの$(\epsilon_1, \epsilon_2)$の実現値に応じて、完全均衡を一意に決定することができる、と考えることができる。具体的には、複数の$(n_1, n_2)$が均衡として考えられる場合、実現するのはその中でプレーヤーの数が最大になるものが唯一のsubgame perfect equilibriumになる。ここから、Type 1の参入数がbindしていない時、$(n_1, n_2)$が実現する確率は

\begin{align*}
  \text{Pr}(N_1 = n_1, N_2 = n_2) =& \int_{\pi_1(n_1 + 1, n_2)}^{\pi_1(n_1, n_2)} \int_{\pi_2(n_1, n_2 + 1)}^{\pi_2(n_1, n_2)} f(u_1, u_2) du_1 du_2 \\
  &  \int_{\pi_1(n_1 + 1, n_2 + 1)}^{\pi_1(n_1 + 1, n_2)} \int_{\pi_2(n_1 + 1, n_2 + 1)}^{\pi_2(n_1, n_2 + 1)} f(u_1, u_2) du_1 du_2
\end{align*}

で表現される。右辺の2項目は、それより大きなプレーヤー数がナッシュ均衡となる場合を除外するための項である。

\subsection{Equilibrium with binding geographic entry restrictions}

次に、entry restrictionがbindしているケースについて、同様に均衡が実現する確率を求める。参入の上限規制$\bar{n}_1$を含む均衡$(\bar{n}_1, n_2)$が実現する必要条件は

\begin{align*}
  & \epsilon_1 \leq \pi_1(\bar{n}_1, n_2) \\
  \pi_2(\bar{n}_1, n_2 + 1) < & \epsilon_2 \leq \pi_2 (n_1, n_2)
\end{align*}

である。Type 2(医師)については参入制約がbindしていない場合と同様であるが、Type 1については、仮に$\bar{n}_1$より大きなプレーヤー数が均衡になる場合でも、参入制約によってそのような均衡を実現することが不可能であるため、$\epsilon_1$$(\bar{n}_1, n_2)$が実現する確率は

\[
\text{Pr}(N_1 = \bar{n}_1, N = n_2) = \int_{- \infty}^{\pi(\bar{n}_1, n_2)} \int_{\pi_2(\bar{n}_1, n_2 + 1)}^{\pi_2(\bar{n}_1, n_2)} f(u_1, u_2) du_1 du_2
\]


\section{推定方法}

推定にあたり、まず観察されるプレーヤー数についてのlikelihood functionを求める。

\[
l = d \text{Pr}(N_1 = \bar{n}_1, N_2 = n_2) + (1-d) \text{Pr}(N_1 = \bar{n}_1, N_2 = n_2)
\]

$d$は参入制約がbindするかどうかを示すindicatorである。

二つのタイプの誤差項$\epsilon$の密度関数$f(.)$は二変量正規分布に従う。変数同士の相関は、二変量正規分布のパラメータ$\rho$によって定義される。

もう一方のタイプが自身の利潤に影響を与えない場合は、モデルは二値の離散選択を行う切断プロビットモデルとして考えることができる。一方、参入制約がbindしていない場合は、モデルは切断のないプロビットモデルとなる。そして、$\rho=0$の場合は、Bresnahan and Reiss (1991b)のモデル同様、正規分布に従う2つの独立な確率変数を用いた伝統的なプロビットモデルで記述されることになる。この論文では、それぞれのモデルと当てはまりを比較することで、2つのタイプの補完関係を仮定するモデルの尤度を検証する。

タイプ$i$の利潤関数は

\[
\pi_i^*(N_1, N_2) = \lambda_i \ln(S) + X \beta_i - \alpha_i^j + \gamma_i^k/ N_i - \epsilon_i
\]

で定義される。$S$はその市場の人口で代替されるマーケットサイズ、$X$は市場特性を表す。$\alpha$はentry fixed costで、市場に存在する自分と同じタイプ$i$のプレーヤーの数$j$によって決定される。$\gamma$もう一方のタイプのプレーヤーの存在から受ける影響で、同じく市場に存在するもう一方のタイプのプレーヤーの数$k$によって求められる。また、その影響の大きさは自分と同じタイプのプレーヤーの数に反比例すると仮定する。

離散選択モデルではpayoffの尺度を明らかにしない。このため、推定においては$\epsilon_i$の標準偏差$\sigma_i=1$に基準化することができる。また、同様の理由から、同タイプのプレーヤー数が1の時の$\alpha_i^1$、異なるタイプのプレーヤー数が0の時の$\gamma_i^0$はそれぞれ0とおくことができる。さらに、パラメータについての追加的仮定

\begin{align*}
  \alpha_i^{j+1} &> \alpha_i^j \\
  \gamma_i^{k+1} &> \gamma_i^k \\
  \alpha_i^{j+1} - \gamma_i^{k+1}/(N+1) &> \alpha_i^i - \gamma_i^k /N_i
\end{align*}

を置く。一つ目は同タイプのプレーヤーが参入した際に自分の利潤は減少すること、二つ目は異なるタイプのプレーヤーが参入した場合は利潤が増加すること、三つめは二つのタイプが同時に参入してきた場合は利潤が減少することを意味している。また、この論文では一つの市場に最大11の薬局が存在するデータを扱うが、4以上の薬局が存在する市場については、そのすべてで参入制約がbindしているため、ここではそれ以上の数の参入に関しては追加的な固定費用が存在しないことを仮定し、パラメータの数を制限する。パラメータ$\alpha_i$、$\gamma_i$の推定には"bottop-up approach"と呼ばれる、参入プレーヤー数の増加がそれぞれの固定費用に与える影響が充分小さくなった場合、それ以上のプレーヤー数については影響が存在しないとみなす方法でパラメータの数を決める。

\section{結果}

\subsection{推定結果}

戦略的補完関係と参入制約を仮定したGeneral Modelとともに、補完関係を仮定しない($\gamma_i = 0$)切断プロビットモデル、補完関係および参入制約の両方を仮定しない二変量プロビットモデルでの推定を行い、それぞれを比較する。

推定結果は仮定と整合的で、参入制約の存在と二つのタイプの補完関係の両方を仮定するモデルが支持された。参入制約を仮定するモデルとしないモデルに対するHausman test は統計的に有意に仮定するモデルの当てはまりを支持した。また、両方を仮定するモデルにのみ導入される$\gamma_i$はいずれも尤度比検定でその有意性を示した。

この参入ゲームにおいて最も重要な影響を持っていたのは、先行研究と同様、マーケットサイズであった。特に、高齢者人口は意思決定に強い正の影響を示した。影響の大きさは参入者のタイプによって異なり、薬局の方が診療所よりも強い影響を受ける、と推定された。失業率の影響もタイプによって異なり、薬局が診療所に比べて強い正の影響を受けることが分かった。また診療所については、Flandarsにおける参入者数が有意に低く、地域による影響も存在することが分かった。

同タイプからの影響$\alpha$はいずれも正の値を取った。また、その値は参入者数について単調増加であり、仮定と整合的な結果が得られた。$\gamma$についても同様な結果が得られ、異なるタイプのentrantが利潤に補完的な影響を与えることが示された。一方、$\gamma_i$の値はプレーヤーのタイプによって異なり、それぞれのタイプが互いに与える影響は非対称であることが分かった。具体的には、診療所が薬局の存在から受ける正の影響の方が、その逆に比べて大きかった。

さらに、筆者は、異なるタイプのプレーヤー数を固定した上で、それぞれのプレーヤー数が支持されるためのマーケットサイズの下限(critical market size)を求めた。結果は$n_i$の値について右上がりの曲線を描き、相手のタイプのプレーヤー数が増えるごとに右方向にシフトした。仮定から、いずれのタイプも価格による競争を行うことはできないので、この結果は戦略的補完関係の頑健性を強調する結果であると言える。また、このモデルではinitial competition levelを推定しないので、独占企業が利潤を独占する状態から企業数が増えるにつれて競争市場に移行していく、という可能性も考えられるが、少なくとも薬局については、多くの市場で参入制約がbindしているため、その可能性は低いとしている。

\subsection{シミュレーション}

シミュレーションの項では、医療サービスにおける法定利益率の引き下げ、および薬局の参入制約の緩和の効果を推定し、これらの政策が一部地域への過小供給を防ぐという目的に合致するものであるかを検討している。

\begin{itemize}
  \item 参入自由化

  現行の各市場における参入の上限を$\Phi = 1$とし、この$\Phi$の値を上げることで自由化を表現する。推定については、誤差項$\epsilon_1, \epsilon_2$を二変量正規分布に基づいて1000回抽出を行い、プレイヤー数の変化を考察する。使用するデータセットは市場レベルではなく、複数の市場が存在する自治体レベルのものであるため、新たな参入者がどの市場に参入するのかが特定できないが、この論文では最も利潤が大きな市場に参入する、と仮定する。

  \item 利益率

  元の法定利益率$\mu$、 $0 \leq \Delta \leq 1$なる$\Delta$を用いて、新たな利益率を$\mu \Delta$と表現する。ここでは、薬局の利潤を

  \[
  \Pi_1^* = S \mu R(N_1, N_2) \exp(-\epsilon_1) - F_1(N_1, N_2)
  \]

  とする。$F_1(.,.)$は参入のための固定費用、$\epsilon_1$は観察不可能な影響を表す。この時、薬局が参入するための必要十分条件は

  \[
  \pi_1^*(N_1, N_2) = \ln(S) + \ln(\mu) + \ln(R(N_1, N_2)/F_1(N_1, N_2))-\epsilon_1 \geq 0
  \]

  と表される。3項目は元のモデルで使用していた$X \bar{\beta}_1 - \bar{\alpha}_1^j + \bar{\gamma}_1$を代入することで、元のモデルと同じ推定式を得る。ここでは、マーケットサイズに対するリターンを1に固定することで、$\epsilon_1$の分散$\sigma_1$を推定することが可能になる。
\end{itemize}

シミュレーションは$\Phi=1, 2, \text{large: 実質的な制約撤廃}$ および$\Delta = 1, .75, .5$それぞれについて推定する形で行われた。$\Delta$については、これにマーケットサイズによって異なる制約がかけられるケース($\Delta = .4, .6$)を加える。

まず、利益率を固定し、参入制約だけを緩和するシミュレーションでは、参入制約がプレイヤーの参入を厳しく規制していることが分かった。$\Phi=2$の時に予想されるプレーヤーの総数は現状よりも54\%多く、参入制約を取り払うとその増分は173\%と推定された。このシミュレーションでは薬局だけでなく、診療所の参入数も増えることが明らかになり、両者の間の補完関係がここでも確認された。また、参入の自由化は各サービスを供給する設備が存在しない市場の数についても好影響を与えると推定された。

シミュレーションは利益率規制についても否定的な結果を導いた。利益率の引き下げは供給の減少とそれぞれのサービスが存在しない市場の増加を導くが、これは参入制約の緩和によって相殺することができるからである。例えば、$\Delta=.75, \Phi = 2$の時、サービスが存在しない市場数は元の242から207に減少した。これにより、供給の問題を深刻化させずに、余剰の配分を消費者に回すことが可能であることが示唆されている。

以上から、筆者は高い利益率と厳しい参入制約は有効な政策であるとはいえず、逆に両者の緩和を同時に行うべきであるとしている。

さらに追加的なシミュレーションとして、供給されるプレーヤー数をほぼ維持したまま行うことができる利率引き下げの最大値を求めている。たとえば$\Phi = 1.5$のとき、政府は供給のない市場を242から259に増加を伴うものの、利率を$\Delta=.533$まで引き下げることができる。

\section{コメント}

この論文では、構造推定から政策評価を行い、実際に設けられている競争規制がその目的を達成しているかどうかを検証している。こうした検証は国家間の制度・環境の違いの影響を強く受けるため、ベルギーの例にとどまらず、他の国・州との比較を行う後続研究が期待される。また、この論文では、複数の異なるタイプが補完的な関係にある場合の産業構造の推定を、いくつかの仮定を行うことで単純化している。これを応用し、タイプが3以上存在するような場合に求められる追加的な仮定を特定することも、重要な知見を導くと考えられる。

\end{document}
