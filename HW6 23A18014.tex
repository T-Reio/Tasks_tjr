\documentclass[11pt]{jsarticle}
\usepackage{mathpazo}
\usepackage{amsmath,amssymb}
\usepackage[top=25truemm,bottom=30truemm,left=25truemm,right=25truemm]{geometry}
\begin{document}

\title{産業組織論 レポート課題6}
\author{経済学研究科 23A18014 丹治伶峰}
\date{}
\maketitle

\large
Schaumans and Verboven (2008) ``Entry and Regulation: evidence from health care professions,'' \textit{RAND Journal of Economics}

\normalsize

\section{目的}

ベルギーの総合医と薬局の参入モデルを用いて、2つの補完的なタイプのentrantが存在する市場で、一方のタイプに地理的な参入制約が設けられている場合に、それが社会的余剰の分配にどのような影響を与えているかを特定する。結果、多くの国で実施されている参入制約は、それが存在しない場合と比べて薬局の数を50\%以上も減らしていること、さらにその補完的な関係性から、総合医の数をも減少させていることが明らかになった。彼らのシミュレーションによると、規制の緩和は薬局の存在しない自治体の数を減らすことなく余剰の再分配を行うことが可能で、このことから、薬局の都市部への過剰供給を防ぐ目的で導入される参入障壁が、実際にはその意図通りに機能していない、と主張している。

\section{新規性}

Entry Gameと社会的余剰を扱った論文にはBerry and Waldfogel (1999)やHsieh and Moretti (2003)が存在する。彼らがFree Entry市場を扱っていたのに対して、この論文では、前述の通り参入制約が設けられているケースを扱う。

また、彼らの研究は垂直的な制約の効果を扱って研究とも関係する。Lafontaine and Slade (2008)は、政府による強制的な参入障壁が、法規制によらない自発的な規制と比べて社会的余剰を制限しやすい、という研究を行っている。

\section{産業の説明}

医療産業の労働供給には、全ての地域が充分な基礎医療サービスを受けることを目的に、多くの国で政府による参入障壁が設けられており、この論文で扱うベルギーも同様である。

医師になるためには、医学の学士を取得していることが求められる。かつては高卒の学位があれば誰でも資格試験を受けることが可能であったことや、'98年以降に医学部への入学者を制限する制度が導入されたことを考えると、これらが潜在的な参入障壁になると解釈することも出来るが、ここではこれらの規制を参入障壁とはみなさず、医師の資格を持つ者が自由に市場に参入することが可能である、と考える。なお、この論文で扱う医師とは内科医(特定の専門分野を持たず、総合的な診療を行う総合医)を指す。医師の資格を持つ者のうち、内科医となるのは全体の25\%程度である。また、内科医院のうち、大部分は一人の医師によって運営されている。

薬剤師も同様に学歴の取得条件を含む資格の保有が求められているが、重要なのは、薬局の設置が参入障壁によって制限されている点である。一つの自治体に参入できる薬局の上限数はその自治体の人口によって決まり、その制限がbindしている場合、薬剤師は既存の薬局を買収することでしか参入することができない。薬局一軒あたりの薬剤師の数は1.5人で、医師と同様に個人での経営が多い。

医師や薬局は広告などによる販売促進を行うことが禁じられているため、消費者による医療機関の選択は、概してその地理的要因に基づいて行われる。実際、居住している地域を出て他の自治体の医療機関を受診する消費者はほとんど存在せず、ベルギー人の多くはかかりつけの医療機関しか利用しないということが報告されている。

医師間の競争には所要時間や診療の質など、利便性に関する要素が大きな影響をもたらす。これは診療報酬に対する保険適用が一定の額までしか行われないことや、医師同士による広告活動の自主規制によって、金銭的、および広告による競争が大きく制限されていることによる。

薬局同士の競争にも同様の規制が存在する。薬品の価格は公的機関(Ministry of Economic Affairs)によって、薬局が受け取る利益率が一定になるように決定されるため、価格差別は一切行うことができない。また、広告による活動も医師の場合と同様、自主規制がかけられているため、彼らが用いることができる差別化戦略は、サービスの質の部分に依存することになる。

この論文で重視しているのが、二つの職業間に存在する補完的な関係である。医師の処方箋は薬局で受け付けられるため、両者が地理的に近い距離に位置していることは、互いの利潤に対して正の効果を持つと考えられる。一方で、薬局は医師の処方箋がなければ販売できない薬品のみを取り扱うわけではないこと、また医師の側も、薬を投与する必要がなかったり、精密検査を行うために大病院へ患者を紹介する場合などで薬を処方しないことがあることから、両者の補完関係は完全なものではないことに留意する。論文中では、医師が製薬会社の試供品を患者に処方する場合や、逆に薬剤師の行う医療行為の幅が拡大していることを挙げて、両者の関係がむしろ代替的になるような要素も存在することを指摘している。

\section{データ}

2001年の市場を用いたクロスセクションデータを利用する。データはベルギーの自治体レベルデータで、the National Institute of Health Insuranceから市場ごとの診療所、および薬局の数のデータを取得。これに3つの統計から得た地理的条件に関するデータをm巣びつけて分析を行う。サンプル数は1136、市場の重なりによって起こるバイアスの問題に対処するため、ここから大都市のデータを除いた847のサンプルを分析に利用する。

都市ごとの診療所と薬局の数の相関係数は0.85で、両者の存在が互いに強い影響を与えていることが分かる。また、薬局の数が参入規制いっぱいになっている自治体は全体の82.3\%を占めており、この規制によって参入を諦めた潜在的なentrantの存在が示唆される。

\section{モデル}

この論文で扱う参入ゲームは

\begin{enumerate}
  \item 売り手のlocation によってpayoffが決まる

  \item 2つのタイプの職業が互いに補完的なサービスを提供する

  \item 一方のタイプの参入には法的な制約が課されている
\end{enumerate}

という3点によって特徴づけられる。

\begin{itemize}
  \item Payoff

  参入ゲームのアウトカムは、それぞれの市場に存在する診療所、薬局の数として観察される。市場に参入したプレーヤーの数は、タイプごとに区別して扱われ、薬局の数は$N_1$、診療所の数は$N_2$で表現される。また、薬局の数には上限規制が存在するため、これを$\bar{n}_1$で表す。このとき、各タイプの利潤関数は

  \[
  \pi^*_i(N_1, N_2) = \pi_i(N_1, N_2) - \epsilon_i
  \]

  で表される。$\pi_i$はそれぞれのタイプのプレーヤーの存在数によって一意に決まる部分であり、$\epsilon_i$は分析者からは観察できない要因の影響を捕捉したものである。

  プレーヤー間の利潤に対する影響は、自分と同じタイプのプレーヤーについて代替的、異なるタイプについて補完的、もしくは独立であると考える。ここから、$\pi_i$に関する二つの仮定

  \begin{enumerate}
    \item strategic substitutes

    \begin{align*}
      \pi_1(N_1 + 1, N_2) &< \pi_1(N_1, N_2) \\
      \pi_2(N_1, N_2 + 1) &< \pi_2(N_1, N_2)
    \end{align*}

    \item strategic complements or independent

    \begin{align*}
      \pi_1(N_1, N_2) &\leq \pi_1(N_1, N_2 + 1) \\
      \pi_2(N_1, N_2) &\leq \pi_2(N_1 + 1, N_2) \\
      & \text{and} \\
      \pi_1(N_1 + 1, & N_2 + 1) < \pi_1(N_1, N_2) \\
      \pi_2(N_1 + 1, & N_2 + 1) < \pi_2(N_1, N_2)
    \end{align*}

  \end{enumerate}

  を置く。前者の仮定は、同じタイプのプレーヤーが新たに参入すると自分の利潤は減少することを、後者は異なるタイプのプレーヤーが参入した場合は、利潤が影響を受けない、もしくは逆に利潤が増加すること、ただし同時に同じタイプのプレーヤーの数が増加した際は、代替性による効果が上回ることを記述している。

  また、このモデルは一方のタイプが与えるもう一方のタイプの利潤への影響が非対称であることを許容する。例えば薬局が医師の利潤に与える補完効果が、医師が薬局に与えるそれと比較して異なるようなケースを記述することができる。

  \item Equilibrium with nonbinding entry restrictions

  次に、ある市場で$(n_1, n_2)$のプレイヤー数が実現する条件を求める。はじめに、薬局のentry restrictionがbindしていないようなケース($n_1 < \bar{n}_1$)について考える。$(n_1, n_2)$がナッシュ均衡となるための必要条件は

  \begin{align*}
    \pi_1(n_1 + 1, n_2)<&\epsilon_1 \leq \pi_1(n_1, n_2) \\
    \pi_2(n_1, n_2 + 1) < & \epsilon_2 \leq \pi_2 (n_1, n_2)
  \end{align*}

  を満たす$(\epsilon_1, \epsilon_2)$が実現したときである。

  この条件が十分条件とならないのは、Assumptionの項で仮定した、二つのタイプの(厳密な)戦略的補完性により、ナッシュ均衡が複数存在し、アウトカムが不確定となる$(\epsilon_1, \epsilon_2)$の領域が存在するからである。両方のタイプにとって、もう一方のプレーヤーの参入者数が自らの利潤に影響しない場合は、不確定な領域は存在しない。この問題に対処するため、この論文では追加的な仮定として、それぞれのプレーヤーがそれまでのhistoryを記憶し、subgame perfect均衡の修正を行うとする。この仮定により、プレーヤーはそれぞれの$(\epsilon_1, \epsilon_2)$の実現値に応じて、完全均衡を一意に決定することができる、と考えることができる。具体的には、複数の$(n_1, n_2)$が均衡として考えられる場合、実現するのはその中でプレーヤーの数が最大になるものが唯一のsubgame perfect equilibriumになる。ここから、Type 1の参入数がbindしていない時、$(n_1, n_2)$が実現する確率は

  \begin{align*}
    \text{Pr}(N_1 = n_1, N_2 = n_2) =& \int_{\pi_1(n_1 + 1, n_2)}^{\pi_1(n_1, n_2)} \int_{\pi_2(n_1, n_2 + 1)}^{\pi_2(n_1, n_2)} f(u_1, u_2) du_1 du_2 \\
    &  \int_{\pi_1(n_1 + 1, n_2 + 1)}^{\pi_1(n_1 + 1, n_2)} \int_{\pi_2(n_1 + 1, n_2 + 1)}^{\pi_2(n_1, n_2 + 1)} f(u_1, u_2) du_1 du_2
  \end{align*}

で表現される。右辺の2項目は、それより大きなプレーヤー数がナッシュ均衡となる場合を除外するための項である。

\item Equilibrium with binding geographic entry restrictions

次に、entry restrictionがbindしているケースについて、同様に均衡が実現する確率を求める。参入の上限規制$\bar{n}_1$を含む均衡$(\bar{n}_1, n_2)$が実現する必要条件は

\begin{align*}
  & \epsilon_1 \leq \pi_1(\bar{n}_1, n_2) \\
  \pi_2(\bar{n}_1, n_2 + 1) < & \epsilon_2 \leq \pi_2 (n_1, n_2)
\end{align*}

である。Type 2(医師)については参入制約がbindしていない場合と同様であるが、Type 1については、仮に$\bar{n}_1$より大きなプレーヤー数が均衡になる場合でも、参入制約によってそのような均衡を実現することが不可能であるため、$\epsilon_1$の下限に関する制約はもはや必要でなくなる。ここから、$(\bar{n}_1, n_2)$が実現する確率は

\[
\text{Pr}(N_1 = \bar{n}_1, N = n_2) = \int_{- \infty}^{\pi(\bar{n}_1, n_2)} \int_{\pi_2(\bar{n}_1, n_2 + 1)}^{\pi_2(\bar{n}_1, n_2)} f(u_1, u_2) du_1 du_2
\]

\end{itemize}

\section{推定方法}



\section{結果}

\section{コメント}



\end{document}
