\documentclass{jsarticle}[12pt]
\usepackage{mathpazo}
\usepackage{amsmath,amssymb}
\usepackage{array}
\begin{document}

\title{Microeconomic Analysis \\
Term Paper}
\author{経済学研究科  \\ 23A18014 Reio TANJI 丹治伶峰}
\date{}
\maketitle

\large

I review the following paper, as the term-paper in the microeconomic analysis :

\vspace{1zw}

\textbf{`Charactaristics of patent litigation : a window on competition'}

Lanjouw and Schankerman (2001) \textit{RAND Journal of Economics}

 \section{Summary}
 
 This paper summarized the pattern of individuals'/firms' litigation strategies about intellectual properties, depending on the charactaristics of the patents, patent owners, or the field of industries involved. Using cases in the U.S., they showed some hints to consider the degree of explosure to litigation risk.
 
  \subsection{Framework}
  
  
  
  \subsection{Data}
  
  They constructed patent-case-level data in the U.S. about whether the patent had been taken into the court or not, how often cited, the charactarisrtics of their patentee and so on. The dataset is obtained from the U.S. Patent and Trademark Office (PTO), covering the period 1975-1991 and including 5452 (3887 is that in the U.S.) patent cases. For matching estimation, they created a control group from the population of all U.S. patents.
  
  One possible issue is selection bias, caused by lack of report from U.S. federal courts to PTO. Especially in 1977-1979, only 22\% of patent disputes were recorded. They had dealt with this problem by checking differences between the reported groups and unreported ones.
  
  \subsection{Result : Description}
  
  
  
  \subsection{Result : Econometric Analysis}
  
  
  
  \subsection{Conclusion}
  
  
 
 \section{Comment}
 
 
 
  \subsection{Contribution}
  
  
  
  \subsection{Extention}
  
  

\end{document}