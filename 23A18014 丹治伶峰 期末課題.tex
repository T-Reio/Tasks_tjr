\documentclass{jsarticle}[12pt]
\usepackage{mathpazo}
\usepackage{amsmath,amssymb}
\usepackage{array}
\begin{document}

\title{Microeconomic Analysis \\
Term Paper}
\author{経済学研究科  \\ 23A18014 Reio TANJI 丹治伶峰}
\date{}
\maketitle

\large

I review the following paper, as the term-paper in the microeconomic analysis :

\vspace{1zw}

\textbf{`Charactaristics of patent litigation : a window on competition'}

Lanjouw and Schankerman (2001) \textit{RAND Journal of Economics}

 \section{Summary}
 
 This paper summarized the pattern of individuals'/firms' litigation strategies about intellectual properties, depending on the charactaristics of the patents, patent owners, or the field of industries involved. Using cases in the U.S., they showed that patent litigation is highly correlated with its variety of innovations.
 
  \subsection{Framework}
  
  They exploited following Cooter and Rubinfield (1989)'s assumption of the key determinants of litigation, specifying the their context :
  
   \begin{enumerate}
   
   \item Potentially litigious situation
   
   : Difficulty in detecting an infringement of patent rights
   
   \item Asymmetry of information and expectation about the outcome of a trial
   
   : In immature fields, also legal procedure itself is changing.
   
   \item Size of stakes
   
   : Value of the patent right and its indirect returns
   
   \item Cost of trial
   
   : Relative to cost of settlement, which can occur between domestic/foreign or corporate/individual patentees.
   
   \end{enumerate}
  
  In addition, they devided litigation into two types : infringement and challenge suits, in which plaintiff claim the patent is invalid. The latter is different from the former in its possible positive externality, that enables all firms to utilize the original innovation freely and raise the incentive of conducting R\&D.
  
  \subsection{Data}
  
  They constructed patent-case-level data in the U.S. about whether the patent had been taken into the court or not, how often cited, whether the patent is owned by an indibidual, or a corporate, its nationality,  and technology group (Drugs and health, Chemical, Electronic, Mechanical and Others) . The dataset is obtained from the U.S. Patent and Trademark Office (PTO), covering the period 1975-1991 and including 5,452 (3,887 is that in the U.S.) patent cases. For matching estimation, they created a control group from the population of all U.S. patents. 
  
  One possible issue is selection bias, caused by lack of report from U.S. federal courts to PTO. Especially in 1977-1979, only 22\% of patent disputes were recorded. They had dealt with this problem by checking differences between the reported groups and unreported ones.
  
  Each patnet contains the elements below :
  
   \begin{itemize}
   
   \item Number of claims
   
   : A patent consists of a set of claims which provide the boundaries of each intellecual property right, what is new in the claim. The patent examiner try to narrow its range before granting.
   
   \item IPC assignments
   
   : Technology-based classfication system of the patent.
   
   \item Citations
   
   : Classfied into two types : backward (prior related patents cited for the corresponding one) and forward (patents that cites the corresponding one).
   
   \item Ownership
   
   : Nationality (domestic Japanese, and other foreign) and type of ownership (individual and corporate).
   
   \item Case type (for patents into litigation)
   
   : Infringement suits versus challenge ones, predicted by whether the patent owner is plaintiff or the defendant in the court.
   
   \end{itemize}
  
  \subsection{Result : Description}
  
  
  
  \subsection{Result : Econometric Analysis}
  
  
  
  \subsection{Conclusion}
  
  They emphasized the existance of a relationship that links the variation in the charastaristics of the patents and its litigation risk. As the next step, they suggested assessing risks and costs to examine owners' actual behavior decision-making : proceeding to trial and the range of trials. Also, they refferred to construction of a large dataset. Finally, they mentioned matching the data with licensing contracts and property insurance.
 
 \section{Comment}
 
 
 
  \subsection{Contribution}
  
  First, this paper is new in aggregating the data from federal court and the characters about the patent and enables us to conduct statistical investigation. 
  
  Moreover, specifying the determinants of the risk of the each patent being litigated enables to serve more accurate insurance plan on intellectual properties. As is mentioned in introduction of the paper, suppliers are forced to limit their plan to pooling price when these key charactaristics are unobservable.
  
  \subsection{Extention}
  
  There should be needed some following research, revising dataset used in this paper, as they mentioned. Even though the selection bias check was conducted, they did not made any time-series research for testing the robustness. As is mentioned in 1.2, as much as 78\% of the cases were dropped for 1977-1979, while 85\% of the all cases was collected for 1985-1987. Now we can obtain the latest information from PTO, so following research to the paper can give some contribution.
  
  As is described in the article, patent is one of the important factor for strategic decision-making of the firms. Nowadays, the influence of the intellectual property on the competition or entry/exit is getting larger, because of the increase of the ``virtual'' goods circulated in the market. Some time-series analysis, therefore, also contributes to specify this importance. Structual estimation that exmaminates the impact of patent litigation to the competition or entry/exit strategies may also be of some contribution.

\end{document}