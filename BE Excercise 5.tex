\documentclass{jsarticle}
\usepackage{mathpazo}
\usepackage{amsmath,amssymb}
\usepackage{array}
\begin{document}

\title{Behavioral Economics \\
Exercise 5 Behavioral Game Theory}
\author{経済学研究科  \\ 23A18014 Reio TANJI 丹治伶峰}
\date{}
\maketitle

\begin{enumerate}

\item [Question 1]

 \begin{enumerate}
 
 \item
 
 Every type of seller chooses to disclose her own type $i = {L, M, H}$, and $p = v_i$.
 
 If the seller disclosed her type, the buyer purchases iff $v_i \geq p$, otherwise purchase iff $\epsilon \geq p$.
 
 \vspace{0.7zw}
 
 \hspace{2zw}When the type is concealed, rational buyer evaluates lemon by its expected value : 
 
 \[ E(v) = v_L \cdot q_L + v_M \cdot q_M + (1 - q_L - q_M) > v_H = 1 \]
 
 and buy only if $E(v) \geq p$.
 
 Then, type-H seller raise her profit by disclosing and set $p = 1$, and the buyer revises her belief
 
 \[ E(v) = v_L \cdot q_L + v_M \cdot q_M > v_M \]
 
 Again, there is an incentive for type-M seller to disclose and set $p = v_M$, and finally, the buyer predicts the expected value $E(v) = v_L = \epsilon$, which makes type-L seller to disclose and set $p = \epsilon$.
 
 \vspace{1zw}
 
  \item \hspace{2zw} When the buyer  is fully cursed, she predicts the predicts the expected value if the private information was hidden is :
  
  \[E(v) = v_L \cdot q_L + v_M \cdot q_M + (1 - q_L - q_M) \equiv E \]
 
 and purchase iff $E(v) > p$.
 
 Note that when the type was disclosed, the buyer purchase iff $v_i > p$.
 
 \hspace{2zw} In Stage 1, then, 
 
  \begin{itemize}
  
  \item Type-H seller : $v_H =1 > E $
  
  She discloses her type and set $p = 1$.
  
  \item Type-L seller : $v_L = \epsilon < E$
  
  She conceal her private information and set $p = E$.
  
  \item Type-M seller : $v_M \in (\epsilon, 1)$
  
  Her strategy is conditional on the value of $v_M$.
  
  If $v_M \geq E$, then she disclose her type and set $p = v_M$.
  
  Otherwise, $v_M < E$, she conceal and set $p = E$.
  
  \end{itemize}
 
 \newpage 

 \item 
 
 
 \end{enumerate}

\end{enumerate}

\end{document}