\documentclass{jsarticle}
\usepackage{mathpazo}
\usepackage{amsmath,amssymb}
\usepackage{array}
\begin{document}

\title{Behavioral Economics \\
Exercise 5 Behavioral Game Theory}
\author{経済学研究科  \\ 23A18014 Reio TANJI 丹治伶峰}
\date{}
\maketitle

\begin{enumerate}

\item [Question 1]

 \begin{enumerate}
 
 \item
 
 Every type of seller chooses to disclose her own type $i = {L, M, H}$, and $p = v_i$.
 
 If the seller disclosed her type, the buyer purchases iff $v_i \geq p$, otherwise purchase iff $\epsilon \geq p$.
 
 \vspace{0.7zw}
 
 \hspace{2zw}When the type is concealed, rational buyer evaluates lemon by its expected value : 
 
 \[ E(v) = v_L \cdot \Hat{q}_L + v_M \cdot \Hat{q}_M + (1 - \Hat{q}_L - \Hat{q}_M) > v_H = 1 \]
 
 where $\Hat{q}_i$ denotes her belief about each $q_i$, and buy only if $E(v) \geq p$.
 
 Then, type-H seller raise her profit by disclosing and set $p = 1$, and the buyer revises her belief : If the seller did not disclose her type, then $\Hat{q}_H = 0$, and so
 
 \[ E(v) = v_L \cdot \Hat{q}_L + v_M \cdot \Hat{q}_M > v_M \]
 
 Again, there is an incentive for type-M seller to disclose and set $p = v_M$, and finally, the buyer predicts $\Hat{q}_L = 1$, and the expected value $E(v) = v_L = \epsilon$, which makes type-L seller to disclose and set $p = \epsilon$.
 
 \vspace{1zw}
 
  \item \hspace{2zw} When the buyer  is fully cursed, she predicts the expected value if the private information was hidden is : $\Hat{q}_i = q_i$ for each $i \in {L,M,H}$
  
  \[E(v) = v_L \cdot q_L + v_M \cdot q_M + (1 - q_L - q_M) \equiv E \]
 
 and purchase iff $E(v) > p$.
 
 Note that when the type was disclosed, the buyer purchase iff $v_i > p$.
 
 \hspace{2zw} In Stage 1, then, 
 
  \begin{itemize}
  
  \item Type-H seller : $v_H =1 > E $
  
  She discloses her type and set $p = 1$.
  
  \item Type-L seller : $v_L = \epsilon < E$
  
  She conceal her private information and set $p = E$.
  
  \item Type-M seller : $v_M \in (\epsilon, 1)$
  
  Her strategy is conditional on the value of $v_M$.
  
  If $v_M \geq E$, then she disclose her type and set $p = v_M$.
  
  Otherwise, $v_M < E$, she conceal and set $p = E$.
  
  \end{itemize}
 
\vspace{1zw}

 \item By the assumption, in this question, $\chi$-cursed buyer's belief is:
 
 \[ \begin{cases}
 \Hat{q}_L = 1 & \text{ if ``rational''} \\
 \Hat{q}_i = q_i & \text{ if ``cursed''}
 \end{cases} \]
   
  Note that if $p > v_L = \epsilon$, then ``rational'' buyer revises her belief same as ``cursed.''
  
  Suppose the seller is type-L. Then, there are some possible strategies as follows :
  
   \begin{itemize}
   
   \item disclose, and set $p = \epsilon$
   
   The seller's expected payoff is:
   
   \[ (1 - \chi ) \epsilon + \chi \epsilon = \epsilon \]
   
   \item conceal, and $p = \epsilon$
   
   \[ (1 - \chi) \epsilon + \chi \epsilon = \epsilon \]
   
   \item conceal, and $p = E = v_L \cdot q_L + v_M \cdot q_M + (1 - q_L - q_M)$
   
   The ``rational'' buyer behave as if she is ``cursed,'' since $p=E$ is off-path. Then ,the seller's expected payoff is :
   
   \[(1- \chi ) E + \chi E = E \]
   
   which yields the best response for the seller. This strategy, however, cannot be a perfect Bayesian Nash equilibrium, since the buyer's bilief is not consistent.
   
   Thus, there is no equilibrium strategies.
   
   \end{itemize}
 
 \item 
 
 \end{enumerate}

\end{enumerate}

\end{document}