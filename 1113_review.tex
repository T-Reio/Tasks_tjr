\documentclass[dvipdfmx,12pt]{beamer}

\usepackage{bxdpx-beamer}
\usepackage{pxjahyper}
\usepackage{minijs}
\usetheme{Annarbor}
\usepackage{mathpazo}
\usepackage{amsmath,amssymb}
\usepackage{graphicx}
\usepackage{array}
\usepackage{tikz}

\title{Quantifying Loss-Averse Tax Manipulation}
\subtitle{Alex Ress-Jones}
\author{Reviewd by Reio TANJI}
\date{Nov 13th, 2018}
\institute{Osaka University}

\begin{document}
\begin{frame}\frametitle{}
\titlepage
\end{frame}

\small

\section{Introduction}
\begin{frame}\frametitle{Abstract}
  Alex Rees-Jones (2018)
  ``Quantifying Loss-Averse Tax Manipulation''
  \textit{Review of Economic Studies (2018) 85, 1251–1278}

  \begin{itemize}
    \item Presents the effects of \textit{loss-aversion} from the evidence of
    US taxpayers.

    \item Taxpayers are engaged to persue tax reduction activity especially
    when they have some positive due near the date of payment.

    \item Distribution of reported tax bill has excess mass around the border
    whether they must pay or not.
  \end{itemize}
\end{frame}

\begin{frame}\frametitle{Institutional Background}
  \begin{itemize}
    \item In the US, one's tax payment in each year is determined by the
    Internal Revenue Service (IRS), based on the difference between the
    reported taxable income and the her/his payment in advance:
    ``balance due.''

    \item If the balance due (denoted by $b$) is positive, the tax filer must
    that amount to the IRS, and if negative, then s/he can receive a refund.

    \item Balance due can be ``manipulated,'' by reporting donation they did,
    or enrollment in charitable contribution.

    $\Rightarrow$ Loss-Averse affects the tax filers' behavior according to
    their initial balance due, resulting in the bunching of the reported
    (observed) payment.

  \end{itemize}
\end{frame}

\begin{frame}
  %payment確定までの流れを図にしたいね
\end{frame}

\begin{frame}\frametitle{contribution}
  This paper contributes in three ways:

  \begin{enumerate}
    \item Illustrate robust and observable features of the presence of loss-
    aversion with minimal assumptions.

    \item Estimate the impact of loss-aversion measured in dollers.

    \item Specify the way to apply similar settings:

    loss-averse individual is able to manipulate an observable outcome.
  \end{enumerate}
\end{frame}

\section{Framework}
\begin{frame}\frametitle{Sequential Manipulation}
  \begin{itemize}

    \item Given $b_{\text{PM}}$: balance due prior to manipulation,
    taxpayers face a sequense of manipulation opportunities,
    each of which is charactarized by the parameters
    : $\{ m_i, c_i \} _{i=1} ^{J}$

    $m_i$ denotes the tax reduction by the $i$th manipulation

    $c_i$ is the intrinsic cost

    \begin{block}{Cost by manipulation}
      \footnotesize
      Taxpayers consider their benefits and costs to decide whether to
      make efforts to tax manipulation.

      \begin{itemize}
        \scriptsize
        \item Blumenthel and Slemrod (1992)

        It spend on average 27 hours documenting and reporting for tax
        reduction

        \item Benzarti (2015)

        They dislike tasks for tax 4.2 times as that for working with
        same time length
      \end{itemize}

    \end{block}
  \end{itemize}
\end{frame}

\begin{frame}
  \begin{itemize}
    \item Ordinary gain-loss function:
    \[
    \Phi(x|r) = \begin{cases}
    x-r & \text{ if } x \geq r \\
    \lambda (x-r) & \text{ if } x < r
  \end{cases}
  \]

    \item Applying this structure, loss-averse taxpayers' evaluattion
    of the benefit from each manipulation:

    \begin{align*}
      V(m_i | b, r) &= \Phi(-b + m_i | r) - \Phi(-b | r) \\
       &= \begin{cases}
       m_i & \text{ if } -b \geq r \\
       \lambda(r+b)+(m_i -b -r) & \text{ if } -b \in [r-m_i, r] \\
       \lambda m_i & \text{ if } -b \leq r - m_i
       \end{cases}
    \end{align*}
  \end{itemize}
\end{frame}

\begin{frame}\frametitle{Gain-Loss Function}
  \begin{tikzpicture}[domain = -3:3, samples = 200, >= stealth]
    \draw[->] (-3,0) -- (3,0) node[right]{$x$};
    \draw[->] (0,-3) -- (0,3) node[above]{$V(x)$};
    \draw plot[domain = 0:2.8] (\x, \x);
    \draw plot[domain = -1.5:0] (\x, {2 * \x});
    \draw (0,0) node [below right] {$r$};
  \end{tikzpicture}
\end{frame}

\end{document}
