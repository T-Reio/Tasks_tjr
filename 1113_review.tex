\documentclass[dvipdfmx,12pt]{beamer}

\usepackage{bxdpx-beamer}
\usepackage{pxjahyper}
\usepackage{minijs}
\usetheme{AnnArbor}
\usepackage{mathpazo}
\usepackage{amsmath,amssymb}
\usepackage{graphicx}
\usepackage{array}
\usepackage{tikz}
\usepackage{wrapfig}
\usepackage{float}
\usepackage{here}
\setbeamertemplate{navigation symbols}{}

\title{Quantifying Loss-Averse Tax Manipulation}
\subtitle{Alex Ress-Jones}
\author{Reviewd by Reio TANJI}
\date{Nov 13th, 2018}
\institute{Osaka University}

\begin{document}
\begin{frame}\frametitle{}
\titlepage
\end{frame}

\small

\section{Introduction}
\begin{frame}\frametitle{Abstract}
  Alex Rees-Jones (2018)
  ``Quantifying Loss-Averse Tax Manipulation''
  \textit{Review of Economic Studies (2018) 85, 1251–1278}

  \begin{itemize}
    \item Presents the effects of \textit{loss-aversion} from the evidence of
    US taxpayers.

    \item Taxpayers are engaged to persue tax reduction activity especially
    when they have some positive due near the date of payment.

    \item Distribution of reported tax bill has excess mass around the border
    whether they must pay or not.
  \end{itemize}
\end{frame}

\begin{frame}\frametitle{Institutional Background}
  \begin{itemize}
    \item In the US, one's tax payment in each year is determined by the
    Internal Revenue Service (IRS), based on the difference between the
    reported taxable income and the her/his payment in advance:
    ``balance due.''

    \item If the balance due (denoted by $b$) is positive, the tax filer must
    that amount to the IRS, and if negative, then s/he can receive a refund.

    \item Balance due can be ``manipulated,'' by reporting donation they did,
    or enrollment in charitable contribution.

    $\Rightarrow$ Loss-Averse affects the tax filers' behavior according to
    their initial balance due, resulting in the bunching of the reported
    (observed) payment.

  \end{itemize}
\end{frame}

\begin{frame}\frametitle{contribution}
  This paper contributes in three ways:

  \begin{enumerate}
    \item Illustrate robust and observable features of the presence of loss-
    aversion with minimal assumptions.

    \item Estimate the impact of loss-aversion measured in dollers.

    \item Specify the way to apply similar settings:

    loss-averse individual is able to manipulate an observable outcome.
  \end{enumerate}
\end{frame}

\begin{frame}\frametitle{Procedure of the Manipulation}
  Every April, taxpayers go through the process below:
  \begin{enumerate}
    \item Report their taxable income, such as wages, salaries,
    tips, business income, investment income, and so on.

    \item Report ``adjustments,'' to claim for things such as donations or
    payments for alimony

    $\Rightarrow$ Adjusted Gross Income (AGI) is calculated: balance due
    before manipulation.

    \item Accept AGI or complete an additional form of reduction: Itemization

    --Report deductable activities such as charitable contributions, medical
    and dental expenses, home mortgage interest payments.

    \item Final balance due is confirmed:

    Claim credits for pursuing tax incentivised behaviour and report other
    taces paid, payments already made to IRS .
  \end{enumerate}
\end{frame}

\section{Framework}
\begin{frame}\frametitle{Sequential Manipulation}
  \begin{itemize}

    \item Given $b_{\text{PM}}$: balance due prior to manipulation,
    taxpayers face a sequense of manipulation opportunities,
    each of which is charactarized by the parameters
    : $\{ m_i, c_i \} _{i=1} ^{J}$

    $m_i$ denotes the tax reduction by the $i$th manipulation

    $c_i$ is the intrinsic cost

    \begin{block}{Cost by manipulation}
      \footnotesize
      Taxpayers consider their benefits and costs to decide whether to
      make efforts to tax manipulation.

      \begin{itemize}
        \scriptsize
        \item Blumenthel and Slemrod (1992)

        It spend on average 27 hours documenting and reporting for tax
        reduction

        \item Benzarti (2015)

        They dislike tasks for tax 4.2 times as that for working with
        same time length
      \end{itemize}

    \end{block}
  \end{itemize}
\end{frame}

\begin{frame}
  \begin{itemize}
    \item Ordinary gain-loss function:
    \[
    \Phi(x|r) = \begin{cases}
    x-r & \text{ if } x \geq r \\
    \lambda (x-r) & \text{ if } x < r
  \end{cases}
  \]

    \item Applying this structure, loss-averse taxpayers' evaluattion
    of the benefit from each manipulation:

    \begin{align*}
      V(m_i | b, r) &= \Phi(-b + m_i | r) - \Phi(-b | r) \\
       &= \begin{cases}
       m_i & \text{ if } -b \geq r \\
       \lambda(r+b)+(m_i -b -r) & \text{ if } -b \in [r-m_i, r] \\
       \lambda m_i & \text{ if } -b \leq r - m_i
       \end{cases}
    \end{align*}

    \item Taxpayers continue to manipulate iff $m_i < c_i$.
  \end{itemize}
\end{frame}

\begin{frame}\frametitle{Gain-Loss Function}

  \begin{tabular}{cr}
      \begin{minipage}[H]{0.5\textwidth}
        \begin{itemize}
          \item If there remains tax due after reduction, then
          all the value of the manipulation is evaluated as loss.

          \item When, on the other hand, manipulation cancels out the due
          before, the margin to be refunded is evaluated as gain.

          \item If s/he does not have to pay more, then the reduction by the
          manipulation is fully counted as gain.
        \end{itemize}
      \end{minipage} &
      \begin{minipage}[H]{0.5\textwidth}
        \begin{tikzpicture}[domain = -2:2, samples = 200, >= stealth]
          \draw[->] (-2,0) -- (2,0) node[right]{$b$};
          \draw[->] (0,-2) -- (0,2) node[above]{$V(b)$};
          \draw plot[domain = 0:1.7] (\x, {0.6 * \x});
          \draw plot[domain = -1.2:0] (\x, {1.5 * \x});
          \draw (0,0) node [below right] {$r$};
          \draw (0, -2.5) node [below] {gain-loss function};
        \end{tikzpicture}
      \end{minipage}

  \end{tabular}

\end{frame}
\begin{frame}\frametitle{Assumption}
  Assume tax filers consder the most efficient manipulation.

  \begin{itemize}
    \item For $i<j$, $m_i/c_i \geq m_j/c_j$.

    : They considers each oppoturnity of manipulation, in the most efficient
    order.

    \item $m_1/c_1 > 1$.

    : There exists at least one desirable manipulaton oppoturnity.

    \item As $n \to \infty$, $m_n/c_n \to 0$.

    : The number of desirable opportunity is finite.
  \end{itemize}

  Taxpayers continue manipulating as long as $V(m_i|b, r) \geq c_i$,
  and stop when $V(m_i|b, r) < c_i$.
\end{frame}

\begin{frame}\frametitle{Thresholds}
  They define two thresholds of $i \in J$ that stop the manipulation depending
  on the gain-loss situation.

  \begin{align*}
    L &= \max \left \{ i: \dfrac{m_i}{c_i} > 1 \right \} \\
    H &= \max \left \{ i: \dfrac{m_i}{c_i} > \dfrac{1}{\lambda} \right \}
  \end{align*}

  \begin{itemize}
    \item $L$ is the threshold for gain phase, while $H$ is the
     one for loss phase.

     \item $L \leq H$, where equality holds if there is no $i$ s.t.
     $m_i/c_i \in (1/\lambda , 1]$.
  \end{itemize}
\end{frame}
\begin{frame}\frametitle{Example}
  \begin{figure}
    \includegraphics[width = 10cm, height = 4cm]{fig_tab/ARJ_T1.png}
  \end{figure}
  \begin{center}
    $\lambda = 2$
  \end{center}

  When balance initial balance due $b_{PM} \leq 22$, s/he continues
  to manipulate until $i=2$, while one with $b_{PM} > 48$ goes till
  $i = 5$:
  $L = 2, H = 5$.
\end{frame}
\begin{frame}
  \begin{tabular}{ll}
    \begin{minipage}[H]{0.4\textwidth}
      \begin{itemize}
        \item Expected range of the balance due after manipulation is
        $(-8, 8)$, which generates excess mass or bunching.
      \end{itemize}
    \end{minipage} &
    \begin{minipage}[H]{0.5\textwidth}
      \begin{figure}
        \includegraphics[width = 6cm, height = 8cm]{fig_tab/ARJ_F1.png}
      \end{figure}
      \end{minipage}
  \end{tabular}
\end{frame}
\begin{frame}\frametitle{Total Amount of Manipulation}
  Total manipulation is expressed as a function of the taxpayer's
  pre-manipulation balance due $b_{PM}$:

  \[
  m^* (b_{PM} | r) = \begin{cases}
  \sum _{i=1}^L m_i & \text{ if }b_{PM} \leq T_1 \\
  \sum _{i=1}^{L+1} m_i & \text{ if }b_{PM} \in (T_1, T_2] \\
  \dots \\
  \sum _{i=1}^{L+J-1} m_i & \text{ if }b_{PM} \in (T_{J-1}, T_J] \\
  \sum _{i=1}^{H} m_i & \text{ if }b_{PM} > T_J \\
\end{cases}
  \]
   where $T_j$ denotes

   \[
   T_j = \max \left \{ b_{PM} : V \left( m_{L + j} | b_{PM}
   + \sum_{i=1}^{L+j-1} m_j, r \right) \leq c_{L+j} \right \}
   \]
\end{frame}
\begin{frame}\frametitle{Distribution after Manipulation}
  \begin{itemize}
    \item $f^{\text{PM}}(b)$ denotes the ditribution of the pre-manipulation
    balance due, while $f(b)$ is that of post-manipulation.

    \item All the taxpayers make manipulation of $i=1$ to $L$, so the
    distribution at least shift to the left uniformly, denoted with $g(b)$.

    \item Furthermore, those with positive balance due after the $L$th
    manipulation continue to reduce tax, until $H$ or when their balance due
    is in the stop range.
  \end{itemize}
\end{frame}
\begin{frame}
  \small
  \[
  f(b) = f^{\text{PM}}(b+m^*) = \begin{cases}
  g(b) &\text{ if } b \leq B_1 \\
  g(b) + E_1 (b) &\text{ if } b \in (r - B_1, r] \\
  g(b + \tilde{m}) + E_2(b) &\text{ if } b \in (r, r + B_2) \\
  g(b + \tilde{m}) &\text{ if } b \geq r + B_2
\end{cases}
  \]
  \normalsize
  where $\tilde{m} = \sum_{i = L+1}^{H}m_i$, and
  \scriptsize
  \begin{align*}
    E_1(b) = g(b + \tilde{m})& \times I
    \left( b + \sum_{i = 1}^{H}m_i > T_J \right )\\
    & + \sum_{j=1}^{J-1}g \left( b + \sum_{j=1}^{L+j}m_i \right)
    \times I \left( \left( b + \sum_{i = L+1}^{L+j}m_i \right)
    \in (T_j, T_{j+1}) \right) \\
    E_2(b) = g(b) \times I &
    \left( b + \sum_{i = 1}^{L}m_i \leq T_1 \right )\\
    & + \sum_{j=1}^{J-1}g \left( b + \sum_{j=1}^{L+j}m_i \right)
    \times I \left( \left( b + \sum_{i = L+1}^{L+j}m_i \right)
    \in (T_j, T_{j+1}) \right)
  \end{align*}
  \normalsize

  $E_1, E_2$ generates the excess mass.
\end{frame}
\section{Data and specification}
\begin{frame}\frametitle{Data}
  \begin{itemize}
    \item Satatistics of Income Panel of Individual Returns

    \begin{itemize}
      \item Random sample of tax filers, according to the Social Security
      Numbers

      \item contain many line items reported on the tax return allowing
      the direct observation of balance due and many steps of its calculation

      \item Data years:1979-1990

      \item 229,116 tax returns filed by 53,177 taxpayers:
      exculudeing those with zero-tax liability, in order to eliminate
      the excess mass owing to non-preference-based discontinuities.
    \end{itemize}
  \end{itemize}
\end{frame}
\begin{frame}\frametitle{Quantification}
  \begin{itemize}
    \item Fitting Distribution

    \[
    \min_{(\tilde{m}, B_1, B_2, \theta_g, \theta_e)}
    \sum_k \left( C_k - \hat{C}(k | \tilde{m}, -B_1, -B_2, \theta_e,
    \theta_g) \right)^2
    \]

    where

    \begin{align*}
      \hat{C}(k| \tilde{m}, -B_1, -B_2, \theta_e, \theta_g)
      = v_g \cdot g(k + \tilde{m} &\cdot I(k>0) | \theta_g) \\
      &+ v_E \cdot E(k | \theta_e, -B_1, B_2)
    \end{align*}

    \begin{align*}
      v_E &= \dfrac{N}{\sum_k E(k|\theta_e, -B_1, B_2)}
      \cdot \int_0^{\tilde{m}}g(x|\theta_g)dx \\
      v_g &= \dfrac{N}{\sum_k g(k+\tilde{m}\cdot I(k>0)|\theta_g)}
      \cdot \left( 1-\int_0^{\tilde{m}}g(x|\theta_g)dx \right)
    \end{align*}

  \end{itemize}
\end{frame}

\section{Results}
\begin{frame}\frametitle{Observed Distribution}

\begin{tabular}{lr}
  \begin{minipage}[H]{0.4\textwidth}
    \begin{itemize}
      \item Full-sample application
    \end{itemize}
    \begin{figure}[H]
      \includegraphics[keepaspectratio, scale = 0.4]{fig_tab/ARJ_F2}
    \end{figure}
  \end{minipage} &
  \begin{minipage}[H]{0.45\textwidth}
    \begin{itemize}

      \item Sharp spike ofserved near zero balance-due:

      excess mass range is -\$159.7(S.E.=27.00) to \$103.2(S.E.=36.20)

      \item Estimated value of $\tilde{m} = 33.8$:

      When facing losses, they made approximately \$34 additional tax
      reduction,as opposed to facing gains.
    \end{itemize}
  \end{minipage}
\end{tabular}

\end{frame}

\begin{frame}\frametitle{Difference among the Groups}
  \begin{figure}[H]
    \includegraphics[keepaspectratio, scale=0.78]{fig_tab/ARJ_T3.png}
  \end{figure}
\end{frame}

\begin{frame}
  \begin{figure}
    \includegraphics[keepaspectratio, scale = 0.6]{fig_tab/ARJ_F3.png}
  \end{figure}
\end{frame}

\begin{frame}
  \begin{itemize}
    \item By income levels
    \begin{itemize}
      \item Loss-aversion pattern was more pronounced among higher income
      filers than those with lower ones:

      higher income tax filers have more options of manipulation.

      \item This supports the hypothesis that the observe behaviour is
      caused by the income distribution, not by the financial sophistication.
    \end{itemize}

    \item By categories of reduction.

    \begin{itemize}
      \item When itemized reduction or adjustment to income are present, the tendency of loss aversion is more pronounced.

      \item Claiming credit, however, showed the opposite pattern:

      it may be relatively small component of overall tax manipulations,  or it is likely a less suitable proxy for total manipulation.
    \end{itemize}
  \end{itemize}
\end{frame}

\begin{frame}\frametitle{``Residual'' Tax Reductions Approach}

    \begin{tabular}{lr}
      \begin{minipage}[H]{0.4\textwidth}
        \begin{itemize}
          \footnotesize
          \item $R$ = (credits) + (marginal tax rate) $\times$
          (adjustments $+$ deductions).

          \item Then, they regressed $R$ on filing-year dummies and individual fixed effects, and estimate the remaining residual.

          $\Rightarrow$ Loss-averse tax reduction is strongly associated with
          ``unusual'' tax reduction activity, rather than the year/
          individual specific effect.
        \end{itemize}
      \end{minipage} &
      \begin{minipage}[H]{0.4\textwidth}

        \begin{figure}[H]
          \includegraphics[keepaspectratio, scale =0.5]{fig_tab/ARJ_F5.png}
        \end{figure}
      \end{minipage}
    \end{tabular}
    %The proxy is still imperfect because it misses evvaded income.
\end{frame}

\begin{frame}\frametitle{Earlier Tax Payments}
  \begin{itemize}
    \item Individuals evaluate their balance due by that on the tax day, so the payment in advance can be interpreted as another opportunity of manipulation.

    \item They made another regression, using total tax prepayment instead of $R$.

    $\Rightarrow$ No association was observed(Figure 5).
  \end{itemize}
\end{frame}

\section{Robustness}
\begin{frame}\frametitle{Alternative Forms of Reference Dependence}
  Two additional forms of reference dependence:

  \begin{enumerate}
    \item Notch

    -- direct discontinuity in utility levels at the reference point, arise if there is a fixed psychological cost when making manipulation.

    \item Diminishing Sensitivity

    -- utility function which is concave over gains and convex over losses, making individual more loss-averse.
  \end{enumerate}

  Observed results did not support these two assumptins.
\end{frame}

\begin{frame}\frametitle{Financial Constraints}
  \begin{itemize}
    \item Tax Evasion

    -- A risky substitute for a loan, implicitly trading present income for future penalties.

    \item Tax filers with financial constraint may take such a loan, but low-income filers have low access to savings or credit.

    \item Restricted sample: Tax filers with positive interest income, 41 \% of the whole sample

    $\Rightarrow$ The estimate of loss averse manipulation is \$30.4, indicating little influence on the patterns.
  \end{itemize}
\end{frame}

\begin{frame}\frametitle{Tax Preparer}
  \begin{itemize}
    \item Many taxpayers have paid tax preparer file their tax returns.

    $\Rightarrow$ Conditional sample on whether filed by tax preparer also shows loss-averse result ($\tilde{m} = 22.6$), but the effect was weakened relative to self-filing sample ($\tilde{m} = 43.7$).

    \item Possible Interpretation

    \begin{enumerate}
      \item Tax preparers themselves are also loss-averse.

      \item They believe the cliants are loss-averse and so incorporate them.
    \end{enumerate}

    \item Distribution of tax preparers-sample has more sharp spike around zero balance due: they have greater access to manipulation.
  \end{itemize}
\end{frame}

\begin{frame}
  \begin{figure}
    \includegraphics[keepaspectratio, scale = 0.7]{fig_tab/ARJ_F7.png}
  \end{figure}
\end{frame}

\begin{frame}\frametitle{Underwithholding Penalty}
  \begin{itemize}
    \item Underwithholding penarlty is not imposed until substantial one has occurred:

    Observed balance-due is interpreted as the manipulation is not observed at zero.

    \item Misunderstandings that any positive balance due leads to a penalty may lead the results
    \begin{enumerate}
      \item Manipulation still occurs for the tax preparers sample.

      \item Restricted samples into those who previously have faced a loss show the same results.
    \end{enumerate}
  \end{itemize}
\end{frame}

\section{Conclusion}
\begin{frame}\frametitle{Discussion}
  \begin{itemize}
    \item The nature of the observed reaction to loss framing has important implications for tax policy and behavioral economics.

    -- Facing losses pursues \$34 of additional manipulation, resulting in \$3.7 billion of additional reduction.

    \item Loss-aversion  provide a quantification of effect sizes tie to more concrete policies and accounts for a significant portion of the costs or benefits to changing withholding policy:

    Incentives for early payment % $80 loss for interest income

    \item Gain-loss framing can assist in controlling tax morale:

    Gain framing could reduce evasion motives in a more cost-effective manner.
  \end{itemize}
\end{frame}

\begin{frame}\frametitle{Beyond Tax Policy}
  \begin{itemize}
    \item Techniques to identify loss-averse behavior from bunching.

    \begin{block}{Related Papers}
      \begin{itemize}
        \item Pope and Simonsohn (2011) ``round numbers as goals''

        \item Abeler \textit{et al} (2011) ``effort provision in the lab''

        \item Becker \textit{et al} (2012) ``price targets in mergers and acquisitions''

        \item Allen \textit{et al} (2017) ``goal-setting behavior of marathon runners''
      \end{itemize}
    \end{block}
  \end{itemize}
\end{frame}
\end{document}
