\documentclass{jsarticle}
\usepackage{mathpazo}
\usepackage{amsmath,amssymb}
\begin{document}

\title{産業組織論 レポート課題5}
\author{経済学研究科 23A18014 丹治伶峰}
\date{}
\maketitle

Pavcnik(2002)

\begin{enumerate}

\item 目的

貿易自由化が企業の生産性に与える影響を、OP法による生産関数の推定によって明らかにする。

\item 新規性

従来の研究では、プラントの生産性を経時変化しない固定効果として取り込んだHarrison(1994)や、生産性を時間の二次式で表現したLiu(1993)、また、プラントの生産に関する意思決定を考慮しないものが多かった。そこでこの論文では、Olley and Pakes(1996)によって考案された生産関数の推定法(OP推定法)を実証研究に応用することで、従来の研究の課題となっていた、セレクションバイアスと同時性バイアスに対処する。具体的には、各生産プラントの生産効率が経時的に変化するという仮定をモデルに組み込み、更に観察される生産性が、貿易自由化によって海外の企業との競争を強いられた結果、市場から退出することを選択した、相対的に生産性の低いプラントを除いたものであることを考慮している。

\item データ、産業

1970年代から'80年代にかけてチリで行われた大規模な貿易自由化を利用する。チリでは1974年から'79年にかけて、非関税障壁を完全に撤廃し、併せて関税も大幅に引き下げる、貿易自由化政策が採られた。その後景気後退に伴う一時的な中断、'84年には関税の引き上げが行われたものの、長期的には継続して貿易の自由化が促進された。

データはチリの国立統計局から、1979年から'86年の期間中に、少なくとも10人以上の従業員を雇用している企業の情報を取得し、パネルデータを生成した。観察された企業の90\%以上は、単一のプラントによって創業している企業であった。

この論文では、プラントが生産を行う財の市場特性を考慮した分類を行う。具体的には海外市場との関わりに注目し、国内企業が輸入によって海外企業と国内で競争を行っている場合、国内企業が海外に輸出を行っている場合、海外市場との取引を行わない傾向にある場合の3つに分類する。自由化されていない状況下では海外企業の参入が難しく、本来取引のある市場が国際的な貿易を行っていないものとして分類してしまう可能性があるが、この論文ではこの内生性の問題は重要でないとしている。

プラントの退出の影響は、この市場特性によって違いがあり、輸入を行う産業で大きな変化を及ぼしていた。ここから、貿易自由化によるプラントの退出が市場において重要な役割を果たしている可能性が指摘される。

\item モデル、推定方法

始めに、モデルのセッティングを行う。ある同一の市場における(: 同じ価格)各企業は、初めにプラント間で異なる生産性と保有する資本量に基づいて、生産活動を行うか、市場から退出するかを選択する。生産を行うプラントは、続いて生産要素の投入量を決定する。

まず、生産関数の推定を行う。

一般に、貿易の自由化は企業の生産性を向上させると考えられている。海外からの生産技術の流入や、競争の激化による生産資源利用の効率化、エージェンシー問題の改善が起こることが理由とされるが、この生産性の向上を特定するにあたっては、自由化政策を施行したあとの生産に関するデータが、全体の生産性が向上し、均衡価格が下落してことで淘汰された、比較的生産性の低いプラントを含まないものとなっていることによっておこるセレクションバイアスの存在を考慮する必要がある。また、この論文で用いるモデルは、後述する通り前期の資本と生産性に依存したプラントの生産性の経時的な変化を認めているとめ、同時性バイアスの問題にも対処する必要がある。

生産関数は、以下の通り定義されるコブ$\cdot$ダグラス型関数を用いる。

 \begin{align*}
 y_{it} &= \beta_0 + \beta x_{it} + \beta_k k_{it} + e_{it} \\
 e_{it} &= \omega_{it} + \mu_{it}
 \end{align*}

$y_{it}$, $x_{it}$, $k_{it}$, $e_{it}$ は、それぞれ時間$t$におけるプラント$i$の生産量、中間投入財、資本、それ以外の観察できない特性を対数表示したものを表す。観察できない特性は、プラント自身のみが観察可能な生産性$\omega$、プラント自身にも予測できない生産へのショック$e_{it}$とに分解される。

生産を行うプラントの利潤最大化問題は

\[ \max \Pi_{ijt} = f(k_{ijt}, \omega_{ijt}) \]

チリの労働契約に関する法規制や、実際の企業の行動から、企業は人的資本、及び各生産要素の投入量を自由に調整可能であると仮定する。一方で、資本量は前期の資本量と投資によって決まり、調整することが出来ないと考える。上の利潤最大化の解によって実現される利潤が、そのプラントを売却した時の価値$L$を下回る場合、プラントは生産を行わずに市場から退出するから、動学生産計画は、Value function $V_t$を用いて

 \begin{align*}
 V_t (\omega_t, k_t) &= \max \{L_t, sup \Pi_t(\omega_t, k_t) -c(i) + 
 				      d E[V_{t+1} (\omega_{t+1}, k_{t+1})| \Omega_{it}] \} \\
  & k_{t+1} = (1- \delta)k_t + i_t
 \end{align*}

と定義される。$c(.)$は投資$i$を実現する為の費用、$d$は時間割引因子、$\Omega_{it}$は時点$t$に得られる将来に関する情報である。市場の条件はプラント間で一定であるとする。

時点$t$に市場に存在するプラントは、保有する資本量と、それによって定まる生産性を元に、生産を行うかどうかの意思決定、及び当期の投資を決定する。

\[X_{t} = \begin{cases}
1, & \text{ if } \omega_t \geq \underline{\omega_{t}} (k_t) \\
0, & \text{ otherwise}
\end{cases} \]

\[i_t = i_t(\omega_t, k_t) \]

$\underline{\omega_{t}}$は生産を行う$\omega$の下限である。このモデルでは、この下限の値と投資を決定する関数の時間を通じた変化を認める。

\vspace{1zw}

以上の設定の下で、生産性の推定を行う。推定は、OP法(semiparametric estimation)を用いる。プラント自身によってのみ観察可能な生産性は、投資の関数から

\[ \omega_t = i^{-1}_t (i_t, k_t) = \theta (i_t, k_t) \]

これを生産関数に代入して

 \begin{align*}
 y_{t} &= \beta x_t + \lambda_t(k_t, i_t) + \mu_t \\
 \text{where} \\
 \lambda_t (k_t, i_t) &= \beta_0 + \beta_k k_t + \theta_t (k_t, i_t) \\
 \end{align*}

このとき、生産量$y_t$は誤差項$\mu_t$と相関しないため、ここから$\beta$の一致推定量が得られる。

次に、プラントの生産性の改善による生産量の変化と、生産性に依存した投資に関する意思決定の影響を識別する。

各プラントの意思決定者は、当期の生産性を元に次期の生産性に対する予測を立てる。生産性は当期の資本と投資、次期の資本は当期の資本と投資に依存するから、次期の生産性の期待値は、当期の生産性に基づく予想を表す関数$g(.)$を用いて

\[ E[\omega_{t+1}|\omega_t, k_{t+1}] = g(\omega_t) - \beta_0 = g(\theta_t(i_t,k_t)) - \beta_0
= g(\lambda_t - \beta_k k_t) - \beta_0 \]

と表現される。これを生産関数に代入して

 \begin{align*}
 y_{t+1} - \beta x_{t+1} &= \beta_0 + \beta_k k_{t+1} + E[\omega_t | \omega_t, k_{t+1}] + \xi_{t+1} + \mu_{t+1} \\
 &= \beta_k k_{t+1} + g(\lambda_t - \beta_k k_t) + \xi_{t+1} + \mu_{t+1}
 \end{align*}
 
 観察される生産量は、保有する資本量に依存して定まる生産性の閾値$\underline{\omega}$を下回る企業が退出した上でのものであるから、これを処理する手順を加える。市場に留まることを選択したプラントの次期の生産性の期待値は
 
 \[E[\omega_{t+1} | \omega_t, k_{t+1}, X_{t+1}=1] 
 = E[\omega_{t+1}| \omega_t, \omega_{t+1} > \underline{\omega_{t+1}}(k_{t+1})]
 = \Phi (\omega_t, \underline{\omega_{t+1}}) - \beta_0 \]
 
 \[\Phi (\omega_t, \underline{\omega_{t+1}} = E[\omega_{t+1}| \omega_t, \omega_{t+1}] + \beta_0 \]
 
企業が市場に留まる、すなわち$X_{t+1} = 1$ となる確率は

 \begin{align*}
 Pr(X_{t+1}) &= Pr\{ \omega_{t+1} > \underline{\omega_{t+1}}(k_{t+1})| \omega_{t+1}, \omega_t \} \\
 &= p_t(\underline{\omega_{t+1}}(k_{t+1}), \omega_t) \\
 &= p_t(\underline{\omega_{t+1}}(k_t, i_t), \omega_t) \\
 &= p_t(k_t, i_t) \equiv P_t
 \end{align*}

となり、$k_t, i_t$の関数として記述できる。

また、$\underline{\omega_{t+1}}$は$\omega_t$と$P_t$によって記述できるから、$\Phi$は

\[ \Phi(\omega_t, \underline{\omega_{t+1}}) = \Phi(\omega_t, P_t) \]

と書き換えられる。これらを代入して、生産関数は

\[ y_{t+1} - \beta x_{t+1} = \beta_k k_{t+1} + \Phi( \lambda_t - \beta_t k_t, P_t) + \xi_{t+1} + \mu_{t+1} \]

推定には$\Phi(.)$に三階の多項式展開を行うnonlinear least squares techniqueを用いる。また、この論文ではOlley and Pakes(1996)で用いた収束計算ではなく、近似を利用した推定を行う。OP法では生存企業が正の投資を行うことを仮定しているが、扱うデータには投資を行っていないプラントも存在するため、これらを取り除いた場合との推定結果の整合性をチェックしている。

\item 推定結果
 
OP法による推定結果の頑健性を担保するため、すべてのプラントを含むデータセットと、当該期間中に市場から退出したプラントを除いたデータセットを用意し、それぞれに対してセミパラメトリック推定の他、OLS、固定効果モデルによる推定を行い、結果の比較を行う。OLSによる推定は、プラントの特性の効果を過大評価する傾向にあり、また上方、下方どちらかは理論上明らかでないが、資本についてもバイアスが生じることが知られている。推定結果はこの仮説と整合的で、分析対象となった8産業中5つの産業で、セミパラメトリック推定による係数の推定値に大きな開きが見られた。また、3つの産業では資本に対する係数の推定値が減少した。これは、セレクションバイアスの影響が同時性バイアスのそれに比べて小さかったことを示している。推定結果は規模の経済効果の存在も示したが、その影響は産業間で開きがあった。一方、固定効果モデルによる推定は、測定誤差による過大評価に対し、より慎重な推定結果を返すため、OLSやセミパラメトリック推定よりも推定値が小さくなる傾向にあるが、この分析結果でも同様の推定が得られた。また、サンプルを生存企業に絞ったセミパラメトリック推定も、一部整合的でない結果が出たものの、重要な変化ではなかったとしている。

次に、産業ごとの経時的な生産性の変化を推定する。各年度ごとの生産性を推定し、これを平均的な生産性の変化と、そのプラントが占めるシェアに基づく推定からの逸脱を示す部分の二つに分解し、企業の退出によって起こる生産要素の再分配による生存企業の生産性向上を特定する。推定結果は、当該期間中の全体的な生産性の成長を支持するもので、この再分配による効率性向上が実際に起こっていることを示した。また、生産性の向上は特に輸入による海外企業との競争がある産業で特に顕著で、貿易自由化が効率性の改善行動を促した、としている。

最後に、上記の結果の頑健性を担保するため、回帰分析による推定を行う。推定式は以下の通り、DID分析を行う。

\[ pr_{it} = \alpha_0 + \alpha_1 (\textit{Time})_{it} + \alpha_2(\textit{Trade})_{it}
+ \alpha_3 (\textit{Time})_{it} \times (\textit{Trade})_{it} + \alpha_4 Z_{it} + v_{it} \]

$\textit{Time}$は年度、$\textit{Trade}$はその産業の海外との貿易関係による分類を表すダミー変数である。交差項の係数を見ることで、景気後退など、産業特性に関わらず同一の影響を与える事象の影響と貿易自由化によるそれを区別する。さらに、退出した企業の生産性の特性を捕捉するため、期間中の退出を示すダミー変数と貿易関係を示すダミーとの交差項を導入する。推定結果から、輸入による海外の企業との競争を行う産業には、貿易の少ない産業と比較して、貿易自由化による生産性の向上が存在した一方、輸出によって海外での競争に直面する産業には有意な変化が見られなかった。これに対し筆者は、海外での競争を行う企業は、海外企業との生産競争に勝つために、輸入産業における国内企業に起こったような生産性の向上がそれ以前に起こっていた、という解釈を与えている。一方、退出した企業の生産性は、生存した企業に比べて生産性が有意に低いという結果が得られたが、産業の貿易特性による差異は見受けられなかった。

上記のDID分析では、為替レートや関税障壁の変化を考慮していないが、この変化の影響は産業の貿易特性によって異なるため、貿易自由化の効果と識別することが出来ない。この論文では最後にこの点を検討している。まず、生産量と生産性の相関をみると、両者の相関はさほど強くなく、為替レートの変動による需要変化が、必ずしも生産性とリンクしていないことが分かる。また、各プラントの在庫の流動的な変化が為替レートの変動と一致しないことからも、同様の解釈が得られる。為替レートと産業の貿易特性を用いたDID分析、為替レートと貿易特性、関税を用いた回帰分析でも、貿易特性による生産性の変化の影響は統計的に有意であり、筆者は経済指標や関税障壁の変動によらない、貿易自由化の効果が存在していることが示されたと結論付けている。

\item コメント

この論文でも取り上げているチリの例は、明示的な貿易自由化の取り組みが観察されることから、貿易自由化の効果を推定する他の多くの研究でも取り上げられたケースである。OP法による結果の頑健性を示したこの研究の後続研究として、その他の事例を扱ったものが期待される。チリの例では単一のプラントで操業する企業が90\%を超えており、より大規模な生産を行うことが出来る、複数プラントによる生産を行う企業に注目した研究からも、新たな知見を得られる可能性がある。

\end{enumerate}

\end{document}