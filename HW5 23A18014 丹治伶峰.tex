\documentclass{jsarticle}
\usepackage{mathpazo}
\usepackage{amsmath,amssymb}
\begin{document}

\title{産業組織論 レポート課題5}
\author{経済学研究科 23A18014 丹治伶峰}
\date{}
\maketitle

Pavcnik(2002)

\begin{enumerate}

\item 目的

貿易自由化が企業の生産性に与える影響を、OP法による生産関数の推定によって明らかにする。

\item 新規性

Olley and Pakes(1996)によって考案された生産関数の推定法(OP推定法)を実証研究に応用することで、従来の方法で取り除くことが難しかった、セレクションバイアスと同時性バイアスに対処する。具体的には、各生産プラントの生産効率が経時的に変化するという仮定をモデルに組み込み、更に観察される生産性が、貿易自由化によって海外の企業との競争を強いられた結果、市場から退出することを選択した、相対的に生産性の低いプラントを除いたものであることを考慮している。

\item データ、産業

1970年代から'80年代にかけてチリで行われた大規模な貿易自由化を利用する。チリでは1974年から'79年にかけて、非関税障壁を完全に撤廃し、併せて関税も大幅に引き下げる、貿易自由化政策が採られた。その後景気後退に伴う一時的な中断、'84年には関税の引き上げが行われたものの、長期的には継続して貿易の自由化が促進された。

\item モデル、推定方法

貿易自由化による影響を推定するために、まず生産関数の推定を行う。

一般に、貿易の自由化は企業の生産性を向上させると考えられている。海外からの生産技術の流入や、競争の激化による生産資源利用の効率化が起こることが理由とされるが、この生産性の向上を特定するにあたっては、自由化政策を施行したあとの生産に関するデータが、全体の生産性が向上し、均衡価格が下落してことで淘汰された、比較的生産性の低いプラントを含まないものとなっている、セレクションバイアスの存在を考慮する必要がある。

\item 推定結果
 


\item シミュレーション
 


\item コメント



\end{enumerate}

\end{document}