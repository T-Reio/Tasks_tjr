\documentclass{jsarticle}
\usepackage{mathpazo}
\usepackage{amsmath,amssymb}
\begin{document}

\title{産業組織論 レポート課題5}
\author{経済学研究科 23A18014 丹治伶峰}
\date{}
\maketitle

Pavcnik(2002)

\begin{enumerate}

\item 目的

貿易自由化が企業の生産性に与える影響を、OP法による生産関数の推定によって明らかにする。

\item 新規性

従来の研究では、プラントの生産性を経時変化しない固定効果として取り込んだHarrison(1994)や、生産性を時間の二次式で表現したLiu(1993)、また、プラントの生産に関する意思決定を考慮しないものが多かった。そこでこの論文では、Olley and Pakes(1996)によって考案された生産関数の推定法(OP推定法)を実証研究に応用することで、従来の研究の課題となっていた、セレクションバイアスと同時性バイアスに対処する。具体的には、各生産プラントの生産効率が経時的に変化するという仮定をモデルに組み込み、更に観察される生産性が、貿易自由化によって海外の企業との競争を強いられた結果、市場から退出することを選択した、相対的に生産性の低いプラントを除いたものであることを考慮している。

\item データ、産業

1970年代から'80年代にかけてチリで行われた大規模な貿易自由化を利用する。チリでは1974年から'79年にかけて、非関税障壁を完全に撤廃し、併せて関税も大幅に引き下げる、貿易自由化政策が採られた。その後景気後退に伴う一時的な中断、'84年には関税の引き上げが行われたものの、長期的には継続して貿易の自由化が促進された。

\item モデル、推定方法

始めに、モデルのセッティングを行う。ある同一の市場における(: 同じ価格)各企業は、初めにプラント間で異なる生産性と保有する資本量に基づいて、生産活動を行うか、市場から退出するかを選択する。生産を行うプラントは、続いて生産要素の投入量を決定する。

まず、生産関数の推定を行う。

一般に、貿易の自由化は企業の生産性を向上させると考えられている。海外からの生産技術の流入や、競争の激化による生産資源利用の効率化、エージェンシー問題の改善が起こることが理由とされるが、この生産性の向上を特定するにあたっては、自由化政策を施行したあとの生産に関するデータが、全体の生産性が向上し、均衡価格が下落してことで淘汰された、比較的生産性の低いプラントを含まないものとなっていることによっておこるセレクションバイアスの存在を考慮する必要がある。また、この論文で用いるモデルは、後述する通り前期の資本と生産性に依存したプラントの生産性の経時的な変化を認めているとめ、同時性バイアスの問題にも対処する必要がある。

生産関数は、以下の通り定義されるコブ$\cdot$ダグラス型関数を用いる。

 \begin{align*}
 y_{it} &= \beta_0 + \beta x_{it} + \beta_k k_{it} + e_{it} \\
 e_{it} &= \omega_{it} + \mu_{it}
 \end{align*}

$y_{it}$, $x_{it}$, $k_{it}$, $e_{it}$ は、それぞれ時間$t$におけるプラント$i$の生産量、中間投入財、資本、それ以外の観察できない特性を対数表示したものを表す。観察できない特性は、プラント自身のみが観察可能な生産性$\omega$、プラント自身にも予測できない生産へのショック$e_{it}$とに分解される。

生産を行うプラントの利潤最大化問題は

\[ \max \Pi_{ijt} = f(k_{ijt}, \omega_{ijt}) \]

チリの労働契約に関する法規制や、実際の企業の行動から、企業は人的資本、及び各生産要素の投入量を自由に調整可能であると仮定する。一方で、資本量は前期の資本量と投資によって決まり、調整することが出来ないと考える。上の利潤最大化の解によって実現される利潤が、そのプラントを売却した時の価値$L$を下回る場合、プラントは生産を行わずに市場から退出するから、動学生産計画は、Value function $V_t$を用いて

 \begin{align*}
 V_t (\omega_t, k_t) &= \max \{L_t, sup \Pi_t(\omega_t, k_t) -c(i) + 
 				      d E[V_{t+1} (\omega_{t+1}, k_{t+1})| \Omega_{it}] \} \\
  & k_{t+1} = (1- \delta)k_t + i_t
 \end{align*}

と定義される。$c(.)$は投資$i$を実現する為の費用、$d$は時間割引因子、$\Omega_{it}$は時点$t$に得られる将来に関する情報である。市場の条件はプラント間で一定であるとする。

時点$t$に市場に存在するプラントは、保有する資本量と、それによって定まる生産性を元に、生産を行うかどうかの意思決定、及び当期の投資を決定する。

\[X_{t} = \begin{cases}
1, & \text{ if } \omega_t \geq \underline{\omega_{t}} (k_t) \\
0, & \text{ otherwise}
\end{cases} \]

\[i_t = i_t(\omega_t, k_t) \]

$\underline{\omega_{t}}$は生産を行う$\omega$の下限である。このモデルでは、この下限の値と投資を決定する関数の時間を通じた変化を認める。

\vspace{1zw}

以上の設定の下で、生産性の推定を行う。推定は、OP法(semiparametric estimation)を用いる。プラント自身によってのみ観察可能な生産性は、投資の関数から

\[ \omega_t = i^{-1}_t (i_t, k_t) = \theta (i_t, k_t) \]

これを生産関数に代入して

 \begin{align*}
 y_{t} &= \beta x_t + \lambda_t(k_t, i_t) + \mu_t \\
 \text{where} \\
 \lambda_t (k_t, i_t) &= \beta_0 + \beta_k k_t + \theta_t (k_t, i_t) \\
 \end{align*}

このとき、生産量$y_t$は誤差項$\mu_t$と相関しないため、ここから$\beta$の一致推定量が得られる。



\item 推定結果
 


\item シミュレーション
 


\item コメント



\end{enumerate}

\end{document}