\documentclass[dvipdfmx, 12pt]{article}
\usepackage{mathpazo}
\usepackage{amsmath,amssymb}
\usepackage{array}
\usepackage[hiresbb]{graphicx}
\usepackage{tikz}
\usepackage{textcomp}
\usepackage{dcolumn}
\usepackage{here}
\usepackage[top=25truemm,bottom=30truemm,left=25truemm,right=25truemm]{geometry}
\begin{document}

\parindent = 0pt

\title{Labor Economics Term Paper}
\author{Reio TANJI 23A18014}
\date{Due date: 2/1}
\maketitle

\paragraph{Q1} \hspace{1zw}

(1) The value functions of the worker are:

\[
\begin{cases}
  rW = w + \lambda (\bar{u} + f_e - W) & \text{ for employment} \\
  r \bar{u} = z + \theta q(\theta) (W - \bar{u}) & \text{for unemployment}
\end{cases}
\]

$W$ and $\bar{u}$ are the values of employment and unemployment. $w$ and $z$ are the instantaneous utility of employment and unemployment, respectively. $r$ is the discount factor, and $q(\theta)$ is the matching function of $\theta = \dfrac{v}{u}$. $v$ and $u$ is the vacancy rate and the unemployment rate of the workers, satisfying $q'(\theta) < 0$. When a worker was dismissed, they receive $f_e$ as the severance payment.

\vspace{1zw}

(2) The value functions of the firm are:

\[
\begin{cases}
  rJ = (p - w) + \lambda (V - f_e - f_a J) & \text{ for having a vacancy filled} \\
  rV = -c + q(\theta) (J - W) & \text{ for vacancy}
\end{cases}
\]

$J$ and $V$ are the values of vacancy filled and vacancy. $p$ is the the instantaneous productivity of the worker and $w$ is the instantaneous wage. $c$ is the expected hiring cost. When a firm fire a worker, then it have to pay the severance payment $f_e$ to the worker, and $f_a$ as the related cost for firing.

\vspace{1zw}

(3) By the free-entry/exit condition, $V = 0$ holds. Then, the value function of vacancy is rewritten as follows:

\[
J = \dfrac{c}{q (\theta)}
\]

Then, substituting this into the value function of having vacancy filled,

\begin{align*}
  rJ &= p - w + \lambda (V - f_e - f_a - J) \\
  p - w &= (r + \lambda) \left( \dfrac{c}{q (\theta)} \right) - \lambda (f_e + f_a) \\
  p - w &= \dfrac{(r + \lambda)c - \lambda (f_e + f_a)}{q (\theta)},
\end{align*}

which corresponds to the job creation condition.

\vspace{1zw}

(4) First, we derive the equilibrium conditions: wage condition and steady state condition.

First, we derive the wage curve.

Given the value functions in (1) and (2), the wage is determined by Nash bargaining rule:

\[
w = \arg \max_w (W - \bar{u})^\beta (J - V)^{1 - \beta},
\]
note that $\beta$ stands for the bargaining power of the worker.

\paragraph{Q2} \hspace{1zw}

(1)"Hazard Function" is an instantaneous rate of an individual moving from a certain state to another. In the settings of job searching, it corresponds to that of leaving from unemployment pool.

Let $T \geq 0$ denote the duration and the cumulative distribution function $F(.)$ is defined as follows:

\[
F(t) = \text{Pr}(T \leq t).
\]
Then, the hazard function $\lambda(.)$ is:

\[
\lambda (t) \equiv \lim_{h \to 0} \dfrac{\text{Pr}(t \leq T \leq t + h | T \geq t)}{h},
\]

where $\text{Pr}(t \leq T \leq t + h | T \geq t)$ is given as:

\[
\text{Pr}(t \leq T \leq t + h | T \geq t) = \dfrac{F(t+h) - F(t)}{1 - F(t)}.
\]

Then, we can rewrite the hazard function by $F(t)$ and $f(t) = F'(t)$,

\[
\lambda(t) = \dfrac{f(t)}{1 - F(t)}.
\]

Now suppose the duration is distribution according to the Weibull distribution. Then, the cumulative distribution $F(t)$ is given as follows.

\[
F(t) = 1 - \exp (- \gamma t^{\alpha})
\]

$\gamma$ and $\alpha$ are parameters. Then the density function is obtained as

\begin{align*}
  f(t) = F'(t) & = \dfrac{d (1 - \exp (- \gamma t^{\alpha}))}{dt} \\
  &= \dfrac{d (1 - \exp (- \gamma t^{\alpha}))}{d (- \gamma t^{\alpha})} \cdot \dfrac{d (- \gamma t^{\alpha})}{d t} \\
  & = \gamma \alpha t^{\alpha - 1}
\end{align*}

Substituting this into $\gamma(t)$, we obtain the hazard function.

\[
\lambda(t) = \gamma \alpha t^{\alpha - 1}
\]

Positive/Negative dependence of the hazard function on the duration is found by checking the first-order differential of the hazard function: if $\lambda' (.) > 0$, then it has positive dependence, and negative if $\lambda(.) < 0$. When the duration has Weibull distribution, by $\lambda'(t) = (\alpha - 1) \gamma \alpha t^{\alpha - 2}$. Since $t > 0$ and $\gamma \geq 0$ is assumed, it depends on the value of $\alpha$. If $\alpha > 1$, then the hazard ratio has the positive dependence on the duration, and $\alpha < 1$ implies negative dependence.

\vspace{1zw}

(2) Suppose observation of the worker who started job-searching after time 0. We observe unemployed workers who started job-searching until time $b$, and we stop observation at the cetain time: after then, we do not observe the worker, even though they have not finished searching. We define the time a worker $i$ enters the initial state (start searching) $a_i$, and time left for her/him until we stop the observation $c_i$.

Then, we cannot observe the true duration $t_i$ if the worker have not finished searching at the censoring time. ``Right-censoring'' stands for possible bias caused by this problem.

On the other hand, when we start observation of the workers who are searching at the time $b$: If s/he have finished searching at time $b$, then we cannot observe her/him, even if s/he started job searching after time $0$. Thus, obtained sample can be selected, that is, ``left-censoring'' problem occurs.

\vspace{1zw}
(3) Denote the ``observed'' duration to be $t^*$. Assume $t^*$ is distributed independent of $a_i$ and $c_i$, then the conditional cumulative distribution function $F(.)$ on the vector of observed covariates $X_i$ is defined as:

\[
F(t^* | X_i, a_i, c_i | \theta) = F(t_i | X_i; \theta).
\]
$\theta$ is the vector of parameters.

If there is no censoring problem discussed in (2), then the conditional density function of duration is $f(t^*_i | X_i; \theta)$. When an observation is completed, we can obtain the likelihood function of this.

Then, we argue the observation with incomplete duration. First, we consider the right-censoring problem. The probability of  right-censored observation is given by $1 - F(c_i | X_i; \theta)$. Then, the conditional likelihood function is obtained as follows:

\[
f(t_i^* | X_i; \theta)^{d_i} [1 - F(c_i | X_i; \theta)]^{1 - d_i}
\]

Second, we consider the left-censoring. All the observed samples satisfy that the individual is still unemployed at time $b$. That is, $t^*_i \geq b_i - a_i$. The probability of this is expressed by $\text{Pr}(t^* \geq b - a_i | X_i, a_i, c_i) = 1 - F(b - a_i | X_i)$.

By the Bayes Rule, the likelihood function including right and left-censoring is:

\[
L(\theta) = \prod_i^N \dfrac{f(t^*_i | X_i; \theta)^{d_i} [1 - F (c_i | X_i, \theta)]^{1 - d_i}}{1 - F(b - a_i | X_i; \theta)}
\]

Maximizing this (or log-likelihood function) with $\theta$, we obtain MLE: the consistent asymptotical normal estimated parameter $\hat{\theta}$.

\vspace{1zw}
(4) Holm\r{a}s. 2002. Keeping nurses at work: a duration analysis \textit{Health Economics}. 11: 493 -- 503 (2002)



\paragraph{Q3} \hspace{1zw}

(1)

\vspace{1zw}
(2)

\vspace{1zw}
(3)

\vspace{1zw}
(4)

\end{document}
