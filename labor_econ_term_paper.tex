\documentclass[dvipdfmx, 12pt]{article}
\usepackage{mathpazo}
\usepackage{amsmath,amssymb}
\usepackage{array}
\usepackage[hiresbb]{graphicx}
\usepackage{tikz}
\usepackage{textcomp}
\usepackage{dcolumn}
\usepackage{here}
\usepackage[top=25truemm,bottom=30truemm,left=25truemm,right=25truemm]{geometry}
\begin{document}

\parindent = 0pt

\title{Labor Economics Term Paper}
\author{Reio TANJI 23A18014}
\date{Due date: 2/1}
\maketitle

\paragraph{Q1} \hspace{1zw}

(1) The value functions of the worker are:

\[
\begin{cases}
  rW = w + \lambda (\bar{u} + f_e - W) & \text{ for employment} \\
  r \bar{u} = z + \theta q(\theta) (W - \bar{u}) & \text{for unemployment}
\end{cases}
\]

$W$ and $\bar{u}$ are the values of employment and unemployment. $w$ and $z$ are the instantaneous utility of employment and unemployment, respectively. $r$ is the discount factor, and $q(\theta)$ is the matching function of $\theta = \dfrac{v}{u}$. $v$ and $u$ is the vacancy rate and the unemployment rate of the workers, satisfying $q'(\theta) < 0$. When a worker was dismissed, they receive $f_e$ as the severance payment.

\vspace{1zw}

(2) The value functions of the firm are:

\[
\begin{cases}
  rJ = (p - w) + \lambda (V - f_e - f_a J) & \text{ for having a vacancy filled} \\
  rV = -c + q(\theta) (J - W) & \text{ for vacancy}
\end{cases}
\]

$J$ and $V$ are the values of vacancy filled and vacancy. $p$ is the the instantaneous productivity of the worker and $w$ is the instantaneous wage. $c$ is the expected hiring cost. When a firm fire a worker, then it have to pay the severance payment $f_e$ to the worker, and $f_a$ as the related cost for firing.

\vspace{1zw}

(3) By the free-entry/exit condition, $V = 0$ holds. Then, the value function of vacancy is rewritten as follows:

\[
J = \dfrac{c}{q (\theta)}
\]

Then, substituting this into the value function of having vacancy filled,

\begin{align*}
  rJ &= p - w + \lambda (V - f_e - f_a - J) \\
  p - w &= (r + \lambda) \left( \dfrac{c}{q (\theta)} \right) - \lambda (f_e + f_a) \\
  p - w &= \dfrac{(r + \lambda)c - \lambda (f_e + f_a)}{q (\theta)},
\end{align*}

which corresponds to the job creation condition.

\vspace{1zw}

(4)

\paragraph{Q2} \hspace{1zw}

(1)


\vspace{1zw}

(2)


\vspace{1zw}
(3)


\vspace{1zw}
(4)

\paragraph{Q3} \hspace{1zw}

(1)

\vspace{1zw}
(2)

\vspace{1zw}
(3)

\vspace{1zw}
(4)

\end{document}
