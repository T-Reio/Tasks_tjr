\documentclass[dvipdfmx,14pt]{beamer}

\usepackage{bxdpx-beamer}
\usepackage{pxjahyper}
\usepackage{minijs}
\usetheme{Annarbor}
\usepackage{mathpazo}
\usepackage{amsmath,amssymb}
\usepackage{graphicx}
\usepackage{array}

\title{生活保護給付システムの設計 \\ 過剰診療問題の効果測定
\subtitle{労働市場政策 期末発表}}
\author{Reio TANJI}
\date{July.17th,2018}
\institute{Osaka University}

\begin{document}
\begin{frame}\frametitle{}
\titlepage
\end{frame}

\section{Research Topic}
\begin{frame}\frametitle{概要}

 \begin{itemize}
 
 \item 生活保護給付システムが引き起こす問題を検討し、より精確な支給を行う為のシステム設計を行うためのヒントを引き出す
 
 \item 医療扶助のサービス提供者への直接給付
 
 : サービス提供者(医師)による過剰診療のリスク
 
 $\Rightarrow$公的統計に基づく調査で影響の大きさを測定
 
 \end{itemize}

\end{frame}

\section{Model}
\begin{frame}\frametitle{モデル}

 \begin{itemize}
 
 \item 伝統的経済モデル
 
 : 給付の形態は経済主体の行動に影響しない
 
 \item 実際には、給付の受け取り手によって受給者の経済活動が変化する可能性
 
 ex.) 住宅扶助を現金給付することで、給付が家賃以外の消費活動に流用されるリスク
 
 $\Rightarrow$サービス提供者(大家)に対価が支払われない
 
 \end{itemize}

\end{frame}

\begin{frame}\frametitle{医療扶助}

 \begin{itemize}
 
 \item 医療扶助はサービス提供者(医療機関)へ直接給付
 
 他の消費活動に対する支出によって受給者が必要な医療行為を受けようとしない問題に対処
 
 \item 過剰診療問題
 
 受診者の支払い能力に関わらず、かかった医療費が全額支払われることを見越して、過剰な診療 $\cdot$ 検査を実施
 
 \end{itemize}

\end{frame}

\begin{frame}\frametitle{制度変更による効率性の変化}

 \begin{itemize}
 
 \item 受給者本人への直接給付 : 正負の効果が共存
 
 現実的には、各システムによる厚生への損失が最小化される制度を採用すべき
 
 $\Rightarrow$各制度による効果の可視化が必要
 
 \end{itemize}

\end{frame}

\end{document}