\documentclass{jsarticle}
\usepackage{mathpazo}
\usepackage{amsmath,amssymb}
\begin{document}

\title{産業組織論 レポート課題4}
\author{経済学研究科 23A18014 丹治伶峰}
\date{}
\maketitle

Berry, Levinsohn \& Pakes(1999)

\begin{enumerate}

\item 目的

1981年にアメリカと日本の二国間で合意に達した、日本の対アメリカ自動車輸出の自主規制政策(VER)が、アメリカの自動車産業、社会総余剰に与えた影響を調べる。

\item 新規性

著者が1995年の論文で発表した、市場レベルのデータについて、消費者の選好の違いを組み込んで需要を推定するRandom Coefficient Model (BLP) モデルの実証利用。日米間のVERによる効果を推定する実証研究はこの論文以前にもいくつか存在し、様々な仮定の下で多様な研究結果が得られている。この論文では、静学寡占モデルを仮定し、VERによる厚生水準の変化を描写している。

\item データ、産業

VERは日本の対アメリカ乗用車輸出を年間1.68百万ドル(1981年当時、のちに引き上げ)に自主規制する、当該二国間における合意で、アメリカ政府とMinistry of Trade and Industry(MITI)の提言によって実現した。MITIは、過去の生産量を元に日本の各メーカーに対して、具体的な生産量の割り当ても行っている。実際の輸出量としては、1981年、'83年から'86年の各期間においては超過が見られる一方で、'87年以降については規制未満の輸出量が実現しているため、VERが輸出の制限に遅れて影響をもたらしたとも考えられるが、同時期の日本企業が在庫投資を増加させていることや、景気の後退があったことから、他の要因も考えられるため、VERがもたらした影響を精査する必要がある。

\vspace{1zw}

データについては、1971年から1990年までの期間における、自動車産業の製品レベルのモデルの数、額面価格、販売数と製品の性能に関する特性を集計。またマクロ経済指標として為替レート、消費者物価指標等、家計に関するデータとして家計数と所得の分布をそれぞれ集計し、時間$t$と製品$m$によるパネルデータを生成する。製品価格は、消費者が個別に追加したオプションに係る費用が観察出来ないため、基本的なモデルの価格を用いている。

製品の質の代理変数としては、馬力を重量で除したエンジン性能、車体の長さと幅の積で表すサイズ、エアコンが搭載されているかどうか(ダミー変数)、ガソリン1ドル分当たりの走行距離で表す燃費を用いる。

\vspace{1zw}

対数価格に対して製品の性能(対数)、地域、景気変動、対数為替レート(前期のレートを含む)、各年度についてVERが適用される製品であること、アメリカ国内で生産された製品をそれぞれ示すダミーを用いたOLS推定(hedonic model)においては、VERの価格に対する影響はいずれの年度においても負で有意であった。これは輸入量制限を行った政策による予想とは正反対の効果であるが、先述したように背景に存在するメカニズムについては更に詳細な検討を行う必要がある。

\item モデル、推定方法

BLPモデルを用いる。消費者が異なる質的特性をもつ製品(J種類、製品を購入しないoutside goodsの存在を許す)の中から、自らの所得制約や質的特性に対する感応度を勘案して効用を最大化する、離散選択モデルを仮定する。また、質的特性に対する感応度は個人間で異なることを許す、Random Coefficient モデルを用いる。消費者の効用関数は

\[U(v_i, p_j, x_j, \xi_j ; \theta ) \]

で与えられる。$v_i$は消費者$i$の所得、世帯人数などの特性、$p_j$は製品$j$の価格、$x_j$、$\xi_j$は製品$j$のそれぞれ観察可能/不可能な特性を表す。$\theta$は推定されるパラメータである。

ここで、製品$j$から得られる効用が他のどの製品$r$から得られるそれよりも高くなる、即ち製品$j$を選択するような個人属性$\mathbf{v}$の集合を$\mathbf{A}_j$とおくと、製品$j$が市場に占めるシェア$s_j$は、以下のように定義される。

\[s_j(\mathbf{p}, \mathbf{x}, \xi ; \theta ) 
= \int_{\mathbf{v} \in \mathbf{A}_j(\theta)} P_0 (d \mathbf{v}) \]

$P_0$は$\mathbf{v}$の分布を表す。

効用関数は、各消費者の、製品の質的特性に対する嗜好の多様性を考慮したRandom Coefficientモデルを用いて

 \begin{align*}
 u_{ij} = x_j \Bar{\beta} &+ \xi_j - \alpha p_j + \sum_k \sigma_k x_{jk} v_{ik} + \epsilon_{ij} \\
 \text{for} & \hspace{1.5zw}j = 1,...,J \\
 \text{while} & \hspace{1.5zw} u_{i0} = \sigma_0 v_0 + \epsilon_{i0}
 \end{align*}

で表される。$\epsilon$は誤差項を表す。$x_j \Bar{\beta} + \xi_j $は個人$i$に依存しない部分で、$\Bar{\beta}$、$\alpha_i$を推定する。但し、$\alpha_i$は個人$i$の所得によって変動する、需要の価格弾力性を示しており、$\alpha_i = \dfrac{\alpha}{y_i}$で表され、実際には$\alpha$を推定することになる。$x_i$には、hedonicモデルと同様に、馬力を重量で除したエンジン性能、車体の長さと幅の積で表すサイズ、エアコンが搭載されているかどうか(ダミー変数)、ガソリン1ドル分当たりの走行距離で表す燃費を用いる。

\vspace{2zw}

次に、供給サイドの限界費用を推定する。

$J$種類の製品に対して、企業数は$F$で表され、一つの企業が複数の製品を生産することを仮定する。限界費用は製品の観察可能な特性によって変動し、Cost Shifter $w_j$に対して線形の

\[ \ln (mc)_j = w_j \gamma + \omega_j \]

で記述される。

この時、企業$f$の利潤関数は

 \begin{align*}
 \pi_f = & \sum_{j \in J_f} (p_j -mc -\lambda \textit{VER}_j) \times \\
  & \times Ms(\mathbf{p}, \mathbf{x}, \xi ; \theta) - \sum_{j \in J_f} \textit{Fixed Costs}_j
 \end{align*}

VERの効果は、生産に対して定率の関税が課された場合と同様の方法でモデルに組み込まれている。シェアは需要推定によって求められたパラメータを用いて算出される。

各企業のマークアップ$\mathbf{b}$を、ownership matrix を内包する行列$\Delta_{jr}$を用いて

\[\mathbf{b}(\mathbf{p}, \mathbf{x}, \xi ; \theta) 
= \Delta(\mathbf{p}, \mathbf{x}, \xi ; \theta)^{-1} \mathbf{s}(\mathbf{p}, \mathbf{x}, \xi ; \theta) \]

と表される。以上から、企業$f$の利潤最大化の一階条件は

 \begin{align*}
 \ln( &mc_j) \\
  & = \ln (p_j - b_j (\mathbf{p}, \mathbf{x}, \xi ; \theta ) - \lambda \textit{VER}_j ) \\
  & = w_j \gamma + \omega
 \end{align*}

\vspace{2zw}

推定にはGMMを用いる。パラメータ$\theta$に依存して変動する$xi_j$,$\omega_j$と、操作変数の関数$H(.)$によって定義されるモーメント条件は、

\[G^J(\theta) = \dfrac{1}{J} \sum ^J_{j=1} E \left[ H_j(z) \begin{pmatrix}
\xi_j(\theta) \\
\omega_j(\theta)
\end{pmatrix}
 \right] = 0\]

パラメータの最適化条件は

\[  \min ||G_J (\theta) ||_{A_j} \]

操作変数には製品の観察可能な特性$\mathbf{x}$、及びCost shifter $w_j$の内、賃金や生産要素価格など、$\mathbf{x}$に含まれないものを用いる。

\item 推定結果
 
需要推定については、製品特性を示す変数の係数$\Bar{\beta}$が、平均、分散ともにすべての変数について統計的に有意であることが示された。また、収入に対する需要弾力性を示す$\alpha$は正で($-p/y$が変数として導入)、直感と整合的であった。

\vspace{2zw}

供給サイドのCost shifterにはhedonicモデルと同様に、製品特性を表す変数と、景気変動、生産を行う企業の国籍、為替レートを用いる。推定の結果、VERがあたかも日本企業にとって課された税金と同様の機能を果たしたことが分かった。

VERの影響を受けたことを示すダミー変数の係数を年度別に考察すると、この生産調整に関する合意がなされた直後の'81年から'85年については、影響の統計的な有意性が見られなかった。ただし、'84年と'85年については、日本企業に対して\$600-\$800の税が課されていた、という帰無仮説を棄却出来ず、何らかの影響が存在していた可能性も残された。自動車産業は、景気後退や利子率の上昇の影響を強く受ける周期性の産業であり、当該期間中にそうした現象が観察されたことが、VERの影響を被覆した可能性がある。また、不景気の下では、製品値上げが功利的な戦略であると受け取られ、売上に悪影響を及ぼすことを危惧した日本企業が値上げを回避したと考えられる。

一方で、'86年以降については、'90年までのすべてのダミーの係数が有意に日本企業の限界費用を高め、平均して8.2\%の税として機能していた、という結果が得られた。影響は'87年、'88年にピークを迎え、その後下降している。これは、日本国内での生産、輸出を行っていた企業がアメリカへの直接投資を行う形に切り替えた影響が考えられる。

以上の結果より、VERは自動車の製品価格を上昇させ、消費者余剰と引き換えに、国内企業の利潤を守ったと結論している。また、その他の指標については、日本車、欧州製の自動車の方が平均的に限界費用が大きく、また一方で為替レートは当期、前期ともに大きな影響を及ぼさなかった。

\item シミュレーション
 
 推定結果を元に、VERが存在しなかった場合の企業の利益と消費者余剰を求める。消費者の嗜好についてはモンテカルロ法を用いてシミュレーションを行った。これによると、VERによって日本車の価格は上昇し、これによって(アメリカの自動車の価格は変化しなかったにもかかわらず)価格への感応度が高い消費者が日本車から国産車に切り替え、アメリカの企業の利潤を改善したことが観察された。一方で、日本企業の利潤も大きく変化することはなかった。ここから、日本車の価格に対する需要弾力性の小さい消費者がついており、彼らの存在が日本企業の余剰悪化を防いだと考えられる。
 
 一方、この余剰の改善は、消費者の厚生悪化によって実現されたことも確認された。VERによって、消費者余剰は低下し、この影響は特に値上げの起こった日本車を購入した消費者において最も大きく、逆にアメリカの車を購入した消費者については、それほど大きな余剰変化は見られなかった。
 
 VERを税金と考えると、税収による余剰が消費者余剰の減少を上回れば効率性の改善と解釈することも出来るが、この余剰はほぼ消費者余剰の減少と等しいことが分かった。VERは他国の生産、貿易を制限する為、相手国からの報復措置を受けることが考えられるが、筆者はVERによる'税収'が、この報復措置を避けるための賄賂として機能した、としている。また、自由貿易を推進するトレンドから、関税の設定は国際的な非難を受ける可能性があったため、関税をVERとして導入することで、この批判を避けることができた、ともしている。
 
 \vspace{2zw}
 
 推定の頑健性を確認するため、Main Resultにおけるものと代替的な仮定をおいた下でのシミュレーションを行い、VERがRandom Coefficient モデルにおいて、産業にどのような影響を与えたかを精査する。具体的には、
 
  \begin{enumerate}
  
  \item Bertrand competitionの代わりにCournot competition を仮定する
  
  \item  Mixed Nash 
  
  VERによる生産調整をより精確に描写するため、日本企業は供給量、それ以外の企業は価格を調整するモデルを考える
  
  \item VERによって、日本企業同士がCollusion を行ったと仮定する
  
  \item 日本企業がアメリカの生産設備で生産を行った場合はVERによる制約を受けないことを考慮し、これらを除外して日本企業の利潤の変化を考える
  
 \item 日本に対するアメリカ車の強制的な輸出を除外する
 
 \item VERの影響がマクロ経済指標に影響し、間接的に消費者の経済行動に影響を与える可能性を考える
 
 \item VERの影響が、同じ日本企業であっても、その規模や成長のベクトルによって異なる可能性を考え、トヨタと日産、及びトヨタとホンダの製品であることを示すダミー変数を導入する
 
 \item 一般的な消費者の経時的な選好の傾向に変化が起こる場合(選好の分布が変化する)
 
 \end{enumerate}

について、それぞれ検証する。その結果、各モデル間で推定の統計的な有意性の微妙な変化は存在したものの、VERダミーが'80年代後半において、日本車の価格を引き上げる影響を及ぼした、という基本的な結果が大きく変わることはなく、Main Result の推定結果の頑健性を認めている。

最後に筆者は、この論文と異なる見解を打ち出している複数の先行研究との比較を行い、データサンプルの取り方や、利用出来た推定方法の違いなどが、この要因になったとしている。また、応用研究として、自動車が耐久財であることを考慮した動学的な分析アプローチや輸出するモデルの選択の内生性の考慮、他の変数の導入や消費者の選好の変化を考慮したモデルによる分析を提案している。

\item コメント

この論文では、特定のショックが産業に与えた影響を測る際に、サンプルをショック以前の期間からも広くとることで、先行研究とは異なる、分析結果を引き出している。また、分析結果の頑健性を確かめるために多くの異なる仮定を用いたシミュレーションを行っている。この論文が扱ったVERは'81年になされた政策であり、より後の時代のショックに対してBLPを用いてショックの影響を推定することが応用研究として考えられる。

\end{enumerate}

\end{document}