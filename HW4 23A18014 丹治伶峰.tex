\documentclass{jsarticle}
\usepackage{mathpazo}
\usepackage{amsmath,amssymb}
\begin{document}

\title{産業組織論 レポート課題4}
\author{経済学研究科 23A18014 丹治伶峰}
\date{}
\maketitle

Berry, Levinsohn \& Pakes(1999)

\begin{enumerate}

\item 目的

1981年にアメリカと日本の二国間で合意に達した、日本の対アメリカ自動車輸出の自主規制政策(VER)が、アメリカの自動車産業、社会総余剰に与えた影響を調べる。

\item 新規性

著者が1995年の論文で発表した、市場レベルのデータについて、消費者の選好の違いを組み込んだRandom Coefficient Model (BLP) モデルをApplyしている。

\item データ、産業

VERは日本の対アメリカ乗用車輸出を年間1.68百万ドル(1981年当時、のちに引き上げ)に自主規制する、当該二国間における合意で、アメリカ政府とMinistry of Trade and Industry(MITI)の提言によって実現した。MITIは、過去の生産量を元に日本の各メーカーに対して、具体的な生産量の割り当ても行っている。実際の輸出量としては、1981年、'83年から'86年の各期間においては超過が見られる一方で、'87年以降については規制未満の輸出量が実現しているため、VERが輸出の制限に遅れて影響をもたらしたとも考えられるが、同時期の日本企業が在庫投資を増加させていることや、景気の後退があったことから、他の要因も考えられるため、VERがもたらした影響を精査する必要がある。

\vspace{1zw}

データについては、1971年から1990年までの期間における、自動車産業の製品レベルのモデルの数、額面価格、販売数と製品の性能に関する特性を集計。またマクロ経済指標として為替レート、消費者物価指標等、家計に関するデータとして家計数と所得の分布をそれぞれ集計し、時系列データを生成する。

製品の質の代理変数としては、馬力を重量で除したエンジン性能、車体の長さと幅の積で表すサイズ、エアコンが搭載されているかどうか(ダミー変数)、ガソリン1ドル分当たりの走行距離で表す燃費を用いる。

\vspace{1zw}

対数価格に対して製品の性能(対数)、地域、景気変動、対数為替レート(前期のレートを含む)、各年度についてVERが適用される製品であること、アメリカ国内で生産された製品をそれぞれ示すダミーを用いたOLS推定(hednic model)においては、VERの価格に対する影響はいずれの年度においても負で有意であった。これは輸入量制限を行った政策による予想とは正反対の効果であるが、先述したように背景に存在するメカニズムについては更に詳細な検討を行う必要がある。

\item モデル、推定方法



\item 推定結果
 
\item コメント

\end{enumerate}

\end{document}