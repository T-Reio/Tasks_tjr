\documentclass[dvipdfmx, 10.5pt]{jsarticle}
\usepackage{mathpazo}
\usepackage{amsmath,amssymb}
\usepackage{array}
\usepackage[hiresbb]{graphicx}
\usepackage{tikz}
\usepackage{textcomp}
\usepackage{dcolumn}
\usepackage{here}
\usepackage{lscape}
\usepackage[top=25truemm,bottom=30truemm,left=25truemm,right=25truemm]{geometry}
\begin{document}

\section{第1日}

\subsection{シンポジウムI}

野球人口減少への取り組み

\subsubsection{原田先生}

  \begin{itemize}
    \item 弘前市
    \begin{itemize}
      \item 児童数減少に伴う野球人口の減少(10年間で70\%以上):

      親の負担(毎日練習・経済的な負担)

      過熱した指導(指導法・試合数の異常な増加)

      取り掛かりの後れ(サッカー・体操・剣道)
    \end{itemize}
    \item 少年野球教室

    コーディネーショントレーニング

    自ら考える練習

    \item 冬場の練習環境提供と未経験者の子供への教室を並行

    鬼ごっこ・ストラックアウト・スタンドティー野球

    \item 指導法の共有・未経験者(女子児童を含む)の取り込み・チームに

    \item 指導を行う高校生の成長・教員希望者の増加

    $\Rightarrow$'16-'17の一年間で28\%増加

    \item 保育園への教室の拡大

    保育士が野球の遊び方・道具の入手方法が分からない

    やったことのないスポーツ・新鮮味

    高校生ならではの視点

    社会性育成への貢献(アウトを取るための自発的な問題解決)

    並びっこ野球

    \item 展望 高野連弘前地区全体での取り組み$\Rightarrow$野球人口増加・高校生の育成

  \end{itemize}

\subsubsection{岡本先生}

  「次世代型少年野球チームの取り組みと少年野球改革」

  \begin{itemize}
    \item 春日学園少年野球クラブの指導法
    \begin{enumerate}
      \item

      \item 週末1/4ルール:土日の午前午後どちらかのみの活動

      \begin{itemize}
        \item 故障リスクの低減

        \item 残った時間の有効活用:家族との時間・他のスポーツ・勉強

        *自主練習は一切制限しない
      \end{itemize}

      \item 大学院生による指導

      \begin{itemize}
        \item 練習メニュー・メンバー選定・スタメン選定の全権を委託

        \item 監督=マネージャー

      \end{itemize}

      \item 適度な試合数と厳密な投球制限

      \begin{itemize}
        \item 3,4年50、5年60、6年70球

        $\Rightarrow$ 野球肘 7名

        \item 大会数も学年ごとに制限

        $\Rightarrow$「もっと野球がしたい」を生み出す・その状態で中学に送り出す
      \end{itemize}

      \item 父母会設立の禁止

      :父親・男性中心の運営(男性の方がマネジメントに向いている・秩序形成)・母親のマネージャー化 親を雑用に使わない

      \item 勝利至上主義の否定

      ルールを遵守しつつ徹底した競争を実施:勝ちに行く姿勢

    \end{enumerate}

    \item 野球人気低下の原因

    \begin{itemize}
      \item 古い

      :お茶出し 罵声 $\Leftarrow$どう変えていくか?

      \item 雑用
    \end{itemize}

    \item 高校野球改革

    :学童野球からの選手供給方法の模索

    \item ごっこ遊びでは伝わらない野球の醍醐味

    日本人と野球の相性?

    \item 大きな大会(マック杯・日ハム杯)の制度改革

    罵声指導の禁止

    100球の球数制限

    \item ライセンス制度の一般化

    単一団体の取り組み$\Rightarrow$全体

    権威の確保・改革のインパクト

    cf.徳川綱吉 生類憐みの令

  \end{itemize}

\subsubsection{勝亦先生}

  「課題解決から野球人口増加へのアプローチ」

  \begin{itemize}
    \item 野球の非日常化・少子化・習い事化

    \item 競技人口減少

    \item スポーツ教育の偏重

    \item 少年期の怪我増加

  \end{itemize}

  「早稲田大学OBの取り組み」

  \begin{itemize}
    \item 「知識は学問から、人格はスポーツから」

    \item 大学・地域との連携

    \begin{itemize}
      \item プレイボールプロジェクト

      \begin{enumerate}
        \item 毎年1回のイベント

        \begin{itemize}
          \item 大人主導・勝利至上的な指導

          \item 受動・従属

          「役割・ポジションの固定」

          $\Rightarrow$「子供に野球を返す」

          \item かんたんベースボール

          \begin{enumerate}
            \item 監督不在

            \item 個・主体性の優先

          \end{enumerate}

          \item 満足度の増加・知らない選手とのプレー

          \item 野球に触れる・楽しむ機会の創出

          \item 中学から始めることを後押しするアプローチ

        \end{itemize}

        \item 定期的なイベント

        \begin{itemize}
          \item 「日常」に野球を取り戻す

          \item 「遊び場」としての野球場の提供 野球遊びへの誘導

          \item バットを使った遊びの楽しみ$\Rightarrow$少年野球への興味との乖離

          :中学からの競技開始

        \end{itemize}



      \end{enumerate}
    \end{itemize}
  \end{itemize}

スポーツの人口維持・増加へのアプローチ

一競技への集中を避ける

技術指導を行わないための技術指導:野球以外のエクササイズ

指導者の変革・確保

成功体験のトップダウン的な継続に対する理解はある

中学高校大学:子供の環境を知らない・30で子供が生まれる$\Rightarrow$30年以上のジェネレーションギャップ

高校生との交流には大きな意味があるのでは

日本とドミニカの違い:さほど大きくないのでは?

強度の高い練習を求める$\Rightarrow$これが「自主練習」の意味なのでは


\subsection{ポスターセッション}

\subsection{情報交換会}

\subsection{その他}

\section{第2日}

女子野球の躍進とこれから

\subsection{シンポジウムII}

\subsubsection{山田先生}

\begin{itemize}
  \item 女子野球の歴史は100年以上

  \begin{itemize}
    \item 1926年、文部省の指導による女子野球の規制

    \item 1940年代- 女子野球チームが誕生、一時は20チーム程が活動

    \item 1967年 プロリーグ休止、広告塔としての女子野球の振興

    \item 1986年大学女子軟式連盟が活動

    \item 1997年高校全国

    \item 2004年 ワールドカップ開始

    \item 2018年、20,000人を超える

    $\Leftarrow$高校教員による普及活動・女子プロ野球・NPBガールズトーナメント・マドンナジャパンの躍進など
  \end{itemize}

  \item 女子野球のこれから

  \begin{enumerate}
    \item 人材育成

    : 女性審判・技術指導

    \item 普及活動

    \item 競技力向上・医学

    \item 国際交流・貢献

    \item マーケティング

  \end{enumerate}

  \item ブームから文化へ

\end{itemize}

\subsubsection{橘田監督}

\begin{itemize}
  \item 侍ジャパン女子
  \begin{itemize}
    \item ワールドカップ6連覇

    \begin{itemize}
      \item トライアウトによる選考

      \item 書類・動画による審査
    \end{itemize}

  \item 監督としての役割
  \begin{enumerate}
    \item 選手との対話型コミュニケーション

    \item 選手の理解

    \item 問題提起ができる環境づくり
  \end{enumerate}

  \item すべてにおいて世界一になる

  \item 異なるカテゴリーからの選出

  \begin{itemize}
    \item カテゴリー間の大会時期のズレ
  \end{itemize}

  \item 男女差

  \begin{itemize}
    \item 経験値の差

    \item 筋力差

    \item 環境の違い
  \end{itemize}

  $\Rightarrow$選手に事前説明してコミットメント・自分たちで決めたことならやる(時代錯誤でも?)
  \end{itemize}

  方法論とは別では?
\end{itemize}

選手とのジェネレーションギャップ: 岡本先生の話とも関連

\subsubsection{小林先生}

女子サッカーの歩み

\begin{itemize}
  \item ドイツワールドカップ優勝

  \item 「男子」サッカーという呼称の誕生

 \item 歴史

  \begin{itemize}
    \item 野球と同様に女子選手の制限

    \item `91年・初の国際大会$\Rightarrow$アトランタ五輪で正式種目に
  \end{itemize}

  \item ロサンゼルス宣言

  -- FIFAによる支援

  -- 意思決定機関への女性登用

  \item JFL Women's Project

  \begin{itemize}
    \item Women's College

    \begin{enumerate}
      \item 47都道府県での講習会・映像資料

      \item 女子指導者の育成・指導者ライセンス取得の推奨
    \end{enumerate}

    \item 三位一体+普及

    \begin{itemize}
      \item チーム強化

      \item ユース育成

      \item 指導者育成
    \end{itemize}

    cf. JFA指導者ライセンス・なでしこvision (2007, 2018)

    \item 各地域にトレーニングセンター (ナショナルトレセン) を設置、チームのレベルによらず有力選手をピックアップ

    \item 技術に対する注目

    母親の賛同--岡本先生の話と関連

    \item なでしこリーグのチームの多くがJの下部組織化

    全国展開(ホーム\&アウェイ方式)・下部組織の所持を義務化

    男性の参加

  \end{itemize}


\end{itemize}

\subsection{オンコートレクチャー}

メジャーリーガー・プロ野球選手のコンディショニング

\subsubsection{井脇さん}

\begin{itemize}
  \item 田澤投手
  \begin{itemize}
    \item 162試合・時差で終日休めない

    $\Rightarrow$いかに休日・リカバリーを作るか

    \item 技術+コンディションの波を最小限に

    \item 選手とのコミュニケーション+短期vs長期のパフォーマンス
  \end{itemize}

  \item 筋肉は立体的・表面が回復していても深い部分に疲労が蓄積

  $\Rightarrow$イメージとの乖離・反応・動作の違和感

  \item 関節・筋肉のケアを丁寧に・違和感の除去

  \item STの練習強度は非常に低い

  $\Leftarrow$シーズン中は維持が主に・「貯金を作る」

  \item 不調時: 代謝の改善・睡眠

  \item 瞬間的な出力に頼らない・予備動作からの力の伝達

  \item トレーナーが「全能感」を持たない

  \item コンディショニング=日々の比較

  \item 目的を意識した手段の選択「準備」「継続」「徹底」

  \item 「張り」「硬さ」の要因

  骨際の硬さ: 筋収縮能力の低下$\Rightarrow$力の伝達・イメージ通りの動きの実現に対する障害に(特にブレーキング系の筋肉は実現可能な出力に直結)

  筋肉の深層のケア

  「肩甲上腕リズム」

  \item 「肘が詰まる」: 肘関節の筋肉が収縮してリリースに影響$\Rightarrow$腕全体を使えない(一部分への過剰な負担)

  \item 代償作用: 負荷のかかる部分の筋量が十分でないためにその他の部分の出力で補おうとする

  \item 足底・筋の張りがパフォーマンスに大きな影響: 「膝から先で捌く(足全体が使えない)」動作に

  \item 臀筋 大きな筋肉

  \item 股関節・腸腰筋: 関節がはまらずに代償動作で逆方向の力に

  \item 内外の筋肉のバランス

  \item 肋幹 (回転運動)

  \item 仕草の違和感から故障を察知

\end{itemize}

\subsubsection{吉井さん}

\begin{itemize}
  \item 登板間のケア

  \begin{itemize}
    \item MLB

    \begin{itemize}
      \item 中4日ローテ

      \item ジョギング

      --疲労除去・リラックス

      \item PP16

      \item 50・30Mダッシュ

      \item 10M

      \item ウェート 登板間2回 60\%・前日80\%

      \item ブルペン2回
      : マダックス 強度低(50\%, ``Touch and Feeling'')・球数少で毎日
    \end{itemize}

    \item クローザー

    \begin{itemize}
      \item 運動量の管理
    \end{itemize}

    \item オフ
    \begin{itemize}
      \item 体力づくり

      \item 2週間休養$\Rightarrow$2月半ばまでトレーニング

      \item 筋力トレーニング

      \item 投球

      : 投球練習も強度を抑えつつできるだけ継続 週3
    \end{itemize}

    \item コーチ

    \begin{itemize}
      \item レギュラークラス 技術的な指導はしない

      :コンディショニング・心的なケアを重視

      \item 当落

      : 若手には明確な目標を選手主導で設定

    \end{itemize}

    \item 違い

    \begin{itemize}
      \item 昔は体力勝負

      \item 現在はトレーニング方法が多様化・基礎体力の養成が疎かに

      : 選手自身が課題を意識・そのためのアプローチを検討

      \item 短・中距離ランの強度は維持・管理
    \end{itemize}

    \item 起用する側とコンディショナーの不一致

    :選手は行けるかと聞かれたら行く

  \end{itemize}

  \item ボールが悪い$\Rightarrow$フォームを維持・実現するための筋肉・身体的な問題の可能性も
\end{itemize}

\subsection{ワークショップ}

力検出型センサーバットによる打撃動作の分析

\subsubsection{センサーバットが教えてくれるもの}

\begin{itemize}
  \item 両手打撃動作: 機構的な閉ループ系

  握力と操作力の同時作用

  \item 閉ループ問題

  打具のキネマティクス的自由度は、両手のそれに比べて小さい(モーメントの自由度分)

  動作計測データだけではその力の分解ができない$\Rightarrow$直接計測する必要

  \item スイング平面座標系(瞬間ごとのバットの動きの方向・力を可視化: バレル側・ノブ側別) 地面反力 ひずみのデータを取得

  \item $x$軸: ノブ側とバレル側の力が偶力・回転力だけが発生

  \item $y$軸: ノブ側の方が大きな力

  ヘッドスピードと強い相関

  高い各速度による高い遠心力・抑えるための力(当たり前っちゃ当たり前)

  \item $z$軸: 構えた位置から落下$\Rightarrow$レベル方向へ変化

  \item フォローに関する研究はしていない

\end{itemize}

\subsubsection{ヘッドスピード獲得の仕組み}

ムチ動作の生成メカニズム

\begin{itemize}
  \item ノブ側の作用力が優勢

  $\Leftarrow$ 動きが高速になると、筋が力を発揮

  \item 運動方程式

  \[
  f = ma
  \]

  $a$: 加速度

  \item 並進/回転 の加速度 = 関節トルク + 重力 + 運動依存(遠心力・コリオリ力・ジャイロ力)

  \item 関節トルク: 即時的効果

  $\Rightarrow$角速度の生成

  \item 運動依存項が発生 : インパクト直前に力を集積 = 累積的(ムチ効果)

  力を入れなくても加速度が発生するメカニズム

  *野球は多種目(やり投げ・テニスのサーブ)と比較して、関節トルク項が全運動量に対して占める割合が小さい

  運動依存項の生成要因を定式化

  一般化速度ベクトル$V(k)$に対する漸化式

  \item ヘッドスピードの違いは運動依存項生成技術の差に依存する: 関節トルク項は大きな違いを生み出さない

  \item 関節トルクの貢献度の中では上肢関節・トルソ関節(ムチ動作とは相関しない)の貢献度が大きい

  \item トルソの回旋には両足の股関節屈曲進展トルクが正の貢献・軸足股関節の内外転軸が負の貢献

  \item インパクトの時の関節トルクはムチ動作の生成に大きな影響を及ぼさない

  前足接地時の貢献度が大きい

  トルソ関節フォワード回転

  ノブ側方内転$\Rightarrow$外転$\Rightarrow$バレル側肩内旋トルクの順にヘッドスピードに貢献

\end{itemize}

cf.
\begin{itemize}
  \small

  \item ひずみ 物体の単位長さあたりの変形量

  \item ひずみゲージ ひずみを電気抵抗の変化に変換

  \item 偶力: 方向が逆で大きさが同じ二つの力$\Rightarrow$回転作用のみを発生させる

  \item 関節トルク: 筋張力と筋の拮抗作用により生じる、関節を回転させようとする力

  \item トルソ関節: 体軸の回転を行う動き(仮想の関節)
\end{itemize}

\subsubsection{ピッチングとの関係}

\begin{itemize}
  \item 打撃と同様、運動依存項の生成する力が大きい

  \item 肩の内旋: 肩の水平内転で相殺: ムチ動作が作用しない
\end{itemize}

\subsection{提言}

\subsubsection{スポーツ社会学の視点からのスポーツ・アスリートを育成するための提言}

\begin{itemize}
  \item 野球とベースボールの違い

  : 楽しめないイメージ$\Leftarrow$ 監督・サインを出す人物の存在(???)

  \item 野球の成立要素ごとに野球を「始めたくない」「続けたくない」要素を列挙

  なんかいろいろ書いてたけど、野球に限った話じゃないし、よくわからんかった。

  \item コーチの役割: 選手とのビジョンの共有・環境整備・人間関係の構築・指導と準備・現場に対する理解と対応・学習と自身の振り返り

  \item 学術文献の参照・経済学はほとんどなし

  \item 育成ガイドライン: どうやって実現するのか、構造を作るか

  \item 年代別の練習ドリルの作成
\end{itemize}

\subsubsection{二段モーションについて、科学的視点からの提言}

\begin{itemize}
  \item 大学生投手を用いた実験

  二段モーションで投げることに大きな差異はない

  やってるものを急に止めると悪影響の可能性はある=「普段と違うフォームで投げる」ことの効果との識別が不十分

  DIDを使えば見られるのでは

  \item 投手のモーション付き打撃マシンを用いた打撃に対する影響

  大きな差異は見当たらない

  前足離地の時間には差が出るが、接地の時間には30ms程度の差

  上げたり上げなかったりすることが問題なのでは?

  「自然な投球動作」とは?
\end{itemize}

\subsubsection{ジュニアからユースまでの育成・予防・安全管理}

\begin{itemize}
  \item 予防の種類

  \item ジュニア期の肘障害が成人後のリスクに影響する
\end{itemize}

\subsection{その他}

事前知識が必要・分野が幅広い

\end{document}
