\documentclass{jsarticle}
\usepackage{mathpazo}
\usepackage{amsmath,amssymb}
\usepackage{array}
\begin{document}

\title{Behavioral Economics \\
Exercise 4 Time Preferences}
\author{経済学研究科  \\ 23A18014 Reio TANJI 丹治伶峰}
\date{}
\maketitle

\begin{enumerate}

\item [Question 1]

\begin{enumerate}

\item

Solve by backward induction.

\begin{itemize}

\item Period 2

By $\beta = \delta = 1$, $t=2$ agent's intertemporal utility is :

\begin{align*}
-c(e_2) & + B(e_1,e_2) \\
\text{where } &  e_1 \text{ is given}
\end{align*}

Then, the best-response of the $t=2$ agent, depending on the parameters is derived as follows :

If $B \geq kc \Leftrightarrow c \leq \dfrac{B}{k}$

\[\text{BR}(e_1) = \begin{cases}
2 & \text{if } e_1=0 \\
1 & \text{if } e_1=1 \\
0 & \text{if } e_1=2
\end{cases} \]

and if $kc > B > 2c \Leftrightarrow c>\dfrac{B}{k}$

\[\text{BR}(e_1) = \begin{cases}
1 & \text{if } e_1=1 \\
0 & \text{if } e_1=0, 2
\end{cases}\]

\item Period 1

Now by $\beta = \delta = 1$, $t=1$ agent's intertemporal utility is :

\[ -c(e_1) - c(\text{BR}(e_1)) + B( e_1, \text{BR}(e_1)) \]

By the assumption $kc > 2c$ and $B>2c$ and the derived best-response of the $t=2$ agent, $e_1$ is the dominant strategy for each case of the parameters.

\newpage

Therefore, the perception-perfect equilibria is :

If $c \leq \dfrac{B}{k}$

\begin{align*}
(e_1^*, e_2^*) & = (1,e_2) \\
\text{where } & e_2=\begin{cases}
2 & \text{if } e_1=0 \\
1 & \text{if } e_1=1 \\
0 & \text{if } e_1=2
\end{cases}
\end{align*}

If $c > \dfrac{B}{k}$

\begin{align*}
(e_1^*, e_2^*) & = (1,e_2) \\
\text{where } & e_2=\begin{cases}
1 & \text{if } e_1=1 \\
0 & \text{if } e_1=0, 2
\end{cases}
\end{align*}

\end{itemize}

\dotfill

\item 

\begin{itemize}

\item Period 2

The intertemporal utility is :

\[ -c(e_2) + \beta B(e_1,e_2) \]

\begin{itemize}

\item When $e_1=0$

For $t=2$ agent, $e_2=1$ is strictly dominated by $e_2=0$, since both of them result in $B(0, e_2)=0$.

Then, the agent prefer $e_2=2$ iff

\begin{align*}
- kc + \beta B & \geq 0 \\
c & \leq \dfrac{\beta B}{k}
\end{align*}

Otherwise, $e_2=0$ is chosen.

\item When $e_1=1$

$e_2=2$ is strictly dominated by $e_2=1$, since $e_2=1$ is enough to satisfy $B(.,.)=B$.

The agent prefer $e_2=1$ iff

\begin{align*}
-c + \beta B & \geq 0 \\
c & \leq \beta B
\end{align*}

\end{itemize}

\newpage

\item Period 1

\begin{itemize}

\item When $c \leq \dfrac{\beta B}{k}$

Even if $e_1=0$, $t=2$ agent choose $e_2=2$ and so $B(e_1, e_2)=B$ realizes.

Then, $t=1$ agent prefer $e_1=2$ to $e_1=0$ iff

\begin{align*}
(1-\beta)kc \leq 0
\end{align*}

which never hold since $\beta<1$, so $e_1=2$ is stricly dominated by $e_1=0$.

Then, $e_1=1$ is preferred iff

\begin{align*}
-c - \beta c + \beta B &\geq - kc + \beta B \\
(1+\beta - \beta k)c &\leq 0 \\
\text{ which holds iff }
k &\geq \dfrac{1+\beta}{\beta}
\end{align*}

Otherwise, $e_1=0$ is chosen.

\item When $\dfrac{\beta B}{k} < c \leq \beta B $

$e_2=2$ is never chosen, which implies that if $e_1=0,B(.,.)=0$.

$t=1$ agent prefer $e_1=2$ to $e_1=0$ iff

\begin{align*}
- kc + \beta B &\geq 0 \\
c &\leq \dfrac{\beta B}{k}
\end{align*}

which never holds, by $c>\dfrac{\beta B}{k}$.

$e_1=1$ is preferred to $e_1=0$ iff

\begin{align*}
-c - \beta c + \beta B &\geq 0 \\
c & \leq \dfrac{\beta B}{1+\beta}
\end{align*}

Note that $\left( \dfrac{\beta B}{k}, \dfrac{\beta B}{1+\beta} \right]$ is not empty interval.

\item $\beta B < c$

Both $t=1,2$ agent choose $e_t=0$, rather than any positive $e_t$.

\end{itemize}

\newpage

Therefore, the perception-perfect equilibria is:

\[(e_1^*, e_2^*)=(e_1,e_2) \]

If $(c \leq \frac{\beta B}{k}) \wedge (k \geq \frac{1+\beta}{\beta})$,

\begin{align*}
e_1 &=1 \\
e_2 &=\begin{cases}
2 & \text{if} e_1=0 \\
1 & \text{if } e_1=1 \\
0 & \text{otherwise}
\end{cases}
\end{align*}

If $(c \leq \frac{\beta B}{k}) \wedge (k < \frac{1+\beta}{\beta})$, 

\begin{align*}
e_1 &=0 \\
e_2 &=\begin{cases}
2 & \text{if } e_1=0 \\
0 & \text{otherwise}
\end{cases}
\end{align*}

If $c \in (\frac{\beta B}{k}, \frac{\beta B}{1+\beta})$,

\begin{align*}
e_1 &=1 \\
e_2 &=\begin{cases}
1 & \text{if } e_1=1 \\
0 & \text{otherwise}
\end{cases}
\end{align*}

If $\beta B < c$,

\begin{align*}
e_1=0 \\
e_2=0
\end{align*}

\end{itemize}

\dotfill

\newpage

\item

$t=2$ agent decides her/his behavior with $\beta<1$, while $t=1$ agent predicts it with $\Hat{\beta}=1$. Then, the agent's prediction and the acual strategy depending on the parameters are summerized as follows :


\begin{center}

 \begin{tabular}{cccc} \hline
 \multicolumn{2}{c}{Parametrical condition} & & \\
 \multicolumn{1}{c}{Belief} & \multicolumn{1}{c}{Actual choice} & $e_1$ & BR$(e_1)$ \\ \hline
 $c \leq \dfrac{B}{k}$ & $c \leq \dfrac{\beta B}{k}$ & 0 &2 \\
  & & 1 & 1 \\
  & & 2 & 0 \\
 $\dfrac{B}{k} < c \leq B$ & $\dfrac{\beta B}{k} < c \leq \beta B$ & 1 &1 \\
  & & 0,2 & 0 \\
 $B < c$ & $\beta B < c$ & any & 0 \\ \hline
 \end{tabular}

\end{center}

\vspace{1zw}

If $k \leq \frac{1}{\beta}$ holds, then $\frac{B}{k} \leq \beta B$ ; otherwise, $\frac{B}{k} \leq \beta B$.

\begin{itemize}

\item $k \geq \frac{1}{\beta}$

 \begin{enumerate}
 
 \item $c \leq \dfrac{\beta}{k}$
 
 Same as (b), $t=1$ agent choose $e_1=1$ with belief that $t=2$ agent choose the following strategy
 
 \[e_2 = \begin{cases}
 2 & \text{if } e_1=0 \\
 1 & \text{if } e_1=1 \\
 0 & \text{if } e_1=2
 \end{cases} \]
 
 iff
 
 \[ k \geq \dfrac{1+\beta}{\beta} \]
 
 and by $c \leq \dfrac{\beta}{k}$, the strategy above is realized as perception-perfect equilibrium.
 
 If $c > \dfrac{\beta}{k}$, then $t=1$ agent choose $e_1=0$, and the strategy of $t=2$ agent is same, and this is perception-perfect equilibrium.
 
 \item $ \dfrac{\beta B}{k} < c \leq \dfrac{B}{k} $
 
 By $c < \dfrac{B}{k}$, $t=1$ agent predicts that $t=2$ agent choose the following strategy :
 
 \[e_2 = \begin{cases}
 2 & \text{if } e_1=0 \\
 1 & \text{if } e_1=1 \\
 0 & \text{if } e_1=2
 \end{cases} \]
 
 The realized strategy of $t=2$ agent, however, is below :
 
 \[e_2 = \begin{cases}
  1 & \text{if } e_1=1 \\
 0 & \text{if } e_1=0,2
 \end{cases} \]
 
 Thus, there does not exist any perception-perfect equilibrium.
 
 \item $\dfrac{B}{k} < c \leq \beta B$
 
 $t=1$ agent predicts $t=2$ agent's strategy as follows, and it will be realized.
 
  \[e_2 = \begin{cases}
  1 & \text{if } e_1=1 \\
 0 & \text{if } e_1=0,2
 \end{cases} \]
 
 $(e_1=0) \succsim (e_1=2)$, since
 
  \begin{align*}
  0 \geq -kc + \beta B \\
  c \geq \dfrac{\beta B}{k}
  \end{align*}
 
 and $ \dfrac{B}{k} > \dfrac{\beta B}{k}$.
 
 $t=1$ agent prefer $e_1=1$ to $e_1=0$ iff
 
  \begin{align*}
  -c - \beta c + B \geq 0 \\
  c \leq \dfrac{B}{1+\beta}
  \end{align*}
  
  $\dfrac{B}{1 + \beta} < \beta B$ if $\dfrac{1}{1 + \beta} < \beta$.
  
  \begin{enumerate}
  
  \item $\dfrac{1}{1 + \beta} < \beta$
  
  The perception-perfect equilibrium depends on the parameter.
  
  If $\dfrac{B}{k} < c \leq \dfrac{B}{1+\beta}$ ,
  
  $t=1$ agent choose $e_1=1$, and $e_2 = \begin{cases}
  1 & \text{if } e_1=1 \\
  0 & \text{if } e_1=0,2
  \end{cases}$.
  
  Then, if $\dfrac{B}{1+\beta} < c \leq \beta B$,
  
  $e_1=0$ and $e_2 = \begin{cases}
  1 & \text{if } e_1=1 \\
  0 & \text{if } e_1=0,2
  \end{cases}$.
  
  \item $\dfrac{1}{1 + \beta} \geq \beta$
  
  Then, for $\dfrac{B}{k} < c \leq \beta B$,
  
   $t=1$ agent choose $e_1=1$, and $e_2 = \begin{cases}
  1 & \text{if } e_1=1 \\
  0 & \text{if } e_1=0,2
  \end{cases}$.
 
  
  \end{enumerate}
 
 \newpage
 
 \item $\beta B < c \leq \dfrac{B}{2}$
 
 $t=1$ agent predicts $t=2$ strategy as :
 
  \[e_2 = \begin{cases}
  1 & \text{if } e_1=1 \\
 0 & \text{if } e_1=0,2
 \end{cases} \]
 
 While the realized strategy is $e_2=0$ for any $e_1$. Thus, there is no perception-perfect equilibrium.
 
 \end{enumerate}

\item $k<\dfrac{1}{\beta}$

 \begin{enumerate}
 
 \item $c \leq \dfrac{\beta B}{k}$
 
 $t=1$ agent's belief :
 
 \[e_2 = \begin{cases}
 2 & \text{if } e_1=0 \\
 1 & \text{if } e_1=1 \\
 0 & \text{if } e_1=2
 \end{cases} \]
 
 which realizes in period 2. Then, the perception-perfect equilibrium is :
 
 \begin{center}
 
 $e_1=1$ and $e_2 = \begin{cases}
 2 & \text{if } e_1=0 \\
 1 & \text{if } e_1=1 \\
 0 & \text{if } e_1=2
 \end{cases} $. if $ k \geq \dfrac{1+\beta}{\beta}$
 
 $e_1=0$ and $e_2 = \begin{cases}
 2 & \text{if } e_1=0 \\
 1 & \text{if } e_1=1 \\
 0 & \text{if } e_1=2
 \end{cases} $. if $ k < \dfrac{1+\beta}{\beta}$
 
 \end{center}
 
 \item $\dfrac{\beta B}{k} < c \leq \beta B$
 
 $t=1$ agent's belief is
 \[e_2 = \begin{cases}
 2 & \text{if } e_1=0 \\
 1 & \text{if } e_1=1 \\
 0 & \text{if } e_1=2
 \end{cases} \]
 
 while $t=2$ agent's strategy is
 
 \[e_2 = \begin{cases}
 1 & \text{if } e_1=1 \\
 0 & \text{if } e_1=0,2
 \end{cases} \]
 
 Thus, there is no perception-perfect equilibrium.
 
 \newpage
 
 \item $\beta B < c \leq \dfrac{\beta B}{k}$
 
 $t=1$ agent predicts $t=2$ agent's strategy:
 
 \[e_2 = \begin{cases}
 2 & \text{if } e_1=0 \\
 1 & \text{if } e_1=1 \\
 0 & \text{if } e_1=2
 \end{cases} \]
 
 $t=2$ agent's strategy is, however, 
 
 \[e_2=0 \text{ for any } e_1 \]
 
 Again, no perception-perfect equilibrium occurs.
 
 \item $\dfrac{\beta B}{k} < c \leq \dfrac{B}{2}$
 
 $t=1$ agent's belief :
 
 \[e_2 = \begin{cases}
 1 & \text{if } e_1=1 \\
 0 & \text{if } e_1=0,2
 \end{cases} \]
 
 while $e_2=0$ for any $e_1$ in period 2, which indicates there is no perception-perfect equilibrium.
 
  \end{enumerate}

\end{itemize}

\dotfill

\item 

$t=2$ agent decides her/his behavior with $\beta<1$, while $t=1$ agent predicts it with $\Hat{\beta} \in (\beta, 1)$. Then, the agent's prediction and the acual strategy depending on the parameters are summerized as follows :

\vspace{1.5zw}

\begin{center}

 \begin{tabular}{cccc} \hline
 \multicolumn{2}{c}{Parametrical condition} & & \\
 \multicolumn{1}{c}{Belief} & \multicolumn{1}{c}{Actual choice} & $e_1$ & BR$(e_1)$ \\ \hline
 $c \leq \dfrac{\Hat{\beta} B}{k}$  & $c \leq \dfrac{\beta B}{k}$ & 0 &2 \\
  & & 1 & 1 \\
  & & 2 & 0 \\
 $\dfrac{\Hat{\beta} B}{k} < c \leq \Hat{\beta} B$ & $\dfrac{\beta B}{k} < c \leq \beta B$ & 1 &1 \\
  & & 0,2 & 0 \\
 $\Hat{\beta} B < c$ & $\beta B < c$ & any & 0 \\ \hline
 \end{tabular}

\end{center}

\vspace{1zw}

If $k \geq \dfrac{\Hat{\beta}}{\beta}$, then $\beta B \geq \dfrac{\Hat{\beta}B}{k}$ ; otherwise, $\beta B < \dfrac{\Hat{\beta}B}{k}$.

\newpage

 \begin{itemize}
 
 \item $k \geq \dfrac{\Hat{\beta}}{\beta}$
 
  \begin{enumerate}
  
  \item $c \leq \dfrac{\beta B}{k}$
  
 $t=1$ agent's belief is
 
 \[e_2 = \begin{cases}
 2 & \text{if } e_1=0 \\
 1 & \text{if } e_1=1 \\
 0 & \text{if } e_1=2
 \end{cases} \]
 
 and by $c \leq \dfrac{\beta B}{k}$, her/his belief is true.
 
 $t=1$ agent choose $e_1=1$ iff $ k \geq \dfrac{1+\Hat{\beta}}{\Hat{\beta}}$, and otherwise $e_1=0$, both of them being the perception-perfect equilibrium in each case.
  
  \item $\dfrac{\beta B}{k} < c \leq \dfrac{\Hat{\beta}B}{k}$
  
  By $c < \dfrac{B}{k}$, $t=1$ agent predicts that $t=2$ agent choose the following strategy :
 
 \[e_2 = \begin{cases}
 2 & \text{if } e_1=0 \\
 1 & \text{if } e_1=1 \\
 0 & \text{if } e_1=2
 \end{cases} \]
 
 The realized strategy of $t=2$ agent, however, is below :
 
 \[e_2 = \begin{cases}
  1 & \text{if } e_1=1 \\
 0 & \text{if } e_1=0,2
 \end{cases} \]
 
 Thus, there does not exist any perception-perfect equilibrium.
   
  \item $\dfrac{\Hat{\beta }B}{k} < c \leq \beta B$
  
   $t=1$ agent predicts $t=2$ agent's strategy as follows, and it will be realized.
 
  \[e_2 = \begin{cases}
  1 & \text{if } e_1=1 \\
 0 & \text{if } e_1=0,2
 \end{cases} \]
 
 $(e_1=0) \succsim (e_1=2)$, since
 
  \begin{align*}
  0 \geq -kc + \beta B \\
  c \geq \dfrac{\beta B}{k}
  \end{align*}
 
 and $ \dfrac{B}{k} > \dfrac{\beta B}{k}$.
 
 \newpage
 
 $t=1$ agent prefer $e_1=1$ to $e_1=0$ iff
 
  \begin{align*}
  -c - \beta c + B \geq 0 \\
  c \leq \dfrac{B}{1+\beta}
  \end{align*}
  
  Note that $\dfrac{\beta B}{k} < \dfrac{\Hat{\beta} B}{k}$.
  
  $\dfrac{B}{1 + \Hat{\beta}} < \beta B$ if $\dfrac{1}{1 + \Hat{\beta}} < \beta$.
  
  \vspace{1.5zw}
  
  \begin{enumerate}
  
  \item $\dfrac{1}{1 + \Hat{\beta}} < \beta$
  
  The perception-perfect equilibrium depends on the parameter.
  
  If $\dfrac{B}{k} < c \leq \dfrac{B}{1+\Hat{\beta}}$ ,
  
  $t=1$ agent choose $e_1=1$, and $e_2 = \begin{cases}
  1 & \text{if } e_1=1 \\
  0 & \text{if } e_1=0,2
  \end{cases}$.
  
  Then, if $\dfrac{B}{1+\Hat{\beta}} < c \leq \beta B$,
  
  $e_1=0$ and $e_2 = \begin{cases}
  1 & \text{if } e_1=1 \\
  0 & \text{if } e_1=0,2
  \end{cases}$.
  
  \vspace{1.5zw}
  
  \item $\dfrac{1}{1 + \Hat{\beta}} \geq \beta$
  
  Then, for $\dfrac{\Hat{\beta}B}{k} < c \leq \beta B$,
  
   $t=1$ agent choose $e_1=1$, and $e_2 = \begin{cases}
  1 & \text{if } e_1=1 \\
  0 & \text{if } e_1=0,2
  \end{cases}$.
 
  
  \end{enumerate}
  
  \item $\beta B < c \leq \Hat{\beta} B $
  
  $t=1$ agent predicts $t=2$ strategy as :
 
  \[e_2 = \begin{cases}
  1 & \text{if } e_1=1 \\
 0 & \text{if } e_1=0,2
 \end{cases} \]
 
 While the realized strategy is $e_2=0$ for any $e_1$. Thus, there is no perception-perfect equilibrium.
 
  \item $\Hat{\beta} B < c$
  
  $t=1$ agent's belief is $e_2=0$ for any $e_1$, which is true. Therefore, there exists perception-perfect equilibrium s.t.
  
  \[e_1=0, \text{ and } e_2=0 \text{ for all }e_1 \]
  
  \end{enumerate}
 
 \newpage
 
 \item $k < \dfrac{\Hat{\beta}}{\beta}$
 
 \begin{enumerate}
  
  \item $c \leq \dfrac{\beta B}{k}$
  
  Conditions are exactly same as i. in $k \geq \dfrac{\Hat{\beta}}{\beta}$, so same equilibrium is derived.
  
  \item $\dfrac{\beta B}{k} < c \leq \beta B $
  
   $t=1$ agent's belief is
 \[e_2 = \begin{cases}
 2 & \text{if } e_1=0 \\
 1 & \text{if } e_1=1 \\
 0 & \text{if } e_1=2
 \end{cases} \]
 
 while $t=2$ agent's strategy is
 
 \[e_2 = \begin{cases}
 1 & \text{if } e_1=1 \\
 0 & \text{if } e_1=0,2
 \end{cases} \]
 
 Thus, there is no perception-perfect equilibrium.
  
  \item $\beta B < c \leq \dfrac{\Hat{\beta} B}{k}$
  
  $t=1$ agent predicts $t=2$ agent's strategy:
 
 \[e_2 = \begin{cases}
 2 & \text{if } e_1=0 \\
 1 & \text{if } e_1=1 \\
 0 & \text{if } e_1=2
 \end{cases} \]
 
 $t=2$ agent's strategy is, however, 
 
 \[e_2=0 \text{ for any } e_1 \]
 
 Again, no perception-perfect equilibrium occurs.
  
  \item $\dfrac{\Hat{\beta} B}{k} < c \leq \Hat{\beta}B$
  
   $t=1$ agent's belief :
 
 \[e_2 = \begin{cases}
 1 & \text{if } e_1=1 \\
 0 & \text{if } e_1=0,2
 \end{cases} \]
 
 while $e_2=0$ for any $e_1$ in period 2, which indicates there is no perception-perfect equilibrium.
 
  
  \item $\Hat{\beta}B < c$,
  
  which is same as v. in $k \geq \dfrac{\Hat{\beta}}{\beta}$, so again same equilibrium strategy derived.  
  
  \end{enumerate}
 
 \end{itemize}

\newpage

\item

 \begin{itemize}
 
 \item Period 2
 
 When $e_1=0$, then $t=2$ agent has no incentive to take any choice other than $e_2=0$, since $B(0,e_2)=0$ for any $e_2$.
 
 When $e_1=1$, $t=2$ agent prefer $e_2=1$ to $e_2=0$ iff
 
 \begin{align*}
 - c + \beta B &\geq 0 \\
 c & \leq \beta B
 \end{align*}
 
 otherwise, $e_2=0$ for any $e_1$
 
 \item Period 1
 
 $t=1$ agent predicts her/his behavior in period 2 with $\Hat{\beta} \in (\beta, 1)$.
 
 Then, the agent's prediction and the acual strategy depending on the parameters are summerized as follows :
 
  \begin{center}
  
  \begin{tabular}{cccc} \hline
 \multicolumn{2}{c}{Parametrical condition} & & \\
 \multicolumn{1}{c}{Belief} & \multicolumn{1}{c}{Actual choice} & $e_1$ & BR$(e_1)$ \\ \hline
 $c \leq \Hat{\beta}B$ & $c \leq \beta B$ & 1 & 1 \\
  & & 0 & 0 \\
 $\Hat{\beta} B < c $ & $\beta B <c$ & any & 0,2 \\ \hline
  \end{tabular}
  
  \end{center}
  
  \vspace{1zw}
 
 \end{itemize}

 \begin{enumerate}
 
 \item $c \leq \beta B$
 
 $t=1$ agent predicts her/his strategy as follows :
 
 \[e_2 = \begin{cases}
 1 & \text{if } e_1=1 \\
 0 & \text{otherwise}
 \end{cases} \]
 
 In the same way as (b), $e_1=2$ is dominated by $e_1=0$.
 
 $e_1=1$ is chosen iff $ k \geq \dfrac{1+\Hat{\beta}}{\Hat{\beta}}$ and then it will be the perceprion-perfect equilibrium.
 
 Otherwise, $ k < \dfrac{1+\Hat{\beta}}{\Hat{\beta}}$, $e_1=0$ and $e_2 = \begin{cases}
 1 & \text{if } e_1=1 \\
 0 & \text{otherwise}
 \end{cases} $
 
 is the perception-perfect equilibrium.
 
 \item $\beta B < c \leq \Hat{\beta} B$
 
 $t=1$ agent's belief is :
 
 \[e_2 = \begin{cases}
 1 & \text{if } e_1=1 \\
 0 & \text{otherwise}
 \end{cases} \]
 
 while, $t=2$ agent's strategy is $e_2=0$ for any $e_1$. Therefore, there exists no perceprion-perfect equilibrium.
 
 \item $\Hat{\beta} B < c < \dfrac{B}{2}$
 
 $t=1$ agent predicts $e_2=0$ for any $e_1$, which turn to be true.
 
 Thus, the perception-perfect equilibrium is
 
 \[e_1=0, \text{ and } e_2=0 \text{ for any }e_1 \]
 
 \end{enumerate}

\end{enumerate}

\end{enumerate}

\end{document}