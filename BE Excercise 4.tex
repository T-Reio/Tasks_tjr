\documentclass{jsarticle}
\usepackage{mathpazo}
\usepackage{amsmath,amssymb}
\usepackage{array}
\begin{document}

\title{Behavioral Economics \\
Exercise 4 Time Preferences}
\author{経済学研究科  \\ 23A18014 Reio TANJI 丹治伶峰}
\date{}
\maketitle

\begin{enumerate}

\item [Question 1]

\begin{enumerate}

\item

Solve by backward induction.

\begin{itemize}

\item Period 2

By $\beta = \delta = 1$, $t=2$ agent's intertemporal utility is :

\begin{align*}
-c(e_2) & + B(e_1,e_2) \\
\text{where } &  e_1 \text{ is given}
\end{align*}

Then, the best-response of the $t=2$ agent, depending on the parameters is derived as follows :

If $B \geq kc \Leftrightarrow c \leq \dfrac{B}{k}$

\[\text{BR}(e_1) = \begin{cases}
2 & \text{if } e_1=0 \\
1 & \text{if } e_1=1 \\
0 & \text{if } e_1=2
\end{cases} \]

and if $kc > B > 2c \Leftrightarrow c>\dfrac{B}{k}$

\[\text{BR}(e_1) = \begin{cases}
1 & \text{if } e_1=1 \\
0 & \text{if } e_1=0, 2
\end{cases}\]

\item Period 1

Now by $\beta = \delta = 1$, $t=1$ agent's intertemporal utility is :

\[ -c(e_1) - c(\text{BR}(e_1)) + B( e_1, \text{BR}(e_1)) \]

By the assumption $kc > 2c$ and $B>2c$ and the derived best-response of the $t=2$ agent, $e_1$ is the dominant strategy for each case of the parameters.

\newpage

Therefore, the perception-perfect equilibria is :

If $c \leq \dfrac{B}{k}$

\begin{align*}
(e_1^*, e_2^*) & = (1,e_2) \\
\text{where } & e_2=\begin{cases}
2 & \text{if } e_1=0 \\
1 & \text{if } e_1=1 \\
0 & \text{if } e_1=2
\end{cases}
\end{align*}

If $c > \dfrac{B}{k}$

\begin{align*}
(e_1^*, e_2^*) & = (1,e_2) \\
\text{where } & e_2=\begin{cases}
1 & \text{if } e_1=1 \\
0 & \text{if } e_1=0, 2
\end{cases}
\end{align*}

\end{itemize}

\item 

\begin{itemize}

\item Period 2

The intertemporal utility is :

\[ -c(e_2) + \beta B(e_1,e_2) \]

\begin{itemize}

\item When $e_1=0$

For $t=2$ agent, $e_2=1$ is strictly dominated by $e_2=0$, since both of them result in $B(0, e_2)=0$.

Then, the agent prefer $e_2=2$ iff

\begin{align*}
- kc + \beta B & \geq 0 \\
c & \leq \dfrac{\beta B}{k}
\end{align*}

Otherwise, $e_2=0$ is chosen.

\item When $e_1=1$

$e_2=2$ is strictly dominated by $e_2=1$, since $e_2=1$ is enough to satisfy $B(.,.)=B$.

The agent prefer $e_2=1$ iff

\begin{align*}
-c + \beta B & \geq 0 \\
c & \leq \beta B
\end{align*}

\end{itemize}

\newpage

\item Period 1

\begin{itemize}

\item When $c \leq \dfrac{\beta B}{k}$

Even if $e_1=0$, $t=2$ agent choose $e_2=2$ and so $B(e_1, e_2)=B$ realizes.

Then, $t=1$ agent prefer $e_1=2$ to $e_1=0$ iff

\begin{align*}
(1-\beta)kc \leq 0
\end{align*}

which never hold since $\beta<1$, so $e_1=2$ is stricly dominated by $e_1=0$.

Then, $e_1=1$ is preferred iff

\begin{align*}
-c - \beta c + \beta B &\geq - kc + \beta B \\
(1+\beta - \beta k)c &\leq 0 \\
\text{ which holds iff }
k &\geq \dfrac{1+\beta}{\beta}
\end{align*}

Otherwise, $e_1=0$ is chosen.

\item When $\dfrac{\beta B}{k} < c \geq \beta B $

$e_2=2$ is never chosen, which implies that if $e_1=0,B(.,.)=0$.

$t=1$ agent prefer $e_1=2$ to $e_1=0$ iff

\begin{align*}
- kc + \beta B &\geq 0 \\
c &\leq \dfrac{\beta B}{k}
\end{align*}

which never holds, by $c>\dfrac{\beta B}{k}$.

$e_1=1$ is preferred to $e_1=0$ iff

\begin{align*}
-c - \beta c + \beta B &\geq 0 \\
c & \leq \dfrac{\beta B}{1+\beta}
\end{align*}

Note that $\left( \dfrac{\beta B}{k}, \dfrac{\beta B}{1+\beta} \right]$ is not empty interval.

\item $\beta B < c$

Both $t=1,2$ agent choose $e_t=0$, rather than any positive $e_t$.

\end{itemize}

\newpage

Therefore, the perception-perfect equilibria is:

\[(e_1^*, e_2^*)=(e_1,e_2) \]

If $c \in (\frac{\beta B}{k}, \frac{\beta B}{1+\beta})$,

\begin{align*}
e_1 &=1 \\
e_2 &=\begin{cases}
1 & \text{if } e_1=1 \\
0 & \text{otherwise}
\end{cases}
\end{align*}

If $(c \leq \frac{\beta B}{k}) \wedge (k \geq \frac{1+\beta}{\beta})$,

\begin{align*}
e_1 &=1 \\
e_2 &=\begin{cases}
2 & \text{if} e_1=0 \\
1 & \text{if } e_1=1 \\
0 & \text{otherwise}
\end{cases}
\end{align*}

If $(c \leq \frac{\beta B}{k}) \wedge (k < \frac{1+\beta}{\beta})$, 

\begin{align*}
e_1 &=0 \\
e_2 &=\begin{cases}
2 & \text{if} e_1=0 \\
0 & \text{otherwise}
\end{cases}
\end{align*}

If $\beta B < c$,

\begin{align*}
e_1=0 \\
e_2=0
\end{align*}

\end{itemize}

\end{enumerate}

\end{enumerate}

\end{document}